
\documentclass{article}
\usepackage{amsmath}
\usepackage{graphicx}
\usepackage{booktabs}
\begin{document}
\begin{equation}
$Costa-$
\end{equation}procuranãoperdermuitoteespecificaçõescontratu\begin{equation}
$'ustos$
\end{equation}\begin{equation}
$etc.,$
\end{equation}\begin{equation}
$Opções:$
\end{equation}OperandoaVolatilidadetemporobjetivoú\begin{equation}
$'iação.$
\end{equation}QuasetodososElivrossobreopçõescomeçamcomum\section{E}\begin{equation}
$|$
\end{equation}\section{OPERAN}coscontratosedesuaimplementaçãoem\begin{equation}
\left( bolsa,\right)
\end{equation}incluindestõescomomarg\begin{equation}
\left( postos,\right)
\end{equation}\section{RT}\begin{equation}
$,$
\end{equation}prosselosdeprecificação\begin{equation}
\left( e,\right)
\end{equation}finalmente\begin{equation}
$*$
\end{equation}CésarLaurodarelacionandoestratégiasdenegociaçãlalidadedosconce\begin{equation}
$:$
\end{equation}deprecificação\begin{equation}
$tratados.$
\end{equation}\begin{equation}
$Opções:$
\end{equation}mpocominf\begin{equation}
\left( ais,\right)
\end{equation}\begin{equation}
€
\end{equation}nemcomexageradodandomaisênfaseà\begin{equation}
\left( precificação,\right)
\end{equation}\begin{equation}
$*$
\end{equation}\begin{equation}
—
\end{equation}baseadanavariabilic\begin{equation}
$aÇOS.$
\end{equation}estudantesdomercadofinanceiroeparaquem\begin{equation}
$quise:$
\end{equation}principalexportécnicasdenegocadefiniçãoclaraqu\begin{equation}
\left( gem,\right)
\end{equation}custosoperacionaiseimguindocommode\begin{equation}
»gociação
\end{equation}queutilizemaplenapotenciza\begin{equation}
$Jpções:$
\end{equation}Opereormaçõessobr\begin{equation}
\left( academicismo,\right)
\end{equation}ilidade dospreEindicadoaparticipardesse\begin{equation}
$mercado.$
\end{equation}\begin{equation}
- D0e - s_{4} + 7438
\end{equation}158N85\begin{equation}
$|$
\end{equation}\begin{equation}
$|$
\end{equation}917885741280eenESESAes\includegraphics[width=0.8\textwidth]{output/image_228png}DadosInternacionaisdeCatalogaçãonaPublicação\begin{equation}
CIP
\end{equation}\begin{equation}
$(Câmara$
\end{equation}Brasileirado\begin{equation}
\left( Livro,\right)
\end{equation}\begin{equation}
\left( SP,\right)
\end{equation}\begin{equation}
$Brasil)$
\end{equation}\begin{equation}
$|$
\end{equation}AP\section{RE}\section{SENTAÇÃO}\begin{equation}
\left( Costa,\right)
\end{equation}CésarLauroda\begin{equation}
$Opções:$
\end{equation}operandoavolatilidade\begin{equation}
$/César$
\end{equation}Lauroda\begin{equation}
$Costa.$
\end{equation}\begin{equation}
—
\end{equation}São\begin{equation}
$Paulo:$
\end{equation}BolsadeMercadorias\begin{equation}
$&$
\end{equation}\begin{equation}
\left( Futuros,\right)
\end{equation}\begin{equation}
1998.0
\end{equation}Desdeoiníciodesuas\begin{equation}
\left( atividades,\right)
\end{equation}emjaneirode\begin{equation}
\left( 1986,\right)
\end{equation}aBolsadeMercadorias\begin{equation}
$&$
\end{equation}FuturosvemincentivandoolançamentodefolhetosLE\begin{equation}
$.$
\end{equation}\begin{equation}
$.$
\end{equation}\begin{equation}
$.$
\end{equation}\begin{equation}
$.$
\end{equation}elivrosparadifundirosmercadosfuturosedeopçõesno\begin{equation}
$Brasil.$
\end{equation}\begin{equation}
$-$
\end{equation}Futurosfinanceiros\begin{equation}
2.0
\end{equation}Opções\begin{equation}
Finanças
\end{equation}Paraidauiriuváriostí\begin{equation}
$:$
\end{equation}L\begin{equation}
$Título.$
\end{equation}araissoadquiriuváriostítulosdeautores\begin{equation}
\left( estrangeiros,\right)
\end{equation}emquesãoapresentadasasexperiênciasvividasemlugaresemqueanegociaçãodederivativosatingiuextraordináriograude\begin{equation}
$sofística-$
\end{equation}\begin{equation}
$ção.$
\end{equation}Masnemporissoa\begin{equation}
BM \wedge F
\end{equation}seesqueceudedartotalapoioaosautores\begin{equation}
\left( nacionais,\right)
\end{equation}paraqueelesigualmentepudessemdeixar\begin{equation}
-2360
\end{equation}\begin{equation}
CDD - 332.645
\end{equation}registradosseusconhecimentossobreasmaisvariadastécnicaseestratégiasde\begin{equation}
\left( operação,\right)
\end{equation}\begin{equation}
- se + tornando
\end{equation}fontedeconsultaparaos\begin{equation}
$parti-$
\end{equation}cipantesdomercado\begin{equation}
$brasileiro.$
\end{equation}Olivro\begin{equation}
$Opções:$
\end{equation}OperandoaVolatilidadeéoresultadodeváriosanosde\begin{equation}
\left( pesquisa,\right)
\end{equation}comoautorprocurandomuitomaisexportécnicasdenegociaçãocomopçõesdoquepromoverumadiscussãocompletasobreseus\begin{equation}
$fundamentos.$
\end{equation}AobrasediferedamaioriadoslivrosquetratadeopçõesporqueISBN\begin{equation}
$85-7438-002-4$
\end{equation}nãosódefineoscontratosesuaimplementaçãoem\begin{equation}
\left( bolsa,\right)
\end{equation}fornecendomodelosdeprecificaçãoerelacionando\begin{equation}
\left( estratégias,\right)
\end{equation}comotambémdámaiorênfaseàsoperaçõesbaseadasnascaracterísticasdevariaçãoÍndicesparacatálogo\begin{equation}
$sistemático:$
\end{equation}\begin{equation}
1.0
\end{equation}\begin{equation}
$Opções:$
\end{equation}\begin{equation}
$Contratos:$
\end{equation}Mercado\begin{equation}
$financeiro:$
\end{equation}Economia\begin{equation}
332.645
\end{equation}dos\begin{equation}
$preços.$
\end{equation}BOLSADEMERCADORIAS\begin{equation}
$&$
\end{equation}FUTUROSCULTURAEDITORESASSOCIADOSApublicaçãodolivro\begin{equation}
$Opções:$
\end{equation}OperandoaVolatilidadecontribuidida\section{dora}\begin{equation}
“o
\end{equation}\section{as}losPparaodesenvolvimentodetécnicaseestruturas\begin{equation}
\left( operacionais,\right)
\end{equation}além\begin{equation}
$-901-$
\end{equation}São\begin{equation}
\left( Paulo,\right)
\end{equation}\begin{equation}
-240
\end{equation}\begin{equation}
—
\end{equation}São\begin{equation}
\left( Paulo,\right)
\end{equation}\begin{equation}
$:$
\end{equation}\begin{equation}
$.$
\end{equation}Vo\begin{equation}
4.0
\end{equation}Telefone\begin{equation}
$(011)$
\end{equation}\begin{equation}
1119
\end{equation}Telefax\begin{equation}
$(011)$
\end{equation}\begin{equation}
-4438
\end{equation}deserboaleituraparaquemquiserparticipardesse\begin{equation}
$mercado.$
\end{equation}Telefax\begin{equation}
$(011)$
\end{equation}\begin{equation}
-7333
\end{equation}\begin{equation}
$http://www.bmf.com.br$
\end{equation}\begin{equation}
$e-mail:$
\end{equation}\begin{equation}
$bmfQbmf.com.br$
\end{equation}ManoelFelixCintraNetoPresidentedaBolsadeMercadorias\begin{equation}
$&$
\end{equation}\begin{equation}
$Futuros-BM&F$
\end{equation}Janeirode1999\begin{equation}
$Impressão:$
\end{equation}5º4º3º2ºJº\begin{equation}
$Ano:$
\end{equation}0302010099\includegraphics[width=0.8\textwidth]{output/image_230png}INTRODUÇÃODelimitaçãoObjetivodolivroTerminologiaenotaçãoGráficosPREÇOSDAOPÇÕESValorintrínsicoeprêmioderiscoRelaçõesnecessáriasMODELOBLACKESCHOLESEquaçõesdiferenciaisModeloBlackeScholesSoluçãoeparâmetrosCríticasaomodeloCOMPORTAMENTODASOPÇÕESGeneralidadesParametrizaçãoEvoluçãodosparâmetrosnotempoPropriedadesdasposiçõesOPERAÇÕESCOMOPÇÕESGeneralidadesEspeculaçãoOperaçõesfinanceirasEstratégiasOPERAÇÕESDEVOLATILIDADEGeneralidadesBalançodoresultadoEstimativasdejurosevolatilidadeAcompanhamentodavolatilidadeimplícitaCarregamentodaposiçãoFledgedosjurosARBITRAGENSGeneralidadesArbitragementrevolatilidadesDescasamentos\section{ÍNDICE}\includegraphics[width=0.8\textwidth]{output/image_232png}\section{INTRODUÇÃO}DelimitaçãoConceitodeopçãoOconceitodeopçãonascecomoumdireitonegociáveldecompraouvendadeumativoaumpreçofuturo\begin{equation}
$predeterminado.$
\end{equation}\begin{equation}
\left( Nasce,\right)
\end{equation}\begin{equation}
\left( dizemos,\right)
\end{equation}\begin{equation}
\left( porque,\right)
\end{equation}apesardeestaseradefiniçãocorretae aessênciadoscontratosdeopçõesmais\begin{equation}
\left( simples,\right)
\end{equation}écadavezmenosútildefinirassimagamadeprodutosgeradosapartirdesta\begin{equation}
$base.$
\end{equation}Ofatodeserumdireitoimplicaqueapartetitularpossuiumaescolhapossível\begin{equation}
—
\end{equation}exercerounãoexercero\begin{equation}
$direito.$
\end{equation}\begin{equation}
\left( Contudo,\right)
\end{equation}nãohápraticamentenenhumtipodeopçãoem\begin{equation}
\left( que,\right)
\end{equation}dadaumasituaçãoeassumidaaracionalidadedo\begin{equation}
\left( titular,\right)
\end{equation}oresultadodaescolhanãoseja\begin{equation}
$conhecido.$
\end{equation}Isto\begin{equation}
\left( é,\right)
\end{equation}\begin{equation}
assumindo - se
\end{equation}queotitularéumagenteracionalquepreferemaisdinheiroamenos\begin{equation}
\left( dinheiro,\right)
\end{equation}na\begin{equation}
$-$
\end{equation}verdadenãoháescolhaalgumasobreo\begin{equation}
\left( exercício,\right)
\end{equation}eaopçãodeixade\begin{equation}
$represen-$
\end{equation}tarumaescolhapararepresentarumperfildefluxodecaixaaseratribuídoaotitularemalgumadata\begin{equation}
$futura.$
\end{equation}Esteperfilé\begin{equation}
\left( sempre,\right)
\end{equation}pelo\begin{equation}
\left( menos,\right)
\end{equation}umafunçãodeumpreçoSemumadata\begin{equation}
$qualquer.$
\end{equation}Podeserfunçãodeoutrascoisas\begin{equation}
$(como,$
\end{equation}demodomais\begin{equation}
\left( simples,\right)
\end{equation}deváriospreços\begin{equation}
$$1,$
\end{equation}\begin{equation}
$S2etc.$
\end{equation}combinadosdamaneiraquese\begin{equation}
$queira),$
\end{equation}massempreguardaumarelaçãoespecialparacomum\begin{equation}
\left( preço,\right)
\end{equation}doqualoprodutoé\begin{equation}
$derivativo.$
\end{equation}Esseperfiléchamadofunção\begin{equation}
\frac{payo}{fou}
\end{equation}\begin{equation}
$simples-$
\end{equation}mentepayoffda\begin{equation}
$opção.$
\end{equation}Comoexemplosde\begin{equation}
$payo/j;$
\end{equation}temosprimeiramenteos\begin{equation}
\left\lfloor{\frac{payo}{s}}\right\rfloor
\end{equation}deumaopçãodecompra\begin{equation}
C
\end{equation}eodeumaopçãodevenda\begin{equation}
7
\end{equation}\begin{equation}
$comuns.$
\end{equation}Essesfluxos\begin{equation}
$represen-$
\end{equation}tamovalordodireitodecomprar\begin{equation}
$(no$
\end{equation}casoda\begin{equation}
$cai?)$
\end{equation}oudevender\begin{equation}
$(no$
\end{equation}casoda\begin{equation}
$put)$
\end{equation}umativoporumpreçopredefinidoemumadata\begin{equation}
\left( predefinida,\right)
\end{equation}emfunçãodopreçoSdesteativonessadata\begin{equation}
$(0):$
\end{equation}\includegraphics[width=0.8\textwidth]{output/image_234png}to\begin{equation}
$Opções:$
\end{equation}OperandoaVolatilidadeUmsegundotipodepayojféaquelequepodeserobtidoapartirdecombinaçõesdeopções\begin{equation}
$comuns.$
\end{equation}Ascombinaçõesassimoriginadaspodemservistascomonovostiposdeopções\begin{equation}
$(8):$
\end{equation}\begin{equation}
\left( Finalmente,\right)
\end{equation}hápayofsquenãopodemsermontadoscomasopçõescomunsemquantidades\begin{equation}
\left( finitas,\right)
\end{equation}masquetambémrepresentamfunções\begin{equation}
$váli-$
\end{equation}\begin{equation}
$das:$
\end{equation}opçõestudoou\begin{equation}
\left( nada,\right)
\end{equation}por\begin{equation}
\left( exemplo,\right)
\end{equation}emqueosvaloresaseremrecebidospelotitularvariamem\begin{equation}
\left( patamares,\right)
\end{equation}ouopçõesquepossuemumabarreirano\begin{equation}
\left( exercício,\right)
\end{equation}isto\begin{equation}
\left( é,\right)
\end{equation}queperdemtodoovalorseopreçoS\begin{equation}
encontrar - se
\end{equation}além\begin{equation}
$(ou$
\end{equation}\begin{equation}
$dentro)$
\end{equation}decertoslimites\begin{equation}
$(9).$
\end{equation}Acaracterísticacomumdessasopçõeséqueopayolfédescontínuoempelomenosum\begin{equation}
\left( ponto,\right)
\end{equation}aparecendonográficocomoumareta\begin{equation}
$vertical.$
\end{equation}Essadescontinuidadepodeser\begin{equation}
\left( aproximada,\right)
\end{equation}no\begin{equation}
\left( limite,\right)
\end{equation}porcomposiçõesdeopçõescomunsemquantidades\begin{equation}
$infinitas:$
\end{equation}\begin{equation}
$Opções:$
\end{equation}OperandoaVolatilidade3VVTodososexemplosanterioressãodeopçõesquepodemser\begin{equation}
$completa-$
\end{equation}mentedefinidasemtermosdasfunçõespayofVxSnadatade\begin{equation}
$vencimento.$
\end{equation}Issonãoesgotatodasaspossibilidadesdosprodutosquehojesechamam\begin{equation}
$opções:$
\end{equation}háopçõesdasquaisopayojfnãopodeserdefinidoa\begin{equation}
\left( priori,\right)
\end{equation}poisdependedoqueaconteceduranteoperíodoatéo\begin{equation}
$vencimento.$
\end{equation}OpçãoamericanaDentre\begin{equation}
\left( esses,\right)
\end{equation}ocasomaissimpleséodeumaopçãodeestilo\begin{equation}
$americano.$
\end{equation}Umaopçãoamericanaéaquelaquecontratualmentepodeserexercidaemqualquerdataatéo\begin{equation}
$vencimento.$
\end{equation}Ooutroestilodeexercícioéo\begin{equation}
\left( europeu,\right)
\end{equation}quesópermiteoexercícionadatade\begin{equation}
$vencimento.$
\end{equation}Paraopçõesdecompraamericanassobreativosquenãoproporcionamrendasadicionaissuperioresàstaxas\begin{equation}
\left( dejuro,\right)
\end{equation}umresultadodizqueoexercícioantecipadonuncaémaisvantajosodoqueavendaereduzoproblemaaodasopções\begin{equation}
$européias.$
\end{equation}Masparaopçõesdevendaamericanassobreomesmotipode\begin{equation}
\left( ativo,\right)
\end{equation}certassituações\begin{equation}
$(notadamente$
\end{equation}aaltadetaxasde\begin{equation}
$juro)$
\end{equation}favorecemoexercício\begin{equation}
$antecipado.$
\end{equation}Avaliarumaput\begin{equation}
$(opção$
\end{equation}de\begin{equation}
$venda)$
\end{equation}americananessascircunstânciasnãopodeserfeitoporBlacke\begin{equation}
\left( Scholes,\right)
\end{equation}enecessitadeumoutro\begin{equation}
$modelo.$
\end{equation}Omesmovalepara\begin{equation}
\left\lfloor{\frac{ca}{s}}\right\rfloor
\end{equation}\begin{equation}
$(opções$
\end{equation}de\begin{equation}
$compra)$
\end{equation}americanassobreativosqueproporcionemrendasmaioresqueastaxasdejuro\begin{equation}
$vigentes.$
\end{equation}OpçãoasiáticaO\begin{equation}
talvez
\end{equation}segundocasomaissimpleséodeumaopçãopela\begin{equation}
\left( média,\right)
\end{equation}chamadacomumentede\begin{equation}
$asiática.$
\end{equation}Ovalorfinaldestaopçãoéigualàdiferençapositivaentreumpreçofixoeumamédiadepreçosdoativo\begin{equation}
5.0
\end{equation}\begin{equation}
$Aparentemen-$
\end{equation}\begin{equation}
\left( te,\right)
\end{equation}elasósediferedeumaopçãocomumpelofatodequeopreçoquedefinirá\includegraphics[width=0.8\textwidth]{output/image_236png}4\begin{equation}
$Opções:$
\end{equation}OperandoaVolatilidadeseuvalornovencimentoéumpreço\begin{equation}
\left( médio,\right)
\end{equation}enãoumpreço\begin{equation}
$final.$
\end{equation}\begin{equation}
\left( Contudo,\right)
\end{equation}essadiferençaimplicaumaseparaçãoradicalentreambosos\begin{equation}
$tipos.$
\end{equation}Aopçãopelamédianãoadmiteumpayoffconhecido\begin{equation}
$antecipadamente.$
\end{equation}Podemexistirinfinitoscaminhosatéummesmovalor\begin{equation}
$&$
\end{equation}\begin{equation}
\left( final,\right)
\end{equation}cadaumdelescomumamédia\begin{equation}
$diferente.$
\end{equation}Por\begin{equation}
\left( exemplo,\right)
\end{equation}seemcincodiasohistóricodopreçodeumativofor\begin{equation}
\left( 100, \  101\right)
\end{equation}\begin{equation}
\left( 105,\right)
\end{equation}\begin{equation}
\left( 103, \  101\right)
\end{equation}suamédiafinalseráde\begin{equation}
102.0
\end{equation}Seestemesmo\begin{equation}
\left( ativo,\right)
\end{equation}emcinco\begin{equation}
\left( dias,\right)
\end{equation}exibirohistórico\begin{equation}
\left( 100, \  99, \  97, \  98, \  101\right)
\end{equation}teráumpreçomédiode\begin{equation}
\left( 99,\right)
\end{equation}apesardeterencerradooquintodianomesmo\begin{equation}
$preço.$
\end{equation}Aquiháqueseabrirumparênteseparaopçõespelamédia\begin{equation}
$geométrica.$
\end{equation}Devidoaumapropriedade\begin{equation}
\left( matemática,\right)
\end{equation}todocaminhoqueresultaremummesmonúmerofinalpossuiamesmamédia\begin{equation}
$geométrica.$
\end{equation}\begin{equation}
\left( Portanto,\right)
\end{equation}opçõessobreamédiageométrica\begin{equation}
—
\end{equation}enãoaritmética\begin{equation}
—
\end{equation}possuemdefatoum\begin{equation}
\frac{payo}{ffixo}
\end{equation}e\begin{equation}
$conhecido.$
\end{equation}OpçãodebarreiraComoúltimoexemplodeopçõesquenãotêmum\begin{equation}
$payof'tixo,$
\end{equation}podemoscitaropçõesdebarreira\begin{equation}
$(Anock-out$
\end{equation}ou\begin{equation}
$knock-in).$
\end{equation}Asopçõestipo\begin{equation}
fxock - out
\end{equation}sãoextintasnocasodealgumeventoocorrerduranteoprazodaopçãoouduranteumperíododefinidoentreduas\begin{equation}
$datas.$
\end{equation}Esteeventogeralmenteéoativoatingirumdeterminado\begin{equation}
\left( preço,\right)
\end{equation}chamadopreçode\begin{equation}
$barreira.$
\end{equation}Asopções\begin{equation}
$Anock-in$
\end{equation}\begin{equation}
$inici-$
\end{equation}almentenão\begin{equation}
\left( existem,\right)
\end{equation}epassamavalerapenasseumdeterminadoevento\begin{equation}
$ocorrer.$
\end{equation}Nocasodeumaopção\begin{equation}
\left( Anock - out,\right)
\end{equation}\begin{equation}
pode - se
\end{equation}devolveraocompradoralgumvalordiferentedezeronocasoemqueaopçãoé\begin{equation}
\left( extinta,\right)
\end{equation}eestevaloréchamadode\begin{equation}
$rebate.$
\end{equation}OpçãoexplosivaExisteumaclassificaçãoquedistinguetambémopçõesexplosivas\begin{equation}
$(exp/odizg$
\end{equation}\begin{equation}
$options),$
\end{equation}quedãoexercícioantecipadoobrigatóriodiantedeum\begin{equation}
\left( evento,\right)
\end{equation}masessasopçõespodemservistascomocasosparticularesdeopções\begin{equation}
$Anock-out.$
\end{equation}Noscasosexemplificadosanteriormente\begin{equation}
$(e$
\end{equation}emvários\begin{equation}
$outros),$
\end{equation}afunçãoVx\begin{equation}
$$$
\end{equation}temdesercondicionadaatodaavidada\begin{equation}
$opção.$
\end{equation}Éimpossívelrepresentarcomapenasumpyojftodasaspossibilidadesecondicionantesdepreçoatéo\begin{equation}
$vencimento.$
\end{equation}OpçõesvanillaxopçõesexóticasParaoescopodeste\begin{equation}
\left( livro,\right)
\end{equation}pelomenosatéopenúltimo\begin{equation}
\left( capítulo,\right)
\end{equation}\begin{equation}
$restringi-$
\end{equation}remosageneralidadeàquelestiposvistosemprimeirolugar\begin{equation}
$(8,$
\end{equation}Oe\begin{equation}
$8),$
\end{equation}comta\begin{equation}
$Opções:$
\end{equation}OperandoaVolatilidadefoconasopçõestipo\begin{equation}
\left( 1,\right)
\end{equation}asquaischamaremosdeopçõesvanilla\begin{equation}
$européias.$
\end{equation}Otermocamila\begin{equation}
baunilha
\end{equation}designaotipocomumde\begin{equation}
$opção.$
\end{equation}\begin{equation}
$|$
\end{equation}Oopostodevanillaé\begin{equation}
$exótico.$
\end{equation}Nosúltimos\begin{equation}
\left( anos,\right)
\end{equation}afronteiraentreoqueéexóticoeoquenãoé\begin{equation}
- se + vem
\end{equation}\begin{equation}
\left( modificando,\right)
\end{equation}conformeostiposmaissofisticadosvãoficandomais\begin{equation}
$corriqueiros.$
\end{equation}Umacaracterísticacomumaquasetodasasopções\begin{equation}
\left( exóticas,\right)
\end{equation}no\begin{equation}
\left( entanto,\right)
\end{equation}éainexistênciadeumpayojffixodefinidoparaumadatafixa\begin{equation}
$(a$
\end{equation}datade\begin{equation}
$vencimento).$
\end{equation}\begin{equation}
Path - dependence
\end{equation}\begin{equation}
“
\end{equation}Ostrêsexemplosanterioresdeopçõesexóticas\begin{equation}
$(opções$
\end{equation}\begin{equation}
\left( americanas,\right)
\end{equation}asiáticasede\begin{equation}
$barreira)$
\end{equation}\begin{equation}
referem - se
\end{equation}aopções\begin{equation}
\left( - \frac{pat}{dependernt!},\right)
\end{equation}querdizer\begin{equation}
$depen-$
\end{equation}dentesdo\begin{equation}
$caminho.$
\end{equation}CaminhoaquiéopercursoqueopreçoSfaráatéadatade\begin{equation}
$vencimento.$
\end{equation}Elastêmdeserespecificadaspeloseventosaquesão\begin{equation}
\left( sensíveis,\right)
\end{equation}alémdacurvaVxS\begin{equation}
$(às$
\end{equation}\begin{equation}
\left( vezes,\right)
\end{equation}acurvaéomenosimportantede\begin{equation}
$tudo).$
\end{equation}Assimcomoasopções\begin{equation}
\left( varnilla,\right)
\end{equation}as\begin{equation}
- dependent + patl
\end{equation}podemser\begin{equation}
\left( precificadas,\right)
\end{equation}admitemocálculodetaxasdehedge\begin{equation}
$(deltas,$
\end{equation}vegas\begin{equation}
$etc.)$
\end{equation}epodemparticipardetooksjuntoaoutrasopções\begin{equation}
$quaisquer.$
\end{equation}Oqueasdiferenciadasvanillanãoé\begin{equation}
\left( isto,\right)
\end{equation}masaaplicabilidadedotratamentomatemáticoqueseráaqui\begin{equation}
$apresentado.$
\end{equation}TratamentomatemáticoOtipodetratamentomatemáticoaquenosreferimosanteriormenteseresumeno\begin{equation}
$seguinte:$
\end{equation}\begin{equation}
$a)$
\end{equation}modelaraspropriedadesdasvariaçõesdepreçodasopçõesdemodoachegaraequaçõesválidasparaquaisquer\begin{equation}
$condições;$
\end{equation}\begin{equation}
$b)$
\end{equation}encontrarsoluçõesanalíticasparaas\begin{equation}
\left( equações,\right)
\end{equation}isto\begin{equation}
\left( é,\right)
\end{equation}fórmulasqueditemovalorVdeumaopçãoemqualquer\begin{equation}
\left( instante,\right)
\end{equation}sobquaisquercondiçõesquesejam\begin{equation}
$dadas;$
\end{equation}\begin{equation}
$c)$
\end{equation}derivarafórmuladasoluçãoanalíticaemrelaçãoacadaumdosfatoresqueinfluenciemovalorda\begin{equation}
\left( opção,\right)
\end{equation}demodoaseterumespelhodaexposiçãoquesetemacada\begin{equation}
$fator.$
\end{equation}\begin{equation}
\left( Notadamente,\right)
\end{equation}osfatoresque nos\begin{equation}
$interessa-$
\end{equation}rãosãotaxasde\begin{equation}
\left( juro,\right)
\end{equation}\begin{equation}
\left( prazos,\right)
\end{equation}oprópriomercadodeS\begin{equation}
—
\end{equation}sãofatoresde\begin{equation}
$mercado.$
\end{equation}Umaposiçãoemopçõeséumaposiçãosimultâneaemvários\begin{equation}
\left( mercados,\right)
\end{equation}eterámaissucessoemlidarcomelaquemconseguirsepararcomprecisãoosefeitoseexposiçõesacadaum\begin{equation}
$deles.$
\end{equation}\begin{equation}
\left( e,\right)
\end{equation}\begin{equation}
\left( finalmente,\right)
\end{equation}\begin{equation}
$d)$
\end{equation}conhecercomoessesfatoresevoluemdeformadinâmicaeemrelaçãounsaos\begin{equation}
$outros.$
\end{equation}Tudooqueforexpostonestelivrovaleráparaasopçõesdecompra\begin{equation}
$(ca/)s)$
\end{equation}edevenda\begin{equation}
pyts
\end{equation}vanilla\begin{equation}
0
\end{equation}\begin{equation}
$européias.$
\end{equation}Asopçõesimediatamentemaisgenéricas\begin{equation}
8
\end{equation}admitemamesmadiscussão\begin{equation}
\left( matemática,\right)
\end{equation}obedecemàmesmaequaçãodiferencialenãosãomaisquecombinaçõesdasopçõesdoprimeiro\includegraphics[width=0.8\textwidth]{output/image_238png}\begin{equation}
$Opções:$
\end{equation}OperandoaVolatilidade\begin{equation}
$tipo.$
\end{equation}\begin{equation}
\left( Porém,\right)
\end{equation}nãosepodedizerquetudooquevaleparaumavanillavaleparauma\begin{equation}
$combinação.$
\end{equation}Por\begin{equation}
$exemplo:$
\end{equation}odeltadeuma\begin{equation}
cai!
\end{equation}vaníliaéumnúmeroentre0e\begin{equation}
\left( 1, \  0\right)
\end{equation}delta deumacombinaçãopodeserumnúmero\begin{equation}
$qualquer. Neste$
\end{equation}\begin{equation}
\left( livro,\right)
\end{equation}asopçõestipo2figurarãocomo\begin{equation}
\left( posições,\right)
\end{equation}combinaçõesou\begin{equation}
$estratégias.$
\end{equation}Àsopçõesdaterceiraespécie\begin{equation}
8
\end{equation}tambémseenquadramnamesmadiscussão\begin{equation}
\left( matemática,\right)
\end{equation}tambémobedecemàmesmaequação\begin{equation}
\left( diferencial,\right)
\end{equation}masnecessitamdeumdesenvolvimentoespecíficoparaasfórmulasdecada\begin{equation}
\left( caso,\right)
\end{equation}enãopodemsertratadascomocombinaçõesde\begin{equation}
\left\lfloor{\frac{ca}{s}}\right\rfloor
\end{equation}eputs\begin{equation}
\left( vanilla,\right)
\end{equation}anãoserno\begin{equation}
$limite.$
\end{equation}Comonãosãomuito\begin{equation}
\left( frequentes,\right)
\end{equation}nãovalea\begin{equation}
\left( pena,\right)
\end{equation}senãoatítulode\begin{equation}
\left( curiosidade,\right)
\end{equation}conhecerasfórmulasrelativasa\begin{equation}
$elas.$
\end{equation}OpçõessobremaisdeumativoÉprecisoquesedigaquenemtudoqueéexóticoé\begin{equation}
$pati-dependent.$
\end{equation}Durantemuito\begin{equation}
\left( tempo,\right)
\end{equation}opçõestudoou\begin{equation}
\left( nada,\right)
\end{equation}\begin{equation}
\left( binárias,\right)
\end{equation}comoasdotipoelementar\begin{equation}
$&$
\end{equation}foramconsideradas\begin{equation}
$exóticas.$
\end{equation}\begin{equation}
\left( Igualmente,\right)
\end{equation}umaopçãosobredoisativos\begin{equation}
$(em$
\end{equation}quesedáodireitoaocompradordeescolheramelhorentreduas\begin{equation}
$rentabilida-$
\end{equation}des\begin{equation}
$variáveis)$
\end{equation}éumaopçãodita\begin{equation}
\left( exótica,\right)
\end{equation}massuaestruturaéadeumavaníllaemqueopayojféumafunçãodeduasvariáveisemvezde\begin{equation}
$uma.$
\end{equation}Assimcomoa\begin{equation}
\left( vanilia,\right)
\end{equation}aopçãosobredoisativosnãoé\begin{equation}
\left( - dependent + patli,\right)
\end{equation}obedeceàequaçãodiferencialdeBlackeScholesepodeseranalisadapelomesmoferramental\begin{equation}
$matemático.$
\end{equation}Adificuldadeemlidarcomelaéaintromissãodeumfatorpouco\begin{equation}
\left( conhecido,\right)
\end{equation}poucoestávelepoucoprevisível\begin{equation}
—
\end{equation}acorrelaçãoentreas\begin{equation}
$rentabili-$
\end{equation}dadesdosativos\begin{equation}
—
\end{equation}nosproblemas\begin{equation}
$relevantes.$
\end{equation}Precisamentepor\begin{equation}
\left( isso,\right)
\end{equation}aopçãosobremaisdeumativoéconsideradasofisticadaeincluídanogrupodas\begin{equation}
$exóticas.$
\end{equation}ObjetivodolivroModelosoperacionaisSendooobjetivodestelivromuitomaisexportécnicasdenegociaçãocomopçõesdoquepromoverumadiscussãocompletasobre\begin{equation}
\left( fundamentos,\right)
\end{equation}\begin{equation}
$deve-$
\end{equation}mosdelimitartambémquantoaissoseu\begin{equation}
$escopo.$
\end{equation}Quasetodososlivrossobreopçõescomeçamcomumadefiniçãoclaradoscontratosedesuaimplementaçãonasbolsas\begin{equation}
—
\end{equation}incluindoquestõescomo\begin{equation}
\left( margem,\right)
\end{equation}custosoperacionaiseimpostos\begin{equation}
\left( —,\right)
\end{equation}prosseguefornecendomodelosdeprecificação\begin{equation}
\left( e,\right)
\end{equation}\begin{equation}
\left( finalmente,\right)
\end{equation}relacionaestratégiasdenegociaçãoque\begin{equation}
$suposta-$
\end{equation}menteutilizemaplenapotencialidadedosconceitosemétodosdeprecificação\begin{equation}
$Opçdes:$
\end{equation}Operando4Volatilidade7\begin{equation}
$vistos.$
\end{equation}Oplanodopresentelivro\begin{equation}
desenvolve - se
\end{equation}deformasemelhanteatéolimiardoscapítulossobretécnicas\begin{equation}
\left( operacionais,\right)
\end{equation}apartirdosquaisdarámaisênfasenãoàsoperaçõesbaseadasna\begin{equation}
\left( precificação,\right)
\end{equation}masaoperaçõesbaseadasnascaracterísticasvariacionaisdos\begin{equation}
$preços.$
\end{equation}Adiferençaseránotadaaolongodo\begin{equation}
\left( livro,\right)
\end{equation}tantoemrelaçãoaostemasselecionadosquantoemdiferençade\begin{equation}
$conteúdo.$
\end{equation}Isso\begin{equation}
deve - se
\end{equation}àexistênciadeduaslinhasoperacionaisdistintas\begin{equation}
$(ou$
\end{equation}duasmaneirasdiferentesdeencararoproblemade\begin{equation}
“como
\end{equation}ganhardinheirocom\begin{equation}
$isso?”),$
\end{equation}asquaisserãodetalhadasa\begin{equation}
$seguir.$
\end{equation}Aprimeira\begin{equation}
\left( linha,\right)
\end{equation}tradicionalmentemais\begin{equation}
\left( antiga,\right)
\end{equation}tratadotrading\begin{equation}
$preço-$
\end{equation}\begin{equation}
\left( orientado,\right)
\end{equation}e\begin{equation}
concentra - se
\end{equation}emnegociaropçõesmais\begin{equation}
$ou'menos$
\end{equation}comosenegociaqualqueroutroativo\begin{equation}
—
\end{equation}procurandovendermaiscarodoquese\begin{equation}
\left( comprou,\right)
\end{equation}oucomprarmaisbaratodoquesevendeu\begin{equation}
—
\end{equation}comaressalvadequeospreçosdasopçõespodemseraproximadamentedefinidosemfunçãodedeterminados\begin{equation}
$fatores.$
\end{equation}Demaneiraparecidaàdosanalistasfundamentalistasde\begin{equation}
\left( ações,\right)
\end{equation}queprocuramcomseusmétodoschegaràmelhorestimativadopreçojustodeum\begin{equation}
\left( título,\right)
\end{equation}os4raders\begin{equation}
- orientados + preço
\end{equation}deopções\begin{equation}
rodeiam - se
\end{equation}detodososrecursosparachegaràmelhorestimativadopreçojustodeuma\begin{equation}
$opção.$
\end{equation}Issoincluitentarexplicaroscomportamentosirregularesqueosprêmiosexibememcondiçõessingularesde\begin{equation}
\left( mercado,\right)
\end{equation}tentar\begin{equation}
- los + reduzi
\end{equation}aumaexpressãomatemáticaquepossibiliteperfeito\begin{equation}
$ajuste.$
\end{equation}Ograndeproblemaquesecolocaaoanalista\begin{equation}
$preço-$
\end{equation}orientadodepossedoprêmiojustodeumaopção\begin{equation}
$é:$
\end{equation}oquefazercom\begin{equation}
$ele?$
\end{equation}Seomercado\begin{equation}
difere - se
\end{equation}\begin{equation}
\left( dele,\right)
\end{equation}\begin{equation}
deve - se
\end{equation}tentaralguma\begin{equation}
$arbitragem?$
\end{equation}Quetipodeoperação\begin{equation}
$utilizar?$
\end{equation}Comootrader\begin{equation}
- orientado + preço
\end{equation}temnoprêmiodasopçõesseualvoepontode\begin{equation}
\left( referência,\right)
\end{equation}ecomoosprêmiossãopredominantementesensíveisaopreçoSdo\begin{equation}
\left( objeto,\right)
\end{equation}otrading\begin{equation}
- orientado + preço
\end{equation}acaba\begin{equation}
$aproximando-$
\end{equation}sedeumaadiçãodevaloraotradingdirecionaldoativo5pelaescolhadomelhorinstrumentoaserutilizadona\begin{equation}
$posição.$
\end{equation}Na\begin{equation}
\left( verdade,\right)
\end{equation}muitasvezesOoperadordeopçõeseotraderprincipaldoativo\begin{equation}
confundem - se
\end{equation}emumasó\begin{equation}
$pessoa.$
\end{equation}Umadasdeficiênciasdalinhaderaciocínio\begin{equation}
- orientado + preço
\end{equation}éaconfusãoentreprodutos\begin{equation}
$possíveis:$
\end{equation}umaoperação\begin{equation}
$/$
\end{equation}posição\begin{equation}
$/estratégia$
\end{equation}pode\begin{equation}
- se + transformar
\end{equation}emoutra\begin{equation}
\left( indefinidamente,\right)
\end{equation}eosoperadoressãolevadosaingressaremumcozfiriuitmque\begin{equation}
\left( éjustificado, \  namaioria\right)
\end{equation}das\begin{equation}
\left( vezes, \  apenas\right)
\end{equation}pelaexposiçãoàdireçãode\begin{equation}
5.0
\end{equation}\begin{equation}
$==$
\end{equation}Asegundalinhadetradingéa\begin{equation}
$volatilidade-orientada.$
\end{equation}Seuobjetopríncipalnãoéoprêmioda\begin{equation}
\left( opção,\right)
\end{equation}esimumdosfatoressubjacentesaele\begin{equation}
—
\end{equation}a\begin{equation}
\left( volatilidade,\right)
\end{equation}queserávistanospróximoscapítulosdeste\begin{equation}
$livro.$
\end{equation}Oconceitodepreçojustotemimportânciasecundáriaparaosmembrosdessa\begin{equation}
\left( escola,\right)
\end{equation}quetrabalhammaisfrequentementecomoproblemainversoaoda\begin{equation}
$precificação:$
\end{equation}dadooprêmiodemercadoda\begin{equation}
\left( opção,\right)
\end{equation}qualavolatilidadeembutida\begin{equation}
$nele?$
\end{equation}Depossedessa\includegraphics[width=0.8\textwidth]{output/image_240png}8\begin{equation}
$Opções:$
\end{equation}OperandoaVolatilidade\begin{equation}
$:$
\end{equation}\begin{equation}
\left( volatilidade,\right)
\end{equation}ofraderpode\begin{equation}
comprá - la
\end{equation}ou\begin{equation}
- la + vendê
\end{equation}pormeiodeoperações\begin{equation}
$específicas.$
\end{equation}Naanalogiacomoanalistafundamentalistade\begin{equation}
\left( ações,\right)
\end{equation}otrader\begin{equation}
- orientado + volatilidade
\end{equation}\begin{equation}
parece - se
\end{equation}comaquele\begin{equation}
\left( que,\right)
\end{equation}emvezdeoperarospreçosdas\begin{equation}
\left( ações,\right)
\end{equation}operaseus\begin{equation}
$P/Ls.$
\end{equation}Algumas\begin{equation}
\left( vezes,\right)
\end{equation}ooperador\begin{equation}
- orientado + volatilidade
\end{equation}poderáestarinteressadoemconfrontaravolatilidadelidadopreçodeumaopçãocomoqueeleestimaseravolatilidadejustadeum\begin{equation}
\left( mercado,\right)
\end{equation}oquequasereduzseumétodoaumainversãodométodo\begin{equation}
\left( - orientado + preço,\right)
\end{equation}masoutrasvezeseleestaráinteressadoapenasemespecularcom\begin{equation}
\left( ela,\right)
\end{equation}damesmaformaquemuitosespeculadoresemaçõesecommoditiespartemparaotradingdirecionalsemseenvolvercomqualquerconsideraçãosobreoqueseriaopreçojustodoativocomoqualestão\begin{equation}
$lidando.$
\end{equation}Emoutras\begin{equation}
\left( vezes,\right)
\end{equation}ofader\begin{equation}
$volatilidade-$
\end{equation}orientadoestaráapenasarbitrando\begin{equation}
\left( volatilidade,\right)
\end{equation}eaísequerlheinteressaqualéoníveljustoouadireçãoda\begin{equation}
$volatilidade.$
\end{equation}Otrader\begin{equation}
- orientado + volatilidade
\end{equation}geralmentetrabalhaAedgeadocontra\begin{equation}
\left( \mathtt{\text{S}},\right)
\end{equation}deformaqueaelenãoimportaemabsolutoqualdireçãotomaráo\begin{equation}
$ativo-objeto.$
\end{equation}Demodo\begin{equation}
\left( abrangente,\right)
\end{equation}seforem\begin{equation}
»
\end{equation}osfatoresqueinfluenciamnoprêmiodeuma\begin{equation}
\left( opção,\right)
\end{equation}eleprocuraráque\begin{equation}
z - —1
\end{equation}fatorespossamserdefinidossemdúvida\begin{equation}
\frac{e}{ou}
\end{equation}\begin{equation}
\left( Aedgeados,\right)
\end{equation}eoperaráofator\begin{equation}
$restante.$
\end{equation}Muitas\begin{equation}
\left( vezes,\right)
\end{equation}0hedgerequeridoserá\begin{equation}
\left( dinâmico,\right)
\end{equation}oquesignificaqueooperadorterádepromoverajustesperiódicosemsua\begin{equation}
$posição.$
\end{equation}Assim\begin{equation}
\left( sendo,\right)
\end{equation}asoperaçõestípicasdessaescolatendemasermais\begin{equation}
$complexas.$
\end{equation}Em\begin{equation}
\left( contrapartida,\right)
\end{equation}tirarãoproveitodemuitasineficiênciasqueo\begin{equation}
\left( mercado,\right)
\end{equation}medianamente\begin{equation}
\left( despreparado,\right)
\end{equation}\begin{equation}
$exibirá.$
\end{equation}Dentreosmembrosdalinha\begin{equation}
- orientada + volatilidade
\end{equation}estãonãosóosquedeliberadamenteoperamvolatilidadecomoseuprincipal\begin{equation}
\left( produto,\right)
\end{equation}mas\begin{equation}
$tam-$
\end{equation}bémosquetêmcomofunçãoatenderàdemandaindistintadomercadopor\begin{equation}
\left( opções,\right)
\end{equation}eporfinalidade\begin{equation}
fazê - lo
\end{equation}semprecom\begin{equation}
$margem.$
\end{equation}OsbookrunnersdeopçõesOTCrepresentamestaúltima\begin{equation}
$categoria.$
\end{equation}\begin{equation}
\left( Abaixo,\right)
\end{equation}sintetizamosasmaioresdiferençasentreasduas\begin{equation}
$linhas.$
\end{equation}\begin{equation}
$Quere-$
\end{equation}mosdeixarclaroquenãoexisteoreconhecimentodenenhumadicotomiaprofundaentreosdois\begin{equation}
\left( esquemas,\right)
\end{equation}nem umaoposiçãonemuma\begin{equation}
$rivalidade.$
\end{equation}Oqueexistesãonichosondeumououtroé\begin{equation}
$predominante.$
\end{equation}Resumidasasduastendênciasprincipaisemoption\begin{equation}
\left( trading,\right)
\end{equation}devemosesclarecerqueopresentelivro\begin{equation}
identifica - se
\end{equation}comalinha\begin{equation}
$volatilidade-orienta-$
\end{equation}\begin{equation}
\left( da,\right)
\end{equation}equeestaseráatônicadoscapítulosquetratamdeassuntos\begin{equation}
$eminentemen-$
\end{equation}te\begin{equation}
$operacionais.$
\end{equation}\begin{equation}
\left( Contudo,\right)
\end{equation}oCapítulo5éreservadoàdescriçãotradicionaldosprodutosde\begin{equation}
\left( opções,\right)
\end{equation}eneleseráencontradomaterialinteressanteaooperadordequalquer\begin{equation}
$orientação.$
\end{equation}Comesta\begin{equation}
\left( delimitação,\right)
\end{equation}ficaclaroquealgunsassuntossãoconsideradosnãopertinentesparaoescopodeste\begin{equation}
$livro.$
\end{equation}Procuraremosidentificartais\begin{equation}
$Opções:$
\end{equation}OperandoaVolatilidade\begin{equation}
$É)$
\end{equation}\begin{equation}
Prêmio - orientada
\end{equation}\begin{equation}
Volatilidade - orientada
\end{equation}DifusãoMajoritáriaMinoritáriaObjetodeespeculaçãoprêmiodasopçõesvolatilidadeimplícitanasopções\begin{equation}
$|$
\end{equation}TécnicaprincipalcalcularprêmiojustocalcularvolatilidadeimplícitaQuestãoprincipalaopçãoestásobreou\begin{equation}
- avaliada + sub
\end{equation}pelo\begin{equation}
$mercado?$
\end{equation}avolatilidadevaisubirouvai\begin{equation}
$cair?$
\end{equation}\begin{equation}
$[Operações$
\end{equation}usuaisspreadse\begin{equation}
$combinações"$
\end{equation}entreopções\begin{equation}
estratégias
\end{equation}operaçõeshedgeadase\begin{equation}
$dinâmicas.$
\end{equation}irregularidadesOperadortípicoespeculadorquefazusooperadorde\begin{equation}
$volatilidade;$
\end{equation}deopçõesnotradingbookrunnerdeOTC\begin{equation}
$direcional;$
\end{equation}opfiortrader\begin{equation}
$*$
\end{equation}Posiçãodiantedemodelaroperar\begin{equation}
$|$
\end{equation}Pontofortepermiteumagamamaiorde\begin{equation}
$operações;$
\end{equation}posiçõesgeralmentemaisalavancadasdefineclaramenteumriscoaser\begin{equation}
$operado;$
\end{equation}sofisticaçãopropiciavantagemcompetitivaPontofracoconfusãoentreprincípiosoperacionaisposiçõespoucoalavancadas\begin{equation}
\left( assuntos,\right)
\end{equation}dandoumaligeiraexplanaçãoenãonosdetendomuito\begin{equation}
$neles.$
\end{equation}EsteémaisumdosmotivospeloqualcentraremosatençãonomodeloBlackeScholescomooriginariamente\begin{equation}
\left( concebido,\right)
\end{equation}pelasuasimplicidadee\begin{equation}
$conveniên-$
\end{equation}\begin{equation}
\left( cia,\right)
\end{equation}emvezdeesparsaroconhecimentopormeiodeoutrosmodelos\begin{equation}
$possivel-$
\end{equation}mentemais\begin{equation}
$acurados.$
\end{equation}TerminologiaenotaçãoPreçoSOtipodeopçãoquenosinteressaé a\begin{equation}
\frac{cal}{ou}
\end{equation}putvanilla\begin{equation}
\left( européia,\right)
\end{equation}queéOtipomaissimplesemaisnegociadode\begin{equation}
$opção.$
\end{equation}Umacallouputvaníllaficacompletamentecaracterizadaporsuadatadeexercícioepor seupreçode\begin{equation}
$exercício.$
\end{equation}\begin{equation}
\left( Além,\right)
\end{equation}é\begin{equation}
\left( claro,\right)
\end{equation}dadefiniçãodequaléoativooucontratosobreoqual\includegraphics[width=0.8\textwidth]{output/image_242png}10\begin{equation}
$Opções:$
\end{equation}OperandoaVolatilidadeÉé\begin{equation}
$lançada.$
\end{equation}\begin{equation}
\left( Genericamente,\right)
\end{equation}aopçãoélançadasobreum\begin{equation}
\left( preço,\right)
\end{equation}que nãoprecisa\begin{equation}
$-$
\end{equation}sernecessariamenteopreçodeumativo\begin{equation}
$(pode$
\end{equation}seruma\begin{equation}
\left( taxa,\right)
\end{equation}um\begin{equation}
\left( índice,\right)
\end{equation}uma\begin{equation}
$-$
\end{equation}quantidadeouqualqueroutracoisaqueseconvencionevaler\begin{equation}
$dinheiro).$
\end{equation}Por\begin{equation}
$;$
\end{equation}\begin{equation}
\left( isso,\right)
\end{equation}usaremosaexpressãopreçoSouapenasSpararepresentaroobjetosobreoqualaopçãoélançada\begin{equation}
\left( e,\right)
\end{equation}nocasode\begin{equation}
\left( exercício,\right)
\end{equation}\begin{equation}
$liquidada.$
\end{equation}Opreço\begin{equation}
\left( 5,\right)
\end{equation}segundoseu\begin{equation}
\left( tipo,\right)
\end{equation}influinotratamentoqueédadoà\begin{equation}
$opção.$
\end{equation}Oquedifereotipodopreço\begin{equation}
$$são$
\end{equation}duas\begin{equation}
$coisas:$
\end{equation}aprimeiraéseu\begin{equation}
\left( caminhamento,\right)
\end{equation}\begin{equation}
$-$
\end{equation}\begin{equation}
\left( ouseja,\right)
\end{equation}queespéciedeevoluçãopodemosesperar\begin{equation}
$dele.$
\end{equation}OpreçodeumcontratodeDI\begin{equation}
\left( futuro,\right)
\end{equation}por\begin{equation}
\left( exemplo,\right)
\end{equation}evoluiemdireçãoa\begin{equation}
$100.000;$
\end{equation}jápreçosdeações\begin{equation}
$:$
\end{equation}partemdeumvalorconhecidohojeepodemadquirirqualquervalorno\begin{equation}
$futuro;$
\end{equation}poroutro\begin{equation}
\left( lado,\right)
\end{equation}preçosdecontratosfuturosvariamtantoquantopreçosde\begin{equation}
\left( ativos,\right)
\end{equation}mastêmadiferençadenãorepresentarcustode\begin{equation}
$carregamento.$
\end{equation}No\begin{equation}
$:$
\end{equation}Brasildasaltastaxasdeinflação\begin{equation}
$+$
\end{equation}\begin{equation}
\left( juros,\right)
\end{equation}poderíamosdizerqueopreçodeum\begin{equation}
$.$
\end{equation}ativoavistadeveriaacompanharocustododinheiroeopreçodeum\begin{equation}
\left( futuro,\right)
\end{equation}\begin{equation}
$:$
\end{equation}\begin{equation}
$não.$
\end{equation}Astrêsfigurasabaixoexibemasdiferençasdepadrãodecaminhamento\begin{equation}
$|$
\end{equation}entreestes\begin{equation}
$exemplos:$
\end{equation}DifuturoativofuturotempotempoAsegundacoisaquedifereotipodoativoéarendaqueseupossuidor\begin{equation}
$aufere.$
\end{equation}Açõesproporcionam\begin{equation}
\left( dividendos,\right)
\end{equation}moedasproporcionam\begin{equation}
\left( juros,\right)
\end{equation}ouroemuitosoutrosativosproporcionam\begin{equation}
$aluguel.$
\end{equation}Aexpressãogeralmente\begin{equation}
$conhe-$
\end{equation}\begin{equation}
$.$
\end{equation}cidadomodeloBlackeScholesfoidesenvolvidaparaativossemrenda\begin{equation}
$(ações$
\end{equation}\begin{equation}
$:$
\end{equation}quenãopagam\begin{equation}
$dividendos)$
\end{equation}e éestetipodeativoquevamosconsiderar\begin{equation}
\left( aqui,\right)
\end{equation}\begin{equation}
$-$
\end{equation}atéporqueestemodelotambémserveparaopçõesprotegidascontra\begin{equation}
$dividen-$
\end{equation}\begin{equation}
$:$
\end{equation}\begin{equation}
\left( dos,\right)
\end{equation}quesãoocaso\begin{equation}
$brasileiro.$
\end{equation}OpçõesprotegidasOpçõesprotegidassãoaquelasquetêmseuspreçosdeexercício\begin{equation}
$automa-$
\end{equation}\begin{equation}
$.$
\end{equation}ticamentealteradosemcasodedistribuiçõesde\begin{equation}
\left( proventos,\right)
\end{equation}comoporexemplotempo\begin{equation}
$:$
\end{equation}\begin{equation}
$Ouções:$
\end{equation}OperandoaVolatilidade11o\begin{equation}
$fatodeumaaçãoirex-dividendo.$
\end{equation}OpçõesprotegidasdevemsertratadaspelaformulaçãooriginaldeBlackeScholesparaaçõessem\begin{equation}
$dividendos.$
\end{equation}Issotornatalformulaçãomuito\begin{equation}
\left( útil,\right)
\end{equation}mesmonosmercados\begin{equation}
\left( reais,\right)
\end{equation}emqueaçõespagam\begin{equation}
$dividendos.$
\end{equation}\begin{equation}
\left( Eventualmente,\right)
\end{equation}apresentamostambémdiscussõesemodelosdeopçõesnãoprotegidassobreativosqueproporcionam\begin{equation}
$renda.$
\end{equation}\begin{equation}
\left( Então,\right)
\end{equation}nosinteressampreçosdeaçõesque nãopagam\begin{equation}
\left( dividendos,\right)
\end{equation}paraosquaisfoidesenvolvidooriginariamentetodooinstrumentaldeprecificaçãode\begin{equation}
$opções.$
\end{equation}Semmuitoesforço\begin{equation}
\left( adicional,\right)
\end{equation}\begin{equation}
pode - se
\end{equation}transportartodoo\begin{equation}
$desenvol-$
\end{equation}vimentoparaocasodecommodities\begin{equation}
$(metais,$
\end{equation}agrícolas\begin{equation}
$emoedas).$
\end{equation}Dedicaremosalgunsparágrafosdoúltimocapítuloaos\begin{equation}
\left( juros,\right)
\end{equation}quenãopodemsertãofacilmente\begin{equation}
$tratados.$
\end{equation}\begin{equation}
\left( Assim,\right)
\end{equation}asopçõesaquenosreferiremosaquisãosobreaçõesou\begin{equation}
\left( commodities,\right)
\end{equation}e\begin{equation}
\left( Sserá,\right)
\end{equation}de\begin{equation}
\left( fato,\right)
\end{equation}opreçodeum\begin{equation}
$ativo.$
\end{equation}Preçodeexercício\begin{equation}
K
\end{equation}Opreçodeexercício\begin{equation}
$(ou$
\end{equation}strikepriceouabreviadamente\begin{equation}
$sZr/ke)$
\end{equation}deumaopçãoéovalordeSalémdoqualoexercíciopossuiumvalor\begin{equation}
$positivo.$
\end{equation}ParavaloresdeSaquémdopreçodeexercício\begin{equation}
$(notado$
\end{equation}por\begin{equation}
$4),$
\end{equation}aopçãonãopossuinenhumvalorno\begin{equation}
$vencimento.$
\end{equation}\begin{equation}
\left\lfloor{\frac{Ca}{s}}\right\rfloor
\end{equation}valem\begin{equation}
$(5$
\end{equation}\begin{equation}
—
\end{equation}\begin{equation}
$K)$
\end{equation}para\begin{equation}
$$ >$
\end{equation}\begin{equation}
\left( K,\right)
\end{equation}ezerocaso\begin{equation}
$contrário;$
\end{equation}putsvalem\begin{equation}
$(K—<S)$
\end{equation}para\begin{equation}
$S<K$
\end{equation}ezerocaso\begin{equation}
$contrário.$
\end{equation}CallsdãoexercícioparavaloresdeSmaioresdoqueopreçode\begin{equation}
$exercício;$
\end{equation}ep1tsdãoexercícioparavaloresde\begin{equation}
$$$
\end{equation}menoresdoqueopreçode\begin{equation}
$exercício.$
\end{equation}Chamamosasquantidades\begin{equation}
$(S-K)$
\end{equation}ou\begin{equation}
$(K—$
\end{equation}\begin{equation}
$5),$
\end{equation}quando\begin{equation}
\left( positivas,\right)
\end{equation}devalorde\begin{equation}
\left( exercício,\right)
\end{equation}isto\begin{equation}
\left( é,\right)
\end{equation}aquantidadededinheiroqueseencaixapormeiodeumexercício\begin{equation}
$favorável.$
\end{equation}Prémio\begin{equation}
9
\end{equation}Opçõessãodireitos\begin{equation}
\left( negociáveis,\right)
\end{equation}oquequerdizerquepossuemumpreçoouprêmio\begin{equation}
$(para$
\end{equation}evitarconfusãocompreçode\begin{equation}
$exercício)$
\end{equation}peloqualsão\begin{equation}
\left( lançadas,\right)
\end{equation}epelo qualsãonegociadasnosmercados\begin{equation}
$organizados.$
\end{equation}NotaremosoprêmiodeumaopçãoporVindependentedeserelauma\begin{equation}
\left\lfloor{\frac{ca}{ou}}\right\rfloor
\end{equation}\begin{equation}
$puí.$
\end{equation}Quandonosreferirmosespecificamenteaoprêmiodeuma\begin{equation}
$ca!/,$
\end{equation}notaremospor\begin{equation}
\left( C,\right)
\end{equation}equandonosreferirmosaoprêmiodeuma\begin{equation}
\left( \frac{pv}{í},\right)
\end{equation}por\begin{equation}
$?.$
\end{equation}Umavezqueopçõessãoderivativos\begin{equation}
—
\end{equation}têmseuspreçoscondicionadosaopreçodeoutroinstrumento\begin{equation}
—
\end{equation}\begin{equation}
espera - se
\end{equation}quesepossaencontrarumafórmulaparaoprêmiojustodeumaopçãoemfunçãodopreçoprimitivoSdentreoutras\begin{equation}
$coisas.$
\end{equation}Todoesforçoemmodelaropçõesgiraemtornodoprêmiojustodadas\begin{equation}
$condições.$
\end{equation}\includegraphics[width=0.8\textwidth]{output/image_244png}12\begin{equation}
$Opções:$
\end{equation}Operandoa\begin{equation}
\left( Volatilidad,\right)
\end{equation}Payoff\begin{equation}
$(V$
\end{equation}x\begin{equation}
$5)$
\end{equation}Asfunçõespayoffquenosinteressarãoaquisãoasfunçõespayoffde\begin{equation}
$opções:$
\end{equation}vanilia\begin{equation}
\left( européias,\right)
\end{equation}asquaistêmperfiscomomostradoem\begin{equation}
$(9).$
\end{equation}A\begin{equation}
$forma:$
\end{equation}matemáticadessasfunções\begin{equation}
$é:$
\end{equation}\begin{equation}
$C*=$
\end{equation}max\begin{equation}
$(0,$
\end{equation}\begin{equation}
$S*-$
\end{equation}\begin{equation}
$X)$
\end{equation}\begin{equation}
$P*=max$
\end{equation}\begin{equation}
$(0,$
\end{equation}\begin{equation}
$K—-$
\end{equation}\begin{equation}
$5%$
\end{equation}sendo\begin{equation}
C e
\end{equation}\begin{equation}
P os
\end{equation}valoresdeexercíciodeuma\begin{equation}
\left\lfloor{\frac{ca}{e}}\right\rfloor
\end{equation}umaput\begin{equation}
$(iguais$
\end{equation}aovalordaopçãonadatade\begin{equation}
$exercício),$
\end{equation}Xostrike\begin{equation}
\left( price,\right)
\end{equation}e\begin{equation}
$S*o$
\end{equation}preçodoativonadatade\begin{equation}
$exercício.$
\end{equation}Oprêmiopeloqualumaopçãoénegociadarefleteasexpectativassobreseuvalorde\begin{equation}
$exercício.$
\end{equation}Nadatade\begin{equation}
\left( exercício,\right)
\end{equation}opreçodeumaopçãoé\begin{equation}
$exatamen-$
\end{equation}teigualaseuvalorde\begin{equation}
\left( exercício,\right)
\end{equation}sendoindiferente\begin{equation}
exercê - la
\end{equation}ou\begin{equation}
$vendê-la:.$
\end{equation}\begin{equation}
$(devido$
\end{equation}aregrasdasbolsase aquestõesde\begin{equation}
\left( liquidez,\right)
\end{equation}émuitomaiscomum\begin{equation}
$o:$
\end{equation}\begin{equation}
$exercício).$
\end{equation}Istosignificaquetodainformaçãoacercadoexercíciodeumaopção\begin{equation}
$:$
\end{equation}estácontidaemseu\begin{equation}
\left( prêmio,\right)
\end{equation}equeoprêmiosempreconvergeparaovalor\begin{equation}
$de:$
\end{equation}\begin{equation}
\left( exercício,\right)
\end{equation}nadatade\begin{equation}
$exercício.$
\end{equation}Refletirasexpectativassignificadizerqueumaopçãodeveria\begin{equation}
$negociar.$
\end{equation}hojeaovalorpresentedeseuvalordeexercício\begin{equation}
$esperado.$
\end{equation}Parasecalcular\begin{equation}
$o.$
\end{equation}valordeexercício\begin{equation}
\left( esperado,\right)
\end{equation}\begin{equation}
deve - se
\end{equation}projetar\begin{equation}
$$$
\end{equation}paraadatadeexercício\begin{equation}
$de:$
\end{equation}alguma\begin{equation}
$forma.$
\end{equation}Estaprojeçãoéfeita\begin{equation}
- se + utilizando
\end{equation}astaxasdejurodo\begin{equation}
$mercado.$
\end{equation}\begin{equation}
Pode - se
\end{equation}questionarseaprojeçãoviataxasdejuroéumaboa\begin{equation}
$estimativa;$
\end{equation}\begin{equation}
\left( afinal,\right)
\end{equation}nenhumativorealéobrigadoacorrigir\begin{equation}
$juros;$
\end{equation}\begin{equation}
pode - se
\end{equation}argumentarquemelhorestimativaseriaprojetarSpelomesmovalorqueeleapresenta\begin{equation}
\left( hoje,\right)
\end{equation}\begin{equation}
$ou:$
\end{equation}nomáximoacrescentaraelea\begin{equation}
$inflação.$
\end{equation}Acontecequeaprojeçãoporjurosnão\begin{equation}
$'$
\end{equation}partedapremissadequetodososativosdevamacompanharos\begin{equation}
\left( juros,\right)
\end{equation}masde\begin{equation}
$:$
\end{equation}umaoutrasutilmente\begin{equation}
$diferente:$
\end{equation}adequeovalormédiovisualizado\begin{equation}
$pelo.$
\end{equation}mercadoparaumativoemdatafuturacoincidacomseuvalor\begin{equation}
$futuro.$
\end{equation}Isto\begin{equation}
\left( é,\right)
\end{equation}\begin{equation}
$:$
\end{equation}o\begin{equation}
\left( mercado,\right)
\end{equation}comoum\begin{equation}
\left( todo,\right)
\end{equation}visualizariaoativonofuturocomosendoo\begin{equation}
$seu.$
\end{equation}preçoavistacarregadoaosjuros\begin{equation}
$correntes.$
\end{equation}Seomercadovisualizasseopreço\begin{equation}
$$acima$
\end{equation}deseuvalor\begin{equation}
\left( futuro,\right)
\end{equation}semdúvidapromoveriaumapressãocompradoraqueacabariaelevando\begin{equation}
$$$
\end{equation}ecorrigindoa\begin{equation}
$diferença;$
\end{equation}sevisualizasse\begin{equation}
\left( abaixo,\right)
\end{equation}\begin{equation}
$:$
\end{equation}promoveriaumapressão\begin{equation}
$vendedora.$
\end{equation}Emambosos\begin{equation}
\left( casos,\right)
\end{equation}ovalordeSquehoje\begin{equation}
$|$
\end{equation}equilibraasexpectativasteriaapropriedade\begin{equation}
\left( de,\right)
\end{equation}carregadoa\begin{equation}
\left( juros,\right)
\end{equation}coincidir\begin{equation}
$|$
\end{equation}comopreçoesperadopelomercadoparaadata\begin{equation}
$futura.$
\end{equation}Ummercadoondeespeculadorestenhamessasexpectativaséditoneutro\begin{equation}
$:$
\end{equation}ao\begin{equation}
\left( risco,\right)
\end{equation}ou\begin{equation}
$visk-newtral.$
\end{equation}Em ummercado\begin{equation}
\left( - newtral + risk,\right)
\end{equation}éfáciljustificaraprojeção\begin{equation}
$:$
\end{equation}\begin{equation}
$Opções:$
\end{equation}OperandoaVolatilidade13paraOfuturopelataxadejurolivrede\begin{equation}
$risco.$
\end{equation}Ofatoéqueédifícilcomprovarocomportamento\begin{equation}
$775k-mextralno$
\end{equation}mercado\begin{equation}
$real.$
\end{equation}No\begin{equation}
\left( entanto,\right)
\end{equation}umaxiomadizqueasuposiçãodeummundo\begin{equation}
$775k-meutral$
\end{equation}nãoénecessáriaparaaprecificaçãodefuturose\begin{equation}
$opções:$
\end{equation}mesmoemummundonão\begin{equation}
$775k-meutral,$
\end{equation}derivativostêmpreçosiguaisaosqueseriamaelesatribuídosemummundo\begin{equation}
\left( - neutral + r15k,\right)
\end{equation}eisto\begin{equation}
dá - se
\end{equation}pelaexistênciade\begin{equation}
$arbitragem.$
\end{equation}Nopróximo\begin{equation}
\left( capítulo,\right)
\end{equation}veremoscomoisso\begin{equation}
$ocorre.$
\end{equation}Juros\begin{equation}
1
\end{equation}eprazo\begin{equation}
6
\end{equation}Umavezqueoprêmiodeumaopçãodependedosvaloresliquidadosno\begin{equation}
\left( futuro,\right)
\end{equation}elesofreinfluênciasdataxadejurolivrederiscocorrentenomercado\begin{equation}
\left( e,\right)
\end{equation}\begin{equation}
\left( logicamente,\right)
\end{equation}doprazoatéovencimentoda\begin{equation}
$opção.$
\end{equation}Ataxadejuroseránotadapor\begin{equation}
$7:$
\end{equation}esteéocustomédiododinheirodesdeadataatualatéadatadevencimentodas\begin{equation}
\left( opções,\right)
\end{equation}expressoemunidadescompatíveisao\begin{equation}
$prazo:$
\end{equation}seoprazoformedidoem\begin{equation}
\left( dias,\right)
\end{equation}rdevesermedidoemporcentoao\begin{equation}
$dia.$
\end{equation}Namaioriadasvezesneste\begin{equation}
\left( livro,\right)
\end{equation}por\begin{equation}
\left( comodidade,\right)
\end{equation}7seráexpressacomotaxaefetivano\begin{equation}
$período;$
\end{equation}algumasoutrasvezesseráexpressacomotaxa\begin{equation}
\left( over,\right)
\end{equation}eassinaladacom\begin{equation}
$%$
\end{equation}\begin{equation}
$over.$
\end{equation}Aliteraturaespecializadaemopçõesgeralmenteintroduzoconceitodetaxacontinuamentecompostaantesdedescreverosmodelosde\begin{equation}
$precificação.$
\end{equation}Umataxadejurocontinuamentecompostaécapitalizada\begin{equation}
\left( infinitesimalmente,\right)
\end{equation}eéumacomodidadeparaquasetodososmodelosde\begin{equation}
\left( derivativos,\right)
\end{equation}poisestesconsideramotempoumavariável\begin{equation}
\left( contínua,\right)
\end{equation}enão\begin{equation}
$discreta.$
\end{equation}\begin{equation}
\left( Assim,\right)
\end{equation}seumataxadejuroétalqueproporcioneovalorfuturo\begin{equation}
$(1$
\end{equation}\begin{equation}
$+$
\end{equation}\begin{equation}
$7)$
\end{equation}apósumaunidadede\begin{equation}
\left( tempo,\right)
\end{equation}ataxacontínuaequivalente7cseráiguala\begin{equation}
$re=n(l+?7)$
\end{equation}eofatorvalorfuturorelativoaessataxasobreoperíodo\begin{equation}
£
\end{equation}seráigualaVE\begin{equation}
$=€$
\end{equation}mxrquecoincidirácom\begin{equation}
$(1$
\end{equation}\begin{equation}
$+$
\end{equation}\begin{equation}
$7).$
\end{equation}Exemplificandoaconversãodetaxasdejuroparataxas\begin{equation}
\left( contínuas,\right)
\end{equation}temosatabela\begin{equation}
\left( abaixo,\right)
\end{equation}emqueaprimeiracolunaéumataxadejuro\begin{equation}
\left( linear,\right)
\end{equation}asegundaéumataxadejurocontinuamente\begin{equation}
\left( composta,\right)
\end{equation}eaterceiraéoresultadode0\begin{equation}
$*'.$
\end{equation}\includegraphics[width=0.8\textwidth]{output/image_246png}14\begin{equation}
$Opções:$
\end{equation}Operandoa\begin{equation}
\left( Volatilidad,\right)
\end{equation}JurolinearTaxacontínuageE\begin{equation}
$1%$
\end{equation}\begin{equation}
$0,995%$
\end{equation}\begin{equation}
$=$
\end{equation}\begin{equation}
$0,00995$
\end{equation}\begin{equation}
$1,01$
\end{equation}So\begin{equation}
$1,98%$
\end{equation}\begin{equation}
$=$
\end{equation}\begin{equation}
$0,0198$
\end{equation}\begin{equation}
$1,02$
\end{equation}\begin{equation}
$5%$
\end{equation}\begin{equation}
$4,88%$
\end{equation}\begin{equation}
$=$
\end{equation}\begin{equation}
$0,0488$
\end{equation}\begin{equation}
$1,05$
\end{equation}\begin{equation}
$10%$
\end{equation}\begin{equation}
$9,53%$
\end{equation}\begin{equation}
$=$
\end{equation}\begin{equation}
$0,0953$
\end{equation}\begin{equation}
\left( 1, \  10\right)
\end{equation}\begin{equation}
$20%$
\end{equation}\begin{equation}
$18,23%$
\end{equation}\begin{equation}
$=$
\end{equation}\begin{equation}
\left( 0, \  1823\right)
\end{equation}\begin{equation}
\left( 1, \  20\right)
\end{equation}\begin{equation}
$50%$
\end{equation}\begin{equation}
$40,54%$
\end{equation}\begin{equation}
$=$
\end{equation}\begin{equation}
\left( 0, \  4054\right)
\end{equation}\begin{equation}
\left( 1, \  50\right)
\end{equation}Nanotaçãoutilizadaneste\begin{equation}
\left( livro,\right)
\end{equation}oscálculosdevalorpresenteedevalorfuturosãonotadospor\begin{equation}
$VP(Je$
\end{equation}\begin{equation}
\left( \operatorname{VF}{\left( \right)},\right)
\end{equation}\begin{equation}
$respectivamente.$
\end{equation}Ficasubentendidoque\begin{equation}
\operatorname{VP}{\left( \right)}
\end{equation}podesignificartanto\begin{equation}
$(1$
\end{equation}\begin{equation}
$+$
\end{equation}\begin{equation}
$7)!$
\end{equation}quanto\begin{equation}
$«**$
\end{equation}dependendodaformacomo\begin{equation}
$se:$
\end{equation}estejaexpressandoataxade\begin{equation}
$juro.$
\end{equation}Porisso\begin{equation}
\left( mesmo,\right)
\end{equation}sempreque\begin{equation}
\left( possível,\right)
\end{equation}pouparemosoleitordeconsideraraconversãodetaxasdejuropara\begin{equation}
$taxas:$
\end{equation}contínuasantesdeentrarnos\begin{equation}
$modelos.$
\end{equation}Consideramosoleitorfundamentadoemmatemática\begin{equation}
$financeira.$
\end{equation}Dadoumfluxodecaixacompostodeumaquantiaaplicadanadatapresente\begin{equation}
aplique
\end{equation}eumaquantiarecebidaemdatafutura\begin{equation}
\left( resgate,\right)
\end{equation}ataxadejuro\begin{equation}
$efetiva.$
\end{equation}noperíodoédada\begin{equation}
$por:$
\end{equation}resgate\begin{equation}
$=$
\end{equation}1\begin{equation}
$|x100%$
\end{equation}aplique\begin{equation}
$saques),$
\end{equation}ataxadejurodesta\begin{equation}
\left( aplicação,\right)
\end{equation}expressaemtaxa\begin{equation}
\left( over,\right)
\end{equation}édada\begin{equation}
$por:$
\end{equation}1tresgate\begin{equation}
$:$
\end{equation}\begin{equation}
$—1|x3.000%$
\end{equation}apliqueOverApartirdejaneirode\begin{equation}
\left( 1998,\right)
\end{equation}foiintroduzidaumanovamodalidadedemedidadetaxasde\begin{equation}
\left( juro,\right)
\end{equation}queé ataxa\begin{equation}
$anual:$
\end{equation}resgate\begin{equation}
$|$
\end{equation}\begin{equation}
$!$
\end{equation}\begin{equation}
$-$
\end{equation}\begin{equation}
\frac{—1}{x}
\end{equation}\begin{equation}
$100%$
\end{equation}apliqueano\begin{equation}
$Opções:$
\end{equation}OperandoaVolatilidade15Oprazoatéovencimentodaopçãoseránotadopor\begin{equation}
£
\end{equation}Otemposignificaprazoenãotempo\begin{equation}
\left( decorrido,\right)
\end{equation}eportantoécontadoapartirdovencimentopara\begin{equation}
$trás.$
\end{equation}Otempo\begin{equation}
$/=$
\end{equation}0\begin{equation}
\left( é,\right)
\end{equation}\begin{equation}
\left( então,\right)
\end{equation}adatadeexercícioda\begin{equation}
$opção;$
\end{equation}sehá20diasúteisdehojeatéo\begin{equation}
\left( exercício,\right)
\end{equation}otempo\begin{equation}
$£=$
\end{equation}20correspondeàdatade\begin{equation}
$hoje.$
\end{equation}\begin{equation}
\left( Rigorosamente,\right)
\end{equation}tseráoprazomedidoemdias\begin{equation}
$úteis.$
\end{equation}OpçãosobrefuturosUmaopçãopodesersobreativosavistaousobrecontratos\begin{equation}
$futuros.$
\end{equation}Demodo\begin{equation}
\left( semelhante,\right)
\end{equation}aexpectativadevalordeexercícibdeumaopçãosobrefuturos\begin{equation}
dá - se
\end{equation}pelaestimativadequalseráacotaçãodocontratofuturonadatade\begin{equation}
$exercício.$
\end{equation}SechamamosacotaçãodeumcontratofuturodeFeutilizamosesteparâmetroparaprecificara\begin{equation}
\left( opção,\right)
\end{equation}entãoaesteparâmetro7nãosedeveacrescentarnenhumacorreçãoa\begin{equation}
\left( juros,\right)
\end{equation}vistoqueelejárepresentaumvalorno\begin{equation}
$futuro.$
\end{equation}Sejaqualforoprazodaopçãoeoprazodofuturo\begin{equation}
\left( subjacente,\right)
\end{equation}Fseráovaloraserutilizadono\begin{equation}
$modelo.$
\end{equation}Dizendoemoutras\begin{equation}
\left( palavras,\right)
\end{equation}aestimativadacotaçãodofuturoemqualquerdataé\begin{equation}
7.0
\end{equation}Antesdoúltimo\begin{equation}
\left( capítulo,\right)
\end{equation}estaremoslidandocomopçõessobre\begin{equation}
\left( ativos,\right)
\end{equation}enãosobre\begin{equation}
$futuros.$
\end{equation}Por\begin{equation}
\left( isso,\right)
\end{equation}anotação7 serámaisútilparadesignarovalorfuturode\begin{equation}
$$$
\end{equation}emumadata\begin{equation}
\left( específica,\right)
\end{equation}especialmentenadatadevencimentodeuma\begin{equation}
$opção.$
\end{equation}Átivocomcarry\begin{equation}
j
\end{equation}Assumindoqueoperíodoentreoapliqueeoresgateéde\begin{equation}
$?dias$
\end{equation}úteis\begin{equation}
$(ou$
\end{equation}\begin{equation}
$|$
\end{equation}Oativo\begin{equation}
$$sobre$
\end{equation}oqualcontratosdeopçõesoudefuturossãoabertospodeterumcarrypositivoounegativoalémdos\begin{equation}
$juros.$
\end{equation}Estecarrpodeserumcusto\begin{equation}
$(custos$
\end{equation}de\begin{equation}
\left( estocagem,\right)
\end{equation}transportee\begin{equation}
$seguro)$
\end{equation}ouumarenda\begin{equation}
$(aluguel).$
\end{equation}Oscasosemqueoativo\begin{equation}
$$proporciona$
\end{equation}umaluguelsãomuitomais\begin{equation}
\left( frequentes,\right)
\end{equation}\begin{equation}
$incluin-$
\end{equation}\begin{equation}
$:$
\end{equation}\begin{equation}
do - se
\end{equation}aíquandoSéumamoeda\begin{equation}
\left( estrangeira,\right)
\end{equation}queproporciona\begin{equation}
$juros.$
\end{equation}Aformaçãodepreçosfuturosnessescasosprevêque7sejaovalorde\begin{equation}
$&$
\end{equation}agiadopelosjurosedesagiadopelo\begin{equation}
\left( carry,\right)
\end{equation}oqueintroduzemtodanossaanálisemaisum\begin{equation}
\left( elemento,\right)
\end{equation}\begin{equation}
$/,$
\end{equation}Ocarrydoativo\begin{equation}
$S.$
\end{equation}Seumaopçãoésobre\begin{equation}
\left( futuros,\right)
\end{equation}tanto7quanto\begin{equation}
$/$
\end{equation}jáestãoembutidosnasexpectativassobre\begin{equation}
$Z:$
\end{equation}estaopçãonãosofreinfluênciassignificativasdessesparâmetros\begin{equation}
$(mais$
\end{equation}\begin{equation}
\left( exatamente,\right)
\end{equation}nãosofreinfluênciaalgumade\begin{equation}
$/$
\end{equation}esofreainfluênciade7somenteporqueestaéataxausadapararetrocederaestimativadevalordeexercícioparao\begin{equation}
\left( presente,\right)
\end{equation}\begin{equation}
obtendo - se
\end{equation}assimoprêmioavistada\begin{equation}
$opção).$
\end{equation}Seaopçãoésobre\begin{equation}
\left( spot,\right)
\end{equation}entãotantorquanto\begin{equation}
$/$
\end{equation}serãofatoresinfluentesno\begin{equation}
$preço.$
\end{equation}EmboraquaisquerfórmulasdasqueveremosaquipossamseradaptadasparaopçõessobreativoscomDOERAMOESENSAIO\includegraphics[width=0.8\textwidth]{output/image_248png}16\begin{equation}
$Opções:$
\end{equation}OperandoaVolatilide\begin{equation}
\left( CarFiy,\right)
\end{equation}Namosnosrestringiraopçõessobreativossem\begin{equation}
\left( carry,\right)
\end{equation}ondejéigualazeroDesse\begin{equation}
\left( modo,\right)
\end{equation}7nãoapareceráemnenhuma\begin{equation}
\left( expressão,\right)
\end{equation}anãoseremcasoespeciaiscomentadosà\begin{equation}
$parte.$
\end{equation}Blacke\begin{equation}
\left( Scholes,\right)
\end{equation}queserávistonoúltimo\begin{equation}
$capítulo.$
\end{equation}Volatilidade\begin{equation}
o
\end{equation}sobreativosemcarry\begin{equation}
$(para$
\end{equation}umativocom\begin{equation}
\left( carry,\right)
\end{equation}éo\begin{equation}
$sexto),$
\end{equation}eénotadapor\begin{equation}
$(sigma,$
\end{equation}amesmaletragregaque\begin{equation}
\left( designa,\right)
\end{equation}em\begin{equation}
\left( estatística,\right)
\end{equation}odesvio\begin{equation}
$padrão).$
\end{equation}Osfatoresanterioresestarãoexaustivamentepresentesnodecorrerd\begin{equation}
$livro.$
\end{equation}Apresentaremosagoradefiniçõesfinaissobrealinguagemdo\begin{equation}
$texto.$
\end{equation}\begin{equation}
\left( VI,\right)
\end{equation}\begin{equation}
\left( Cr,\right)
\end{equation}\begin{equation}
$Pres*$
\end{equation}Semprequeumasterisco\begin{equation}
$(*)$
\end{equation}vierdepoisdeumsímbolo\begin{equation}
$(por$
\end{equation}\begin{equation}
\left( exemplo,\right)
\end{equation}\begin{equation}
$C'9,:$
\end{equation}éigualaopreçodeumnegócioatermo\begin{equation}
$(considerando$
\end{equation}queataxadejuronão\begin{equation}
está - se
\end{equation}identificandoumacondiçãonadatade\begin{equation}
$exercício.$
\end{equation}\begin{equation}
\left( Assim,\right)
\end{equation}\begin{equation}
$C*,$
\end{equation}\begin{equation}
P e
\end{equation}\begin{equation}
$V'sã$
\end{equation}todossímbolosdeprêmiosdeopçãonãoemummomento\begin{equation}
\left( qualquer,\right)
\end{equation}maprêmiosnadatade\begin{equation}
\left( exercício,\right)
\end{equation}ou\begin{equation}
\left( seja,\right)
\end{equation}valoresde\begin{equation}
$exercício.$
\end{equation}TodasascurvasxSexibidasnesteprimeirocapítulodolivro\begin{equation}
\left( são,\right)
\end{equation}na\begin{equation}
\left( verdade,\right)
\end{equation}curvas\begin{equation}
V x
\end{equation}5poissóseverificamnoexercícioda\begin{equation}
$opção.$
\end{equation}NotaçãodiferencialMuitas\begin{equation}
\left( vezes,\right)
\end{equation}estaremoslidandocomdiferenciaisemvezdequantidadesMuitosdosmétodosaplicadosaoperaçõescomopções\begin{equation}
preocupam - se
\end{equation}antecomvariaçõesdevaloresdoquecomospróprios\begin{equation}
$valores.$
\end{equation}Além\begin{equation}
\left( disso,\right)
\end{equation}bopartedateoriadeprecificaçãodeopçõesutilizaequações\begin{equation}
$diferenciais.$
\end{equation}Notaremospordxumavariaçãoinfinitesimaldagrandeza\begin{equation}
$x.$
\end{equation}Poràxamesmacoisaquandonocontextodediferenciaisparciais\begin{equation}
$(isto$
\end{equation}\begin{equation}
\left( é,\right)
\end{equation}quandohouvervariaçõeinfinitesimaisdeoutrasgrandezas\begin{equation}
$envolvidas).$
\end{equation}EporDxumavariaçãofinita\begin{equation}
\left( extensa,\right)
\end{equation}\begin{equation}
\left( mensurável,\right)
\end{equation}dagrandeza\begin{equation}
$x.$
\end{equation}Dxpodeserdefinidocomo\begin{equation}
$(x1$
\end{equation}\begin{equation}
—
\end{equation}\begin{equation}
$20),$
\end{equation}otadiferençaentreovalorfinaleoinicialdexduranteumprocessoqualquer\begin{equation}
$Ouções:$
\end{equation}OperandoaVolatilidade17SeriaoequivalenteaoAxdamatemáticaefísica\begin{equation}
\left( elementares,\right)
\end{equation}esónãoo\begin{equation}
»screvemos
\end{equation}destamaneiraparaevitaraconfusãocomoparâmetrodeltadas\begin{equation}
$+$
\end{equation}\begin{equation}
\left( opções, \  que\right)
\end{equation}tambémérepresentadopelaletragregadelta\begin{equation}
$(A).$
\end{equation}Conceberqualseriaoefeitodocarrynopreçodeumaopçãosobre\begin{equation}
\left( ativo,\right)
\end{equation}avistaé\begin{equation}
$simples:$
\end{equation}\begin{equation}
\left( simplificadamente,\right)
\end{equation}ocarrypossuiumefeitosempre\begin{equation}
$contrário:$
\end{equation}aodataxadejuro\begin{equation}
$(se$
\end{equation}ocarryfor\begin{equation}
\left( negativo,\right)
\end{equation}possuiumefeitosemelhanteao\begin{equation}
\left( dos,\right)
\end{equation}\begin{equation}
$juros;$
\end{equation}\begin{equation}
\left( afinal,\right)
\end{equation}ojuroéumcarrynegativo\begin{equation}
—
\end{equation}umcusto\begin{equation}
$financeiro).$
\end{equation}Para\begin{equation}
\left( uma,\right)
\end{equation}avaliação\begin{equation}
\left( exata,\right)
\end{equation}existeomodelode\begin{equation}
\left( Garman - Kohlhagen,\right)
\end{equation}umaadaptação\begin{equation}
$do.$
\end{equation}ConceitosoperacionaisOlivroéeminentemente\begin{equation}
\left( operacional,\right)
\end{equation}eojargãodemesadeoperaçõesé\begin{equation}
$utilizado.$
\end{equation}Consideramosoleitorentrosadoobastanteparaquenãolhesejam\begin{equation}
$:$
\end{equation}totalmenteestranhostermoscomoposição\begin{equation}
$(quantidade$
\end{equation}deunidadesquese\begin{equation}
«
\end{equation}possuideumativoou\begin{equation}
$derivativos),$
\end{equation}exposição\begin{equation}
$(tamariho$
\end{equation}deumaapostaou\begin{equation}
$*$
\end{equation}risco\begin{equation}
$específico),$
\end{equation}tradingdedireção\begin{equation}
$(aposta$
\end{equation}naaltaounabaixadeum\begin{equation}
$preço)$
\end{equation}\begin{equation}
$|$
\end{equation}hedge\begin{equation}
$(redução$
\end{equation}ouanulaçãodeumriscomedianteaadiçãodeinstrumentosOúltimofatorquefaltaserinvestigadoéa\begin{equation}
\left( volatilidade,\right)
\end{equation}quemerece\begin{equation}
$um.$
\end{equation}capítuloà\begin{equation}
$parte.$
\end{equation}Elaéoquintoelementoqueinfluinopreçodeuma\begin{equation}
$opção.$
\end{equation}4adequadosà\begin{equation}
$carteira),$
\end{equation}\begin{equation}
$"operação$
\end{equation}\begin{equation}
$seca"$
\end{equation}\begin{equation}
$(operação$
\end{equation}emapenasum\begin{equation}
$instrumento),$
\end{equation}ouoperação\begin{equation}
travada”
\end{equation}\begin{equation}
$(operação$
\end{equation}emmaisdeum\begin{equation}
\left( instrumento,\right)
\end{equation}geralmenteemdireções\begin{equation}
$contrárias).$
\end{equation}Nomomento\begin{equation}
\left( oportuno,\right)
\end{equation}definiremoscontratos\begin{equation}
$futuros.$
\end{equation}Procuraremos\begin{equation}
diferenciá - los
\end{equation}decontratosatermonosentidoemquenestesúltimosaliquidação\begin{equation}
dá - se
\end{equation}no\begin{equation}
\left( vencimento,\right)
\end{equation}equenos\begin{equation}
\left( primeiros,\right)
\end{equation}elaéfeitaporajustes\begin{equation}
$diários.$
\end{equation}Todocontratoaquireferidocomoumcontratofuturoéliquidadoporajustes\begin{equation}
\left( diários,\right)
\end{equation}oquefazcomqueaequivalênciaentreposiçõesavistaeemfuturossejamaispróximadaigualdadeentreseusvaloresfinanceirosdoquedaigualdadeentresuas\begin{equation}
$quantidades.$
\end{equation}\begin{equation}
\left( Contudo,\right)
\end{equation}opreçodeumcontratofuturoejacorrelacionadacomopreçodo\begin{equation}
$ativo),$
\end{equation}eseránotadopor\begin{equation}
7.0
\end{equation}Emalgunspontosdo\begin{equation}
\left( livro,\right)
\end{equation}estaremoslidandocom\begin{equation}
\left( posições,\right)
\end{equation}quepodemerindistintamentechamadasde\begin{equation}
\left( carteiras,\right)
\end{equation}\begin{equation}
\frac{port}{olios}
\end{equation}\begin{equation}
$etc.$
\end{equation}NotaremosporIlovalorfinanceirolíquidodeuma\begin{equation}
$posição.$
\end{equation}Algumaconfusãopodeser\begin{equation}
\left( feita,\right)
\end{equation}poisamesmadefiniçãoéàsvezesnotadapor\begin{equation}
\left( V,\right)
\end{equation}nosentidoqueumaposiçãode\begin{equation}
“opçõesnão
\end{equation}deixadeserumaopçãogenérica\begin{equation}
$(como$
\end{equation}asdotipo8anteriormente\begin{equation}
$isto),$
\end{equation}possuindoumvalor\begin{equation}
Vquerelaciona - se
\end{equation}comosfatoresfundamentaisdamesmaformaqueovalordeumaopçãoindividual\begin{equation}
$faz.$
\end{equation}\begin{equation}
\left( Contudo,\right)
\end{equation}deacordoomo\begin{equation}
\left( contexto,\right)
\end{equation}\begin{equation}
pode - se
\end{equation}dizerquesemprequeestivermosdesignandoumaarteiraporVelaestarásendoabordadapelopontodevistasintéticosegundoqualumacarteiraequivaleaumaúnica\begin{equation}
\left( opção,\right)
\end{equation}esemprequeaestivermosesignandoporIIelaestarásendoabordadapelopontodevistaanalíticoegundooqualumacarteiraéumconjuntodeopçõesemquantidades\begin{equation}
$efinidas.$
\end{equation}Serácomumlidartambémcomoconceitode\begin{equation}
\left( resultado,\right)
\end{equation}quenotamosportOresultadodeumaposiçãoéavariação\begin{equation}
—
\end{equation}positiva\begin{equation}
- deseu + ounegativa
\end{equation}valor\includegraphics[width=0.8\textwidth]{output/image_250png}18\begin{equation}
$Opções:$
\end{equation}Operandoa\begin{equation}
\left( Volatilida,\right)
\end{equation}\begin{equation}
\left( líquido,\right)
\end{equation}entredoisestados\begin{equation}
$quaisquer.$
\end{equation}\begin{equation}
\left( Formalmente,\right)
\end{equation}X\begin{equation}
$=$
\end{equation}DV\begin{equation}
$(ou$
\end{equation}R\begin{equation}
$=$
\end{equation}DiMuitas\begin{equation}
\left( vezes,\right)
\end{equation}osdoisestadosentreosquaissedáoresultadodaoperaçãosãaentradana\begin{equation}
\left( posição,\right)
\end{equation}nadata\begin{equation}
$atual;$
\end{equation}eoexercícioda\begin{equation}
\left( posição,\right)
\end{equation}nadatadvencimentodasopçõesquea\begin{equation}
$formam.$
\end{equation}Esteé ocasoimplícitoemtodosgráficos\begin{equation}
R x
\end{equation}\begin{equation}
$$*$
\end{equation}queserãovistosnapróxima\begin{equation}
$seção.$
\end{equation}Opresentelivrotrabalhacomoconceitodemarcaramercado\begin{equation}
$(7rk-$
\end{equation}\begin{equation}
$fo.$
\end{equation}\begin{equation}
$market)$
\end{equation}emquesecalculaoresultadodetodasaspontasno\begin{equation}
\left( dia,\right)
\end{equation}entreofechamentododiaanterioreofechamentododia\begin{equation}
\left( atual,\right)
\end{equation}\begin{equation}
desconta - se
\end{equation}ocusto\begin{equation}
$de;$
\end{equation}carregamento\begin{equation}
$(custo$
\end{equation}de\begin{equation}
$oportunidade)$
\end{equation}parainstrumentosqueimpliquemcaixa\begin{equation}
$(ativos$
\end{equation}avistaeprêmiosde\begin{equation}
$opções)$
\end{equation}e\begin{equation}
- se + soma
\end{equation}oresultadodedaytradeou\begin{equation}
$giro.$
\end{equation}Para\begin{equation}
\left( simplificar,\right)
\end{equation}posiçõesdeativosavistaeopçõessãocalculadas\begin{equation}
considerando - se
\end{equation}opreçodefechamentododiaanterioracrescidodeum\begin{equation}
\left( CDI,\right)
\end{equation}demodoajádescontarocustode\begin{equation}
$carregamento.$
\end{equation}Umafórmulaparaoresultadodeumdiadeumaposição\begin{equation}
$é:$
\end{equation}\begin{equation}
$R=5R,$
\end{equation}OndeRéoresultadoda\begin{equation}
i - ésima
\end{equation}\begin{equation}
$ponta.$
\end{equation}Seestapontaforumaposiçãoemativoavistaou\begin{equation}
\left( opções,\right)
\end{equation}seuresultadoédado\begin{equation}
$por:$
\end{equation}\begin{equation}
\left( R,\right)
\end{equation}\begin{equation}
$=$
\end{equation}\begin{equation}
\left( \mathtt{\text{<sympy.assumptions.ask.AssumptionKeys object at 0x0000023BE3D4AF90>}},\right)
\end{equation}x\begin{equation}
$[P;.$
\end{equation}\begin{equation}
—Piti
\end{equation}x\begin{equation}
$(1$
\end{equation}\begin{equation}
$+CDD]+$
\end{equation}Riagiroonde\begin{equation}
\left( \mathtt{\text{<sympy.assumptions.ask.AssumptionKeys object at 0x0000023BE3D4AF90>}},\right)
\end{equation}éaquantidadedoinstrumentoaofinaldoúltimo\begin{equation}
\left( pregão,\right)
\end{equation}\begin{equation}
\left( p,\right)
\end{equation}ésua\begin{equation}
\left( cotação,\right)
\end{equation}osíndicesfe\begin{equation}
$/—$
\end{equation}1\begin{equation}
referem - se
\end{equation}aosfechamentosdehojeede\begin{equation}
\left( ontem,\right)
\end{equation}é\begin{equation}
\left( R,\right)
\end{equation}shoéoresultadode\begin{equation}
\left( giro,\right)
\end{equation}ou\begin{equation}
\left( seja,\right)
\end{equation}oresultadodasoperaçõesrealizadasnéprópriodia\begin{equation}
$(hoje).$
\end{equation}Esteresultadodegiropodefechamentotodososfradesrealizadosno\begin{equation}
\left( dia,\right)
\end{equation}ou\begin{equation}
$seja:$
\end{equation}\begin{equation}
\left( Ro,\right)
\end{equation}giro\begin{equation}
$=5,0$
\end{equation}\begin{equation}
$x(p,s$
\end{equation}\begin{equation}
$Pit)$
\end{equation}jondeoíndice\begin{equation}
$/significa$
\end{equation}cada\begin{equation}
$/radeexecutado$
\end{equation}\begin{equation}
$(compra$
\end{equation}ou\begin{equation}
$venda)no$
\end{equation}instrumere\begin{equation}
$|$
\end{equation}\begin{equation}
R x
\end{equation}5º\begin{equation}
Note - se
\end{equation}queosegundoexemplomostraumaposiçãoqueprevêum\begin{equation}
\left( to,\right)
\end{equation}\begin{equation}
\left( \mathtt{\text{<sympy.assumptions.ask.AssumptionKeys object at 0x0000023BE3D4AF90>}},\right)
\end{equation}significaaquantidade\begin{equation}
\left( negociada,\right)
\end{equation}e\begin{equation}
\left( p,\right)
\end{equation}\begin{equation}
$,$
\end{equation}\begin{equation}
$,$
\end{equation}Significaopreçoemque\begin{equation}
$fo:$
\end{equation}\begin{equation}
$realizado.$
\end{equation}Sea\begin{equation}
i - ésima
\end{equation}pontaforumcontrato\begin{equation}
\left( futuro,\right)
\end{equation}liquidadoporajustes diários\begin{equation}
\left( R, \  será\right)
\end{equation}ovalordoajuste\begin{equation}
\left( diário,\right)
\end{equation}\begin{equation}
$ou:$
\end{equation}\begin{equation}
$R=0,x$
\end{equation}E\begin{equation}
—
\end{equation}\begin{equation}
$Pira]$
\end{equation}\begin{equation}
$+$
\end{equation}Riosercalculado\begin{equation}
levando - se
\end{equation}i\begin{equation}
$Opções:$
\end{equation}OperandoaVolatilidade19\begin{equation}
Observe - se
\end{equation}queaquiestamosfalandodefuturosde\begin{equation}
\left( ações,\right)
\end{equation}\begin{equation}
\left( moedas,\right)
\end{equation}commodities\begin{equation}
\left( ou,\right)
\end{equation}\begin{equation}
\left( generalizadamente,\right)
\end{equation}defuturos\begin{equation}
$autênticos.$
\end{equation}Mercados\begin{equation}
$spo?$
\end{equation}liquidadosporajustes\begin{equation}
\left( diários,\right)
\end{equation}comoéocasodoDI\begin{equation}
\left( futuro,\right)
\end{equation}têmoajuste diáriodescontadodeum\begin{equation}
\left( CDI,\right)
\end{equation}esuaformadecálculoésemelhanteàdosativose\begin{equation}
$opções.$
\end{equation}Oresultadoacumuladodaoperaçãoaolongodotemposóéconhecido\begin{equation}
- se + somando
\end{equation}osresultadosdiáriosacumuladospelo\begin{equation}
$CDI.$
\end{equation}Neste\begin{equation}
\left( livro,\right)
\end{equation}aúnicaincorreçãoquesecometeénãocarregarosresultadosdiáriospeloCDInasua\begin{equation}
\left( acumulação,\right)
\end{equation}demodoatornaraverificaçãodascontasmais\begin{equation}
$simples.$
\end{equation}Dificilmenteentraremosemexemplosquenecessitemdocálculoderesultado\begin{equation}
\left( degito,\right)
\end{equation}poisé\begin{equation}
\left( comum,\right)
\end{equation}principalmenteem\begin{equation}
\left( simulações,\right)
\end{equation}considerarqueostradessãoexecutadosaopreçodefechamentodomercadono\begin{equation}
$dia.$
\end{equation}GráficosQuantoaosgráficosapresentadosno\begin{equation}
\left( livro,\right)
\end{equation}elessãodedois\begin{equation}
$tipos:$
\end{equation}curvasVx S\begin{equation}
payoffs
\end{equation}nasquaisestárepresentadoovalorouprêmiodeumaopçãoouposiçãoemfunçãodeváriosvalorespossíveisde\begin{equation}
\left( 5,\right)
\end{equation}ougráficosderesultadoRx\begin{equation}
$$$
\end{equation}deuma\begin{equation}
$operação.$
\end{equation}Ascurvas\begin{equation}
V x
\end{equation}\begin{equation}
$$S*são$
\end{equation}casosparticularesdecurvasVx\begin{equation}
\left( 5,\right)
\end{equation}emostramopayojfatribuídonoexercícioaopossuidordeumadadaposiçãoem\begin{equation}
$opções.$
\end{equation}Estefluxodecaixanãoprecisasernecessariamentenãonegativoparatodososvaloresde\begin{equation}
$$*$
\end{equation}\begin{equation}
$(o$
\end{equation}éemcasodecomprasecadeuma\begin{equation}
$opção):$
\end{equation}muitasestratégiascomopçõespremiamopossuidorcomumvalordeexercícionegativoemcertas\begin{equation}
$situações.$
\end{equation}Jáográficoderesultadodeumaposiçãoéasomaadeseu\begin{equation}
$payorfV*$
\end{equation}x\begin{equation}
$S*com$
\end{equation}oencaixe\begin{equation}
$(ou$
\end{equation}\begin{equation}
$desencaixe)$
\end{equation}realizadonadataatualparaconstruira\begin{equation}
$posição.$
\end{equation}Nocasodacompradeuma\begin{equation}
\left( opção,\right)
\end{equation}háumdesencaixenadata\begin{equation}
\left( atual,\right)
\end{equation}quepodenãosercobertopeloexercício\begin{equation}
$(no$
\end{equation}casode\begin{equation}
V ser
\end{equation}\begin{equation}
\left( nulo,\right)
\end{equation}oumesmopositivoporém\begin{equation}
$inferior),$
\end{equation}daíumapossibilidadedeprejuízo\begin{equation}
$(resultado$
\end{equation}\begin{equation}
$negativo)$
\end{equation}na\begin{equation}
$posição.$
\end{equation}Por\begin{equation}
\left( isso,\right)
\end{equation}acurva\begin{equation}
R x
\end{equation}\begin{equation}
S“de
\end{equation}umacomprasecadeopçãoapresentaumaporção\begin{equation}
$-$
\end{equation}\begin{equation}
\left( negativa,\right)
\end{equation}equivalenteaoprêmioquesepagoupela\begin{equation}
$opção.$
\end{equation}Abaixoestãodoisexemplosdecurvas\begin{equation}
V x
\end{equation}\begin{equation}
$S*e$
\end{equation}seusrespectivosgráficosfluxodecaixanegativono\begin{equation}
\left( exercício,\right)
\end{equation}equeéconstruídacomencaixe\begin{equation}
$hoje.$
\end{equation}Emcada\begin{equation}
\left( caso,\right)
\end{equation}estárepresentadonográficoderesultadoaquantidadededinheirodesencaixadaouencaixadaparamontara\begin{equation}
$posição:$
\end{equation}\includegraphics[width=0.8\textwidth]{output/image_252png}\begin{equation}
$Opções:$
\end{equation}Operandoa\begin{equation}
\left( Volatilida,\right)
\end{equation}\begin{equation}
$s.$
\end{equation}À\begin{equation}
$S*$
\end{equation}\begin{equation}
$|$
\end{equation}\section{PREÇOS}\section{DAS}\section{OPÇÕES}ValorintrínsecoeprêmioderiscoDandoseguimentoàdiscussãoiniciadanocapítulo\begin{equation}
\left( anterior,\right)
\end{equation}emque\begin{equation}
$*$
\end{equation}vimosrapidamentequaisosfatoresqueinfluenciamoprêmiodeuma\begin{equation}
\left( opção,\right)
\end{equation}\begin{equation}
$:$
\end{equation}fazemosaquiumainterpretaçãomaisrigorosadessa\begin{equation}
$influência.$
\end{equation}\begin{equation}
$*$
\end{equation}ValorintrínsecoAté\begin{equation}
\left( agora,\right)
\end{equation}aúnicacoisaquesabemossobreopreçodeumaopçãoéqueeledeveserovalorpresentedaexpectativadevalordeexercícioemum\begin{equation}
$-$
\end{equation}mundoneutroao\begin{equation}
$risco.$
\end{equation}Estevalordeexercícioéobtido\begin{equation}
carregando - se
\end{equation}\begin{equation}
$$$
\end{equation}aos\begin{equation}
“juros
\end{equation}7peloprazo\begin{equation}
£
\end{equation}e\begin{equation}
achando - se
\end{equation}adiferençasobre\begin{equation}
$X.$
\end{equation}Seestadiferençafor\begin{equation}
$*$
\end{equation}favorávelao\begin{equation}
\left( exercício,\right)
\end{equation}esteéovalordeexercícioda\begin{equation}
$opção;$
\end{equation}\begin{equation}
\left( senão,\right)
\end{equation}seuvalorde\begin{equation}
$*$
\end{equation}exercícioé\begin{equation}
$zero.$
\end{equation}Uma\begin{equation}
$ca//$
\end{equation}valeomaiorentrezeroe\begin{equation}
$S*—$
\end{equation}Kumaputvaleomaior\begin{equation}
$*$
\end{equation}entrezeroe\begin{equation}
K—
\end{equation}5ºEmambosos\begin{equation}
\left( casos,\right)
\end{equation}umavezqueoprêmiodaopçãoéuma\begin{equation}
$*$
\end{equation}quantidadededinheiroa\begin{equation}
\left( vista,\right)
\end{equation}\begin{equation}
- se + traz
\end{equation}ovalordeexercícioavalorpresente\begin{equation}
$.$
\end{equation}paraseteraprimeiraaproximaçãodopreçodeuma\begin{equation}
$opção:$
\end{equation}\begin{equation}
\left( C,\right)
\end{equation}\begin{equation}
$=VPGmnaxlO,$
\end{equation}\begin{equation}
$VE(S)—-K))$
\end{equation}\begin{equation}
$=$
\end{equation}\begin{equation}
\left( maxi_{0}, \  S—\right)
\end{equation}\begin{equation}
$VP(K)]$
\end{equation}\begin{equation}
\left( P,\right)
\end{equation}\begin{equation}
$=VP(max(0,$
\end{equation}\begin{equation}
K—
\end{equation}\begin{equation}
$VF(S))$
\end{equation}\begin{equation}
$=$
\end{equation}\begin{equation}
$max(0,$
\end{equation}\begin{equation}
$VP(K)—S]$
\end{equation}Estassãoasexpressõesparaovalorintrínseco\begin{equation}
$(Ce$
\end{equation}\begin{equation}
$?,$
\end{equation}genericamente\begin{equation}
$V)$
\end{equation}\begin{equation}
$'$
\end{equation}das\begin{equation}
$opções.$
\end{equation}Ovalorintrínsecodeumaopçãoéaporçãodeseupreçoquesedeveàvantagemrealque\begin{equation}
\left( \mathtt{\text{S}},\right)
\end{equation}emrelaçãoaX\begin{equation}
$proporciona.$
\end{equation}Ovalorintrínsecodeumaopçãopodeserzeroemqualquer\begin{equation}
\left( tempo,\right)
\end{equation}aindaqueseuprêmionuncaosejaantesdo\begin{equation}
$vencimento.$
\end{equation}\includegraphics[width=0.8\textwidth]{output/image_254png}\begin{equation}
$Opções:$
\end{equation}OperandoaVolatilidade19\begin{equation}
\left[ 8\right]
\end{equation}Comadefinição\begin{equation}
\left( acima,\right)
\end{equation}\begin{equation}
pode - se
\end{equation}traçarumgráficodeVxSem\begin{equation}
\left( qualquer,\right)
\end{equation}\begin{equation}
\left( instante,\right)
\end{equation}queéumareproduçãodacurva\begin{equation}
V x
\end{equation}\begin{equation}
$S*trazida$
\end{equation}avalorpresente\begin{equation}
$.$
\end{equation}\begin{equation}
$(exemplificando$
\end{equation}comuma\begin{equation}
$caí):$
\end{equation}É\begin{equation}
\left( C,\right)
\end{equation}\begin{equation}
\operatorname{VP}{\left(K \right)}
\end{equation}Convexidade\begin{equation}
\left( Contudo,\right)
\end{equation}sefizermosumaplotagemdeváriospares\begin{equation}
$(5,$
\end{equation}\begin{equation}
$C)$
\end{equation}\begin{equation}
$observados.$
\end{equation}duranteumoudoisdiasde\begin{equation}
\left( mercado,\right)
\end{equation}encontraremosumgráfico\begin{equation}
\left( diferente,\right)
\end{equation}\begin{equation}
$deste.$
\end{equation}Ográficodospreçosreaisdas\begin{equation}
\left\lfloor{\frac{ca}{s}}\right\rfloor
\end{equation}emfunçãodopreçoSé\begin{equation}
$suave enão,$
\end{equation}\begin{equation}
“quebrado”
\end{equation}comoográficodovalor\begin{equation}
$intrínseco.$
\end{equation}Nossaplotagemestaria\begin{equation}
$mais:$
\end{equation}pertodeparecercomafigura\begin{equation}
$abaixo:$
\end{equation}Cc\begin{equation}
\operatorname{VP}{\left(K \right)}
\end{equation}Emoutras\begin{equation}
\left( palavras,\right)
\end{equation}acurvaCx\begin{equation}
$$$
\end{equation}apresentauma\begin{equation}
\left( convexidade,\right)
\end{equation}queafaseaproximarassintoticamentedacurva\begin{equation}
\frac{C}{x}
\end{equation}5\begin{equation}
$(quebrada).$
\end{equation}Osextremosdambasascurvassãocadavezmais\begin{equation}
\left( próximos,\right)
\end{equation}\begin{equation}
\left( mas,\right)
\end{equation}em\begin{equation}
\left( teoria,\right)
\end{equation}jamaissetoca\begin{equation}
Ouções!
\end{equation}OperandoaVolatilidade23Nocentrodo\begin{equation}
\left( gráfico,\right)
\end{equation}paraumvalordeSpróximoa\begin{equation}
\left( \operatorname{VP}{\left(K \right)},\right)
\end{equation}adistânciaentreasduascurvasé\begin{equation}
$máxima.$
\end{equation}Podeparecerclaroporquedeveser\begin{equation}
$assim.$
\end{equation}Emprimeiro\begin{equation}
\left( lugar,\right)
\end{equation}nenhuma\begin{equation}
\left( opção,\right)
\end{equation}pormaisimprovávelquesejaseu\begin{equation}
\left( exercício,\right)
\end{equation}podevalerzeroantesdo\begin{equation}
$vencimento.$
\end{equation}\begin{equation}
Excluindo - se
\end{equation}custosde\begin{equation}
\left( corretagem,\right)
\end{equation}qualquerumcompraumaE\begin{equation}
\left( opção,\right)
\end{equation}qualquerque\begin{equation}
\left( seja,\right)
\end{equation}seelalheforoferecidaapreço\begin{equation}
$zero:$
\end{equation}equivalea\begin{equation}
$*$
\end{equation}ganhardepresenteumbilhetede\begin{equation}
\left( loteria,\right)
\end{equation}quepodeterumachancemuito\begin{equation}
$*$
\end{equation}pequenadeser\begin{equation}
\left( sorteado,\right)
\end{equation}maschanceestaquejamaisserá\begin{equation}
$nula.$
\end{equation}\begin{equation}
\left( Então,\right)
\end{equation}mesmo\begin{equation}
$|$
\end{equation}opçõesdevalorintrínseconulotêmdecustaralguma\begin{equation}
$coisa.$
\end{equation}Podeserintuitivo\begin{equation}
$*$
\end{equation}\begin{equation}
\left( que,\right)
\end{equation}quantomaisabaixode\begin{equation}
\operatorname{VP}{\left(K \right)}
\end{equation}estiver\begin{equation}
\left( 5,\right)
\end{equation}isto\begin{equation}
\left( é,\right)
\end{equation}quantomaisimprovávelo\begin{equation}
“
\end{equation}exercícioda\begin{equation}
\left( opção,\right)
\end{equation}tantomenosela\begin{equation}
$valha.$
\end{equation}Issoestádeacordocomnosso\begin{equation}
$:$
\end{equation}grá\begin{equation}
$fico.$
\end{equation}OlimitedestasituaçãoéSinfinitamenteabaixode\begin{equation}
\left( \operatorname{VP}{\left(K \right)},\right)
\end{equation}oquetorna\begin{equation}
$*$
\end{equation}opreço\begin{equation}
€
\end{equation}virtualmente\begin{equation}
$nulo.$
\end{equation}\begin{equation}
\left( Agora,\right)
\end{equation}sabemosporqueaporçãoinferiordográficodeveserpositivae\begin{equation}
$assintótica;$
\end{equation}restasaberpor queaporçãosuperior\begin{equation}
$*$
\end{equation}tambémo\begin{equation}
$é.$
\end{equation}Operadoresdomercadodeopçõesconhecembemaoperaçãochamada\begin{equation}
$*$
\end{equation}\begin{equation}
\left( reversão,\right)
\end{equation}emquesevendeumativoe\begin{equation}
compra - se
\end{equation}amesmaquantidadede\begin{equation}
\left\lfloor{\frac{ca}{s}}\right\rfloor
\end{equation}\begin{equation}
$*$
\end{equation}sobre\begin{equation}
\left( ele,\right)
\end{equation}em\begin{equation}
$suma:$
\end{equation}\begin{equation}
- se + troca
\end{equation}umaquantidadedeativopor\begin{equation}
$ca/is.$
\end{equation}Essaoperaçãotemdoisúnicos\begin{equation}
$efeitos:$
\end{equation}\begin{equation}
$a)$
\end{equation}gerar\begin{equation}
\left( caixa,\right)
\end{equation}pois\begin{equation}
recebe - se
\end{equation}pelavendadeumativo\begin{equation}
$*$
\end{equation}\begin{equation}
epaga - se
\end{equation}peloprêmiodeuma\begin{equation}
\left( opção,\right)
\end{equation}queébem\begin{equation}
$menor;$
\end{equation}\begin{equation}
$b)$
\end{equation}eliminaroriscodebaixado\begin{equation}
\left( ativo,\right)
\end{equation}poisuma\begin{equation}
\left\lfloor{\frac{ca}{é}}\right\rfloor
\end{equation}umprodutoquemantémaexposiçãodurante\begin{equation}
$*$
\end{equation}\begin{equation}
\left( aalta,\right)
\end{equation}eareduzatézerona\begin{equation}
$baixa.$
\end{equation}Qualquergeraçãodecaixaemummercadoeficienteequivaleatomarumempréstimoàstaxasdejuro\begin{equation}
$correntes.$
\end{equation}Esseempréstimoterádeserquitadoemalgumadata\begin{equation}
\left( futura,\right)
\end{equation}erealmenteoénesta\begin{equation}
$-$
\end{equation}\begin{equation}
\left( operação,\right)
\end{equation}\begin{equation}
\left( pois,\right)
\end{equation}duranteodiado\begin{equation}
\left( exercício,\right)
\end{equation}orevertedoréobrigadoarecomprar\begin{equation}
$*$
\end{equation}seu\begin{equation}
$ativo.$
\end{equation}EleorecomprapelopreçodeexercícioXdaopçãoquevendeu\begin{equation}
$|$
\end{equation}\begin{equation}
$(assumindo-se$
\end{equation}queaopçãodeu\begin{equation}
$exercício).$
\end{equation}\begin{equation}
\left( Então,\right)
\end{equation}quemfazumaoperaçãode\begin{equation}
$*$
\end{equation}reversão\begin{equation}
\left( está,\right)
\end{equation}emtermosdefluxode\begin{equation}
\left( caixa,\right)
\end{equation}recebendo\begin{equation}
$($—$
\end{equation}\begin{equation}
$C)hoje$
\end{equation}parapagarKnadatade\begin{equation}
$vencimento.$
\end{equation}Senósescrevermosqueestefluxodecaixadeveseenquadrarnastaxasdejurocorrentesno\begin{equation}
\left( mercado,\right)
\end{equation}teremos\begin{equation}
$que:$
\end{equation}\begin{equation}
$(S—-OD=$
\end{equation}\begin{equation}
\operatorname{VP}{\left(K \right)}
\end{equation}ou\begin{equation}
$C=S-VP(K)$
\end{equation}queéaexpressãodovalorintrínsecodeuma\begin{equation}
$ca//$
\end{equation}emfunçãode\begin{equation}
$$$
\end{equation}naporçãosuperiordo\begin{equation}
$gráfico.$
\end{equation}\begin{equation}
\left( Então,\right)
\end{equation}ovalorintrínsecodeuma\begin{equation}
\left\lfloor{\frac{ca}{é}}\right\rfloor
\end{equation}exatamenteaqueleprêmiopeloqualumrevertedortrocaráativoporelatomandodinheiroàstaxasdejurodo\begin{equation}
$mercado.$
\end{equation}Àpergunta\begin{equation}
$é:$
\end{equation}jáqueumareversãonãoéapenasum\includegraphics[width=0.8\textwidth]{output/image_256png}24\begin{equation}
$Opções:$
\end{equation}Operandoa\begin{equation}
\left( Volatilidade,\right)
\end{equation}\begin{equation}
\left( empréstimo,\right)
\end{equation}mastambémuma\begin{equation}
\left( proteção,\right)
\end{equation}nãoseriajustoqueorevertedordevessepagarumpoucomaispela\begin{equation}
$ca//?$
\end{equation}Colocandoemoutros\begin{equation}
$termos:$
\end{equation}quemsearriscariaadarliquidezparaumrevertedor\begin{equation}
$(e$
\end{equation}fazeraoperação\begin{equation}
$contrária);$
\end{equation}sabendo\begin{equation}
\left( que,\right)
\end{equation}namelhordas\begin{equation}
\left( hipóteses,\right)
\end{equation}estaráaplicandoseudinheiroàstaxas\begin{equation}
$correntes?$
\end{equation}Aconclusãoéde\begin{equation}
\left( que,\right)
\end{equation}assimcomoas\begin{equation}
\left\lfloor{\frac{ca}{s}}\right\rfloor
\end{equation}deexercícioimprovávelnãopodemcustar\begin{equation}
\left( zero,\right)
\end{equation}pelaoportunidadequeelas\begin{equation}
\left( encerram,\right)
\end{equation}\begin{equation}
\left\lfloor{\frac{ca}{s}}\right\rfloor
\end{equation}deexercíciprovávelnãopodemcustarapenasovalor\begin{equation}
\left( intrínseco,\right)
\end{equation}pelaoportunidadequeelasabririampara\begin{equation}
$revertedores.$
\end{equation}\begin{equation}
$(Há$
\end{equation}umasituaçãoemquenãovaleoqueseacaboude\begin{equation}
$explicar:$
\end{equation}é\begin{equation}
\left( quando,\right)
\end{equation}háfaltadeativodisponívelpara\begin{equation}
$reverter.$
\end{equation}Nesse\begin{equation}
\left( caso,\right)
\end{equation}nadaimpedealgumasopçõesdeseremnegociadasatémesmoabaixodovalor\begin{equation}
$intrínseco.$
\end{equation}Osoperdoresqueconhecemaoperaçãoeseupotencialdeganhocostumamprocuraravidamenteporativosquepossamreverternessas\begin{equation}
$horas.$
\end{equation}Seexisteum\begin{equation}
merca - “
\end{equation}dodealugueldeativo\begin{equation}
\left( disponível,\right)
\end{equation}oproblemadosoperadoresestáresolvidmas\begin{equation}
- se + tem
\end{equation}dededuzirdeseuganhopotencialoquantosepagarádealuguelChegamosentãoaumaexplicaçãoplausíveldeporqueosprêmios\begin{equation}
\left( das,\right)
\end{equation}opçõestêmdesersuperioresaseuvalor\begin{equation}
$intrínseco.$
\end{equation}Issonão\begin{equation}
\left( explica,\right)
\end{equation}por\begin{equation}
\left( si,\right)
\end{equation}\begin{equation}
$convexidade;$
\end{equation}ográficoseguinteatendeatodasasnossas\begin{equation}
\left( exigências,\right)
\end{equation}mas\begin{equation}
$não:$
\end{equation}é\begin{equation}
$convexo:$
\end{equation}\begin{equation}
\operatorname{VP}{\left(K \right)}
\end{equation}Convexidadeéumapropriedade\begin{equation}
\left( que,\right)
\end{equation}emtermos\begin{equation}
\left( geométricos,\right)
\end{equation}querdizequalquerpontointermediárioaoutrosdoisdeve\begin{equation}
- se + situar
\end{equation}abaixodaretaqueuneesses\begin{equation}
$dois.$
\end{equation}Sevocêtentarconstruirumgráficonoqualtodosospontostenhamesta\begin{equation}
\left( característica,\right)
\end{equation}vocêchegaráinevitavelmenteaumdesenhocomooqueobservamosno\begin{equation}
\left( mercado,\right)
\end{equation}ou\begin{equation}
$seja:$
\end{equation}toim\begin{equation}
$opções:$
\end{equation}OperandoaVolatilidadeOperaçãobutterflyExistedefatoummotivoparaqueospreçosdasopçõessejam\begin{equation}
$convexos:$
\end{equation}éaoperaçãode\begin{equation}
$butterffy.$
\end{equation}Suponhamostertrês\begin{equation}
\left\lfloor{\frac{ca}{s}}\right\rfloor
\end{equation}depreçosdeexercício\begin{equation}
\left( KZ,\right)
\end{equation}K2e\begin{equation}
\left( K_{3},\right)
\end{equation}ondeK2éamédiaentreAZeX3\begin{equation}
$(por$
\end{equation}\begin{equation}
\left( exemplo,\right)
\end{equation}\begin{equation}
\left( 100,\right)
\end{equation}110e\begin{equation}
$120).$
\end{equation}Umabutterfly\begin{equation}
oorboleta
\end{equation}entreessastrêsopçõeséuma\begin{equation}
$posição:$
\end{equation}compradana\begin{equation}
\left( primeira,\right)
\end{equation}compradanaterceiraevendidaduasvezesnadomeio\begin{equation}
$(porexemplo,$
\end{equation}compradaem500lotesdaopçãode\begin{equation}
\left( KZ,\right)
\end{equation}compradaem500lotesdadeX3evendida em\begin{equation}
1.0
\end{equation}lotesdadestrike\begin{equation}
$K2).$
\end{equation}Essaéumaposiçãoquepossuiumvalordeexercíciopositivoounulono\begin{equation}
$vencimento.$
\end{equation}Seu\begin{equation}
$payoffé:$
\end{equation}\begin{equation}
$V+*$
\end{equation}\begin{equation}
$|$
\end{equation}\begin{equation}
$|$
\end{equation}\begin{equation}
$[$
\end{equation}\begin{equation}
$*$
\end{equation}KIK3óAmaneiradesechegaraeste\begin{equation}
\left\lfloor{\frac{\left\lfloor{\tilde{\infty} pn}\right\rfloor}{é}}\right\rfloor
\end{equation}\begin{equation}
$simples.$
\end{equation}Lembremos\begin{equation}
\left( que,\right)
\end{equation}acimado\begin{equation}
\left( strike,\right)
\end{equation}umacallvale\begin{equation}
$S*—$
\end{equation}Kabaixode\begin{equation}
\left( KT,\right)
\end{equation}astrêsopçõesvalem\begin{equation}
$zero;$
\end{equation}entreXZe\begin{equation}
\left( K_{2},\right)
\end{equation}apenasaprimeiravale\begin{equation}
$algo:$
\end{equation}vale\begin{equation}
$S*—$
\end{equation}\begin{equation}
$KT;$
\end{equation}entreK2e\begin{equation}
\left( X_{5},\right)
\end{equation}aposiçãovale\begin{equation}
$(S”$
\end{equation}KU\begin{equation}
$-$
\end{equation}A\begin{equation}
$S*-K2)$
\end{equation}Em\begin{equation}
\left( K_{3},\right)
\end{equation}estevaloré\begin{equation}
\left( zero,\right)
\end{equation}eassimpermanece\begin{equation}
\left( além,\right)
\end{equation}poisacimadeK3acarteira\begin{equation}
$vale:$
\end{equation}\begin{equation}
$(S*—$
\end{equation}\begin{equation}
$KI)$
\end{equation}\begin{equation}
—
\end{equation}\begin{equation}
$2(S*$
\end{equation}\begin{equation}
—
\end{equation}\begin{equation}
$KD)$
\end{equation}\begin{equation}
$+$
\end{equation}\begin{equation}
$(S*-K3)=KI—-2K2$
\end{equation}\begin{equation}
$+$
\end{equation}KS\includegraphics[width=0.8\textwidth]{output/image_258png}26\begin{equation}
$Opções:$
\end{equation}Operandoà\begin{equation}
\left( Volatilidate,\right)
\end{equation}ComoK2é\begin{equation}
$(K1$
\end{equation}\begin{equation}
$+$
\end{equation}\begin{equation}
$K3)/2,$
\end{equation}oresultadoacimaé\begin{equation}
$nulo.$
\end{equation}Umacarteiraquevalenomínimozeroenomáximoalgumvalor\begin{equation}
positivo”
\end{equation}nadatadevencimentodevecustaralgumprêmiopositivoparaser\begin{equation}
$adquirida.$
\end{equation}OprêmioquesepagaporestaposiçãoéC7\begin{equation}
—
\end{equation}202\begin{equation}
$+$
\end{equation}\begin{equation}
$C3;$
\end{equation}quedevesermaiorqu\begin{equation}
$zero.$
\end{equation}IstosóocorreseC2formenorqueamédiaentreCZe\begin{equation}
$C3.$
\end{equation}Compouco\begin{equation}
\left( esforço,\right)
\end{equation}épossívelmodificareste\begin{equation}
\left( raciocínio,\right)
\end{equation}elaborado\begin{equation}
$para;$
\end{equation}oespaçodospreçosde\begin{equation}
\left( exercício,\right)
\end{equation}paraoespaçodosvaloresde\begin{equation}
$S-$
\end{equation}isto\begin{equation}
$é:$
\end{equation}nossa\begin{equation}
$|$
\end{equation}observaçãoprimáriadeconvexidade\begin{equation}
dá - se
\end{equation}emfunçãode\begin{equation}
\left( 5,\right)
\end{equation}enãoem\begin{equation}
$função:$
\end{equation}de\begin{equation}
$K.$
\end{equation}Bastaimaginarumaopção\begin{equation}
€C
\end{equation}\begin{equation}
\left( ocupando,\right)
\end{equation}devidoaomovimentode\begin{equation}
\left( 5,\right)
\end{equation}\begin{equation}
0.0
\end{equation}lugardecadaumadastrêsopções\begin{equation}
$anteriores:$
\end{equation}seSestáem\begin{equation}
\left( baixa,\right)
\end{equation}aopção\begin{equation}
$Cvale:$
\end{equation}\begin{equation}
$C3:$
\end{equation}seSestáem\begin{equation}
\left( alta,\right)
\end{equation}vale\begin{equation}
$C$;$
\end{equation}eseSé\begin{equation}
\left( intermediário,\right)
\end{equation}vale\begin{equation}
\left( C_{2},\right)
\end{equation}Emcada\begin{equation}
$uma;$
\end{equation}dessas\begin{equation}
\left( possibilidades,\right)
\end{equation}haveráoutrasduasopçõesemmeioàsquaisCdeveráatenderàcondiçãode\begin{equation}
$convexidade.$
\end{equation}\begin{equation}
\left( Portanto,\right)
\end{equation}intermediárioC2nãopossaescaparàmédiaentreosvaloresextremosCZe\begin{equation}
$C3.$
\end{equation}Aoperaçãode4ufterflynãoprecisanecessariamentesermontada\begin{equation}
$com:$
\end{equation}opçõesigualmentedistanciadasentresiÉpossívelmontaraoperação\begin{equation}
$com.$
\end{equation}preciso\begin{equation}
$um:$
\end{equation}Íj\begin{equation}
$|$
\end{equation}\begin{equation}
$Telebrás;$
\end{equation}aúnicacoisaquediferenosdoismercadoséoníveldeoscilaçãoopçõesquenãotenhamstrikesigualmente\begin{equation}
$intercalados;$
\end{equation}sóéalgebrismoparaencontrarasproporçõesqueanulamacarteirapara5\begin{equation}
$>$
\end{equation}\begin{equation}
$AS.$
\end{equation}Estaéaformulaçãomaisgeralde\begin{equation}
\left( convexidade,\right)
\end{equation}mascomoelaimplica\begin{equation}
\left( no,\right)
\end{equation}caso\begin{equation}
\left( particular,\right)
\end{equation}eocasoparticularimplica\begin{equation}
\left( nela,\right)
\end{equation}deixemoscomo\begin{equation}
\left( está,\right)
\end{equation}\begin{equation}
$que;$
\end{equation}\begin{equation}
S€
\end{equation}\begin{equation}
$:$
\end{equation}diretamenteopreçodeumaopçãosobre\begin{equation}
5.0
\end{equation}Ovaloresperadode\begin{equation}
\left( 5,\right)
\end{equation}emum\begin{equation}
$:$
\end{equation}mundoneutroao\begin{equation}
\left( risco,\right)
\end{equation}\begin{equation}
“esperado
\end{equation}pairaumanuvemdeprobabilidades\begin{equation}
\left( que,\right)
\end{equation}quantomais\begin{equation}
\left( ampla,\right)
\end{equation}\begin{equation}
$-$
\end{equation}maisfavoreceotitulardeopção\begin{equation}
$(nunca$
\end{equation}nosesqueçamosqueotitulartemestá\begin{equation}
$claro.$
\end{equation}PrêmioderíscoPorqualquercaminhoquesetenteexplicara\begin{equation}
\left( convexidade,\right)
\end{equation}ela\begin{equation}
\left( mostra - se,\right)
\end{equation}asempreconsequênciadorisco\begin{equation}
$(ou$
\end{equation}\begin{equation}
$oportunidade)$
\end{equation}associadoàposiçãode\begin{equation}
$opções.$
\end{equation}UmaAutterflydevecustaralgopositivoporqueseuretornoé\begin{equation}
$provavel-,$
\end{equation}édeseesperarqueo\begin{equation}
$valor:$
\end{equation}\begin{equation}
\left( condição,\right)
\end{equation}tendodesermenorque\begin{equation}
\left( a,\right)
\end{equation}\begin{equation}
\left( seaorisco,\right)
\end{equation}édeseesperar quehajaumnívelderiscomensurávelevariávelqueele\begin{equation}
\left( seja,\right)
\end{equation}no\begin{equation}
\left( entanto,\right)
\end{equation}nuncaé\begin{equation}
$zero:$
\end{equation}estaéuma\begin{equation}
$certeza.$
\end{equation}\begin{equation}
\left( Então,\right)
\end{equation}opreçodabutterfly\begin{equation}
$-$
\end{equation}queemúltimaanáliseéoresponsávelpelacurvaturadográfico\begin{equation}
—
\end{equation}éapagadorisco\begin{equation}
$(ou$
\end{equation}\begin{equation}
$oportunidade)$
\end{equation}aela\begin{equation}
$associado.$
\end{equation}Issonoslevaaformularguoexcessodeprêmioqueumaopçãoapresentaacimadoseuvalorintrínsecoéumpagamentopelo\begin{equation}
\left( risco,\right)
\end{equation}éumprêmiode\begin{equation}
$risco.$
\end{equation}Eessafoiamelhorexplicaçãoatéhojeencontradaparaosprêmiosdasopçõesa\begin{equation}
“gordura”
\end{equation}queumaopçãocarregasobreográficodoseuvalorintrínsecoé\begin{equation}
$um.$
\end{equation}prêmiode\begin{equation}
$risco.$
\end{equation}AconvexidaderestringemuitoosperfisVxS\begin{equation}
$possíveis:$
\end{equation}\begin{equation}
$o.$
\end{equation}únicograudeliberdadequeresta\begin{equation}
$(a$
\end{equation}únicacaracterísticaquepodefazer\begin{equation}
$as.$
\end{equation}ÀÀ\begin{equation}
“seria
\end{equation}esperadadopreço\begin{equation}
$S.$
\end{equation}Aexperiênciaé\begin{equation}
$simples:$
\end{equation}listarosdesvios\begin{equation}
$(acrésci-$
\end{equation}curvasdiferirementre\begin{equation}
$si)$
\end{equation}é adistânciaouaberturada\begin{equation}
$curva:$
\end{equation}\begin{equation}
$x:$
\end{equation}OperandoqVolatilidade\begin{equation}
$[fig$
\end{equation}\begin{equation}
$2.1]$
\end{equation}\begin{equation}
$a?$
\end{equation}Nográfico\begin{equation}
\left( acima,\right)
\end{equation}ascurvas\begin{equation}
\left( 1, \  2\right)
\end{equation}e3representamtrêspossibilidadesdeprêmiode\begin{equation}
$opções.$
\end{equation}Oqueas\begin{equation}
$difere?$
\end{equation}Seacurvaturadográficodepreço\begin{equation}
$deve-$
\end{equation}quealtereospreçosdas\begin{equation}
$opções.$
\end{equation}Éfato\begin{equation}
\left( que,\right)
\end{equation}emmercadosmais\begin{equation}
\left( nervosos,\right)
\end{equation}asopçõessãoproporcionalmentemais\begin{equation}
$caras:$
\end{equation}umacertaopçãodeourocusta\begin{equation}
$2%$
\end{equation}dopreçodo\begin{equation}
\left( ouro,\right)
\end{equation}aopassoqueumaopçãosemelhante\begin{equation}
$(mesmo$
\end{equation}prazoemesmaprobabilidadede\begin{equation}
$exercício)$
\end{equation}deTelebráscusta\begin{equation}
$7%$
\end{equation}dopreçodedos\begin{equation}
\left( preços,\right)
\end{equation}maiornabolsadevalores quenomercadode\begin{equation}
$metais.$
\end{equation}\begin{equation}
$Chegou-$
\end{equation}seàconclusãodequeaincertezasobreovalorfuturodopreçoSafeta\begin{equation}
”
\end{equation}éseuvalor\begin{equation}
\left( futuro,\right)
\end{equation}masemtornodessevalorumriscodeperdalimitadoeumaprobabilidadedeganho\begin{equation}
$ilimitada)$
\end{equation}\begin{equation}
—
\end{equation}porissoaopçãotemdelhesermais\begin{equation}
$cara.$
\end{equation}mente\begin{equation}
$positivo.$
\end{equation}\begin{equation}
\left( Provavelmente,\right)
\end{equation}\begin{equation}
$dizemos:$
\end{equation}nãoé\begin{equation}
$certo;$
\end{equation}háum\begin{equation}
$risco.$
\end{equation}Por\begin{equation}
$menor:$
\end{equation}PrecificaçãoprobabilísticaEstaéaessênciadaprecificação\begin{equation}
$probabilística:$
\end{equation}todaagamadeesforçosparadescobrirpor queasopçõestêmospreçosqueelas\begin{equation}
\left( têm,\right)
\end{equation}quesebaseienasuposiçãodequeoprêmioderiscoédealgumaformadependentedadistribuiçãodeprobabilidadede\begin{equation}
$$$
\end{equation}no\begin{equation}
$futuro.$
\end{equation}DistribuiçãodeS\begin{equation}
Parte - se
\end{equation}assimparaverificarquetipodedistribuiçãodeprobabilidade\includegraphics[width=0.8\textwidth]{output/image_260png}EA\begin{equation}
“3
\end{equation}\begin{equation}
$Ovções:$
\end{equation}OperandoaVofatilidamosoudecréscimosde\begin{equation}
$preço)$
\end{equation}verificadosemumperíodofixode\begin{equation}
\left( dias,\right)
\end{equation}edesenharum\begin{equation}
\left( histograma,\right)
\end{equation}para\begin{equation}
$começar.$
\end{equation}\begin{equation}
Anota - se
\end{equation}opreçodoativoS\begin{equation}
\left( hoje,\right)
\end{equation}\begin{equation}
\left( amanhã,\right)
\end{equation}depois\begin{equation}
$etc.$
\end{equation}Daquia30\begin{equation}
\left( dias,\right)
\end{equation}por\begin{equation}
\left( exemplo,\right)
\end{equation}\begin{equation}
- se + toma
\end{equation}quantoo\begin{equation}
preço!
\end{equation}\begin{equation}
desviou - se
\end{equation}doqueera\begin{equation}
$hoje;$
\end{equation}a31\begin{equation}
\left( dias,\right)
\end{equation}oquanto\begin{equation}
desviou - se
\end{equation}doqueera\begin{equation}
\left( amanhã,\right)
\end{equation}assimpor\begin{equation}
\left( diante,\right)
\end{equation}e\begin{equation}
- se + tem
\end{equation}umespelhodasvariaçõesquepodemser\begin{equation}
$esperadas;$
\end{equation}dentrodeumperíodode30\begin{equation}
$dias.$
\end{equation}\begin{equation}
\left( Porém,\right)
\end{equation}éirracionalsuporqueodesviomédiodeumpreçopossaser\begin{equation}
\left( um,\right)
\end{equation}valorabsolutoparaqualquerníveldepreço\begin{equation}
$(uma$
\end{equation}açãoqueoscileemmédia\begin{equation}
$$10$
\end{equation}pordiaquandocotadaa\begin{equation}
$$100$
\end{equation}nãopodeoscilarosmesmos\begin{equation}
$$10$
\end{equation}quando\begin{equation}
$cotada:$
\end{equation}a\begin{equation}
$$1).$
\end{equation}Ospreçosnãopodemsetornarnegativosdevidoàs\begin{equation}
$oscilações.$
\end{equation}Osegundopassoquenospareceráadequadoénosdeparamoscompreços\begin{equation}
$negativos.$
\end{equation}Aúnicaformadecontornaroproblemadospreçosnegativosésupor\begin{equation}
\left( que,\right)
\end{equation}ospreços\begin{equation}
$$$
\end{equation}tenhamamesmaprobabilidadedevariarparacimaepara\begin{equation}
$baixo:$
\end{equation}porum\begin{equation}
$fator.$
\end{equation}\begin{equation}
\left( Digamos,\right)
\end{equation}\begin{equation}
$1,01:0$
\end{equation}preçoSteriaiguaisprobabilidadesdesubirparaí\begin{equation}
$1,01$
\end{equation}vezesseu\begin{equation}
\left( valor,\right)
\end{equation}oucairpor\begin{equation}
$1,01$
\end{equation}vezesseu\begin{equation}
$valor.$
\end{equation}Umaaltade\begin{equation}
$100%$
\end{equation}\begin{equation}
$(duas.$
\end{equation}vezesseu\begin{equation}
$valor)$
\end{equation}teriaamesmaprobabilidadedeumareduçãopor2\begin{equation}
$(uma$
\end{equation}\begin{equation}
baixa!
\end{equation}de\begin{equation}
$50%).$
\end{equation}Pormaiorquesejaum\begin{equation}
\left( fator,\right)
\end{equation}elejamaislevaráumpreço\begin{equation}
$asernegativo.,$
\end{equation}Aexpressãomatemáticamaissimplesparaumadistribuiçãoemquea\begin{equation}
$multi-;$
\end{equation}plicaçãoeadivisãoporumfatorsãoequiprováveisé adeumadistribuiçãodediferençasentrelogaritmosde\begin{equation}
$preços.$
\end{equation}Emumadistribuiçãodediferençasentrelogaritmosde\begin{equation}
\left( preços,\right)
\end{equation}umaalta\begin{equation}
\left( por,\right)
\end{equation}digamos\begin{equation}
\left( 1, \  15\right)
\end{equation}\begin{equation}
$(uma$
\end{equation}altade\begin{equation}
$15%)$
\end{equation}temamesmaprobabilidadeque\begin{equation}
$uma;$
\end{equation}baixapor\begin{equation}
\left( 1, \  15\right)
\end{equation}\begin{equation}
$(uma$
\end{equation}baixade\begin{equation}
$13,04%).$
\end{equation}Issoporqueadistânciaentreo\begin{equation}
$logaritmo.$
\end{equation}de100eode115éamesmaqueaquelaentreologaritmode100eode100\begin{equation}
$=:$
\end{equation}\begin{equation}
\left( 1, \  15\right)
\end{equation}\begin{equation}
$=$
\end{equation}\begin{equation}
$86,96:$
\end{equation}In\begin{equation}
115
\end{equation}\begin{equation}
$=$
\end{equation}In\begin{equation}
100
\end{equation}\begin{equation}
$=$
\end{equation}\begin{equation}
\left( 0, \  1397\right)
\end{equation}In\begin{equation}
100
\end{equation}\begin{equation}
$-$
\end{equation}In\begin{equation}
\left( 86, \  96\right)
\end{equation}\begin{equation}
$=$
\end{equation}\begin{equation}
\left( 0, \  1397\right)
\end{equation}Éestadiferença\begin{equation}
$(no$
\end{equation}\begin{equation}
\left( exemplo,\right)
\end{equation}\begin{equation}
$0,1397)$
\end{equation}quedesejamosagora\begin{equation}
$tabular.$
\end{equation}Elapassaaseranossavariável\begin{equation}
$aleatória.$
\end{equation}\begin{equation}
Pode - se
\end{equation}chegaraomesmonúmeropelologaritmodosretornos\begin{equation}
$(ou$
\end{equation}\begin{equation}
$fatores),$
\end{equation}queéaformamais\begin{equation}
$conhecida:$
\end{equation}admitirque\begin{equation}
- não
\end{equation}asvariações\begin{equation}
\left( absolutas,\right)
\end{equation}massimas\begin{equation}
\left( percentuais,\right)
\end{equation}seenquadrememalguma\begin{equation}
$distri-.$
\end{equation}buição\begin{equation}
$conhecida.$
\end{equation}Seríamoslevadosacrerqueumaaltade\begin{equation}
$1%$
\end{equation}eumabaixa\begin{equation}
$de:$
\end{equation}\begin{equation}
$1%$
\end{equation}têmamesmaprobabilidade\begin{equation}
$(a$
\end{equation}açãooscilaria\begin{equation}
$$810$
\end{equation}quandocotadaa\begin{equation}
$$100$
\end{equation}e\begin{equation}
$$0,10$
\end{equation}quandocotadaa\begin{equation}
$$1).$
\end{equation}\begin{equation}
\left( Contudo,\right)
\end{equation}teríamos dificuldadeemconceber\begin{equation}
\left( que,\right)
\end{equation}tamanhodebaixateriaamesmaprobabilidadequeumaaltade\begin{equation}
$120%.$
\end{equation}\begin{equation}
$Denovo;$
\end{equation}\begin{equation}
$::$
\end{equation}OperandoàVolatilidade29In\begin{equation}
$(115$
\end{equation}\begin{equation}
—
\end{equation}\begin{equation}
$100)$
\end{equation}\begin{equation}
$=$
\end{equation}\begin{equation}
\left( 0, \  1397\right)
\end{equation}In\begin{equation}
$(100$
\end{equation}\begin{equation}
—
\end{equation}\begin{equation}
$86,96)$
\end{equation}\begin{equation}
$=$
\end{equation}\begin{equation}
\left( 0, \  1397\right)
\end{equation}Parapassardeumadiferençalogarítmicaparaofator\begin{equation}
\left( equivalente,\right)
\end{equation}\begin{equation}
$eleva-$
\end{equation}seonúmeroe\begin{equation}
\left( 2, \  718281\right)
\end{equation}a\begin{equation}
$ela;$
\end{equation}parapassardofatorà\begin{equation}
\left( diferença,\right)
\end{equation}\begin{equation}
- se + tira
\end{equation}ologaritmoneperiano\begin{equation}
$dele.$
\end{equation}Paranúmerospequenos\begin{equation}
$(diferençasaté$
\end{equation}\begin{equation}
$0,05),$
\end{equation}0fatoréaproximadamenteadiferençalogarítmicamais\begin{equation}
\left( 1,\right)
\end{equation}eomovimentopercentualemS\begin{equation}
$(tanto$
\end{equation}paracimaquantopara\begin{equation}
$baixo)$
\end{equation}éaproximadamenteigualàprópria\begin{equation}
$diferença.$
\end{equation}Abaixoestáumatabelacomdiferenças\begin{equation}
\left( logarítmicas,\right)
\end{equation}fatorese\begin{equation}
$tama-$
\end{equation}nhodomovimentopercentualcorrespondenteem\begin{equation}
\left( 5,\right)
\end{equation}paraabaixaeparaa\begin{equation}
$alta:$
\end{equation}DiferençaFatorMovimentoMovimentologarítmicadebaixa\begin{equation}
$(%)$
\end{equation}dealta\begin{equation}
$(%)$
\end{equation}\begin{equation}
$0,01$
\end{equation}\begin{equation}
$1,010$
\end{equation}10LO\begin{equation}
$0,02$
\end{equation}\begin{equation}
$1,020$
\end{equation}2020\begin{equation}
$0,05$
\end{equation}\begin{equation}
$1,051$
\end{equation}4951\begin{equation}
\left( 0, \  10\right)
\end{equation}\begin{equation}
\left( 1, \  105\right)
\end{equation}\begin{equation}
\left( 9, \  5\right)
\end{equation}\begin{equation}
\left( 10, \  5\right)
\end{equation}\begin{equation}
\left( 0, \  20\right)
\end{equation}\begin{equation}
\left( 1, \  221\right)
\end{equation}\begin{equation}
\left( 18, \  1\right)
\end{equation}\begin{equation}
$22,),$
\end{equation}\begin{equation}
\left( 0, \  50\right)
\end{equation}\begin{equation}
\left( 1, \  649\right)
\end{equation}\begin{equation}
\left( 39, \  3\right)
\end{equation}\begin{equation}
\left( 64, \  9\right)
\end{equation}\begin{equation}
\left( 0, \  70\right)
\end{equation}2014\begin{equation}
\left( 50, \  3\right)
\end{equation}1014\begin{equation}
\left( 1, \  0\right)
\end{equation}\begin{equation}
\left( 2, \  718\right)
\end{equation}\begin{equation}
\left( 63, \  2\right)
\end{equation}\begin{equation}
\left( 171, \  8\right)
\end{equation}Distribuiçãolognormal\begin{equation}
\left( Então,\right)
\end{equation}apartirdeumasériedediferenças\begin{equation}
«
\end{equation}entrelogaritmosde\begin{equation}
$S(ou$
\end{equation}deumasériedelogaritmosderetornosde\begin{equation}
\left( \mathtt{\text{S}},\right)
\end{equation}oqueéamesma\begin{equation}
$coisa),$
\end{equation}chegaríamosàconclusãodequeosdadosdestasériesãonormalmente\begin{equation}
$distribuídos.$
\end{equation}Dizemosdesta\begin{equation}
\left( distribuição,\right)
\end{equation}\begin{equation}
\left( então,\right)
\end{equation}queéumadistribuiçãolognormalde\begin{equation}
$preços.$
\end{equation}Umavezidentificadaadistribuiçãodeprobabilidadesquemelhorseadaptaàssériesreaisde\begin{equation}
\left( preço,\right)
\end{equation}partimosparaidentificarnelaumamedidade\begin{equation}
\left( incerteza,\right)
\end{equation}ou\begin{equation}
$dispersão.$
\end{equation}Essamedidaéodesviopadrãoda\begin{equation}
\left( distribuição,\right)
\end{equation}enossoprimeirofocode\begin{equation}
$interesse.$
\end{equation}Seassumirmosqueosretornosde5nãosãoautocorrelacionados\begin{equation}
$(isto$
\end{equation}\begin{equation}
\left( é,\right)
\end{equation}queoretornodeumdianãotemcorrelaçãocomosdosdias\begin{equation}
$anteriores)$
\end{equation}equeavariânciadosretornosdiáriosé\begin{equation}
\left( constante,\right)
\end{equation}nossoexperimentopodeser\begin{equation}
$simplificado:$
\end{equation}emvezdesetomaremasdiferençasobservadasemperíodosiguaisde30\begin{equation}
\left( dias,\right)
\end{equation}\begin{equation}
podem - se
\end{equation}tomardiferenças\begin{equation}
$obser-$
\end{equation}vadasdeumdiaparao\begin{equation}
\left( outro,\right)
\end{equation}poisavariânciacalculadapara30diasdeverá\includegraphics[width=0.8\textwidth]{output/image_262png}30\begin{equation}
$Opções:$
\end{equation}OperandoàVolatilidadser30vezesavariânciacalculadaparaum\begin{equation}
$dia!.$
\end{equation}Isto\begin{equation}
\left( é,\right)
\end{equation}odesviopadrãodsdistribuiçãodediferençasparaumperíodode30diasdeveráser\begin{equation}
$/30$
\end{equation}\begin{equation}
$=5$
\end{equation}\begin{equation}
$as,$
\end{equation}vezesodesviopadrãodadistribuiçãodediferençasdeum\begin{equation}
$dia.$
\end{equation}\begin{equation}
$:$
\end{equation}Quantomaiorforestedesvio\begin{equation}
\left( padrão,\right)
\end{equation}maislargoéointervalode\begin{equation}
$preços:$
\end{equation}em quesepodeencontrar\begin{equation}
$S*com$
\end{equation}alguma\begin{equation}
$probabilidade.$
\end{equation}Porumapropriedaddadistribuição\begin{equation}
\left( normal,\right)
\end{equation}\begin{equation}
pode - se
\end{equation}dizerqueovalordeumavariável\begin{equation}
$aleatória.$
\end{equation}tem\begin{equation}
$68%$
\end{equation}dechancede\begin{equation}
- se + situar
\end{equation}dentrodointervalodeumdesviopadrãoabaixoaumdesviopadrãoacimadeseuvalormais\begin{equation}
$provável.$
\end{equation}\begin{equation}
\left( Assim,\right)
\end{equation}se\begin{equation}
$o.$
\end{equation}desviopadrãopara30diasdasdiferençaslogarítmicasdeumpreçoS\begin{equation}
$for;$
\end{equation}\begin{equation}
\left( 0, \  1397\right)
\end{equation}por\begin{equation}
\left( exemplo,\right)
\end{equation}éporqueháumaprobabilidadede\begin{equation}
$68%$
\end{equation}deopreço\begin{equation}
$S.$
\end{equation}\begin{equation}
\left( - se + situar,\right)
\end{equation}daquia30\begin{equation}
\left( dias,\right)
\end{equation}entrec157\begin{equation}
$=$
\end{equation}\begin{equation}
\left( 1, \  15\right)
\end{equation}vezesabaixoe\section{17}\begin{equation}
$=$
\end{equation}\begin{equation}
\left( 1, \  15\right)
\end{equation}vezestacimadeseuvalormaisprovável\begin{equation}
$(entre$
\end{equation}\begin{equation}
$100+$
\end{equation}\begin{equation}
\left( 1, \  15\right)
\end{equation}\begin{equation}
$=$
\end{equation}\begin{equation}
$86,96%$
\end{equation}e100x\begin{equation}
$1,15=$
\end{equation}\begin{equation}
$115%$
\end{equation}deseu\begin{equation}
\left( valor,\right)
\end{equation}ouentreumabaixade\begin{equation}
$13,04%$
\end{equation}eumaaltade\begin{equation}
$15%).$
\end{equation}VolatilidadeOdesviopadrãodeumadistribuiçãolognormaldepreçosé\begin{equation}
$definido.$
\end{equation}comosua\begin{equation}
\left( volatilidade,\right)
\end{equation}representadoporo\begin{equation}
\left( \sigma,\right)
\end{equation}eindicadoem\begin{equation}
\left( percentual,\right)
\end{equation}Umavolatilidadede\begin{equation}
$20%$
\end{equation}significaqueodesviopadrãodas\begin{equation}
$diferenças:$
\end{equation}logarítmicasé\begin{equation}
\left( 0, \  20.0\right)
\end{equation}UmpreçoSquetenhavolatilidadepara30diasiguala\begin{equation}
$20%$
\end{equation}\begin{equation}
$;$
\end{equation}éumpreçocujopadrãodeincerteza\begin{equation}
$é:$
\end{equation}\begin{equation}
“daqui
\end{equation}a30\begin{equation}
\left( dias,\right)
\end{equation}há\begin{equation}
$68%$
\end{equation}dechancede\begin{equation}
$.$
\end{equation}\begin{equation}
encontrá - lo
\end{equation}entre\begin{equation}
$e?”$
\end{equation}\begin{equation}
$=$
\end{equation}\begin{equation}
\left( 0, \  82\right)
\end{equation}\begin{equation}
$(ou$
\end{equation}\begin{equation}
$18%$
\end{equation}\begin{equation}
$abaixo)$
\end{equation}e\begin{equation}
$&?$
\end{equation}\begin{equation}
$=$
\end{equation}\begin{equation}
\left( 1, \  22\right)
\end{equation}\begin{equation}
$(ou$
\end{equation}\begin{equation}
$22%$
\end{equation}\begin{equation}
$acima)$
\end{equation}de\begin{equation}
$seu.$
\end{equation}valormais\begin{equation}
$provável”.$
\end{equation}Ê\begin{equation}
Note - se
\end{equation}\begin{equation}
\left( que,\right)
\end{equation}parapequenos\begin{equation}
\left( valores,\right)
\end{equation}avolatilidadeéidênticaao\begin{equation}
$movi-$
\end{equation}mentopercentualem\begin{equation}
$$.$
\end{equation}Umativoqueexibavolatilidadeparardiasiguala\begin{equation}
$2%$
\end{equation}\begin{equation}
$.$
\end{equation}éumativo\begin{equation}
\left( que,\right)
\end{equation}emtermos\begin{equation}
\left( práticos,\right)
\end{equation}possui\begin{equation}
$68%$
\end{equation}dechancedeser\begin{equation}
\left( encontrado,\right)
\end{equation}\begin{equation}
$'$
\end{equation}findosx\begin{equation}
\left( dias,\right)
\end{equation}entre\begin{equation}
$2%$
\end{equation}abaixoouacimadeseuvalormais\begin{equation}
$provável.$
\end{equation}Estavolatilidadeéamedidaderiscodeum\begin{equation}
$mercado.$
\end{equation}Éavariávelquevínhamosprocurandoparaquantificarosefeitosdoriscosobreospreçosdas\begin{equation}
$"$
\end{equation}NotemosporvímatoretornomensaldopreçoSapartirdadata\begin{equation}
£
\end{equation}Supondohaver21diasúteis\begin{equation}
$(e$
\end{equation}2observações\begin{equation}
$diárias) no$
\end{equation}intervalodeum\begin{equation}
\left( mês, \  »\right)
\end{equation}\begin{equation}
$=In(S,$
\end{equation}\begin{equation}
$,$
\end{equation}\begin{equation}
$+$
\end{equation}\begin{equation}
$S)=InS,,=5)X(S,,7S,)X...x$
\end{equation}Crea\begin{equation}
$S,.$
\end{equation}\begin{equation}
$2)].$
\end{equation}Usandoaspropriedadesdos\begin{equation}
\left( logaritmos,\right)
\end{equation}issoéigualaIn\begin{equation}
$(S,,,$
\end{equation}\begin{equation}
$+$
\end{equation}\begin{equation}
$S)+In(S,$
\end{equation}\begin{equation}
$-+In(S,$
\end{equation}\begin{equation}
$.$
\end{equation}\begin{equation}
$$,,$
\end{equation}\begin{equation}
$20):$
\end{equation}Senotarmosoretornodiáriodopreço\begin{equation}
$$na$
\end{equation}data\begin{equation}
£por
\end{equation}\begin{equation}
$1,,$
\end{equation}temosque4safe\begin{equation}
$+$
\end{equation}\begin{equation}
\left( Hg,\right)
\end{equation}Àvariânciadoretornomensal\begin{equation}
\left( 4,\right)
\end{equation}podeserescritacomovar\begin{equation}
$(rd=Var(m,$
\end{equation}ta\begin{equation}
$“+$
\end{equation}\begin{equation}
\left( ty,\right)
\end{equation}\begin{equation}
$n)-$
\end{equation}SUpondoqueosretornosdiáriosnãosão\begin{equation}
\left( autocorrelacionados,\right)
\end{equation}podemosescrever que\begin{equation}
$varín)$
\end{equation}\begin{equation}
$= =$
\end{equation}\begin{equation}
$var)$
\end{equation}\begin{equation}
$+$
\end{equation}\begin{equation}
$vartitd)$
\end{equation}\begin{equation}
$++$
\end{equation}\begin{equation}
$VAL)$
\end{equation}Esobahipótesedequeasvariânciasdiáriassão\begin{equation}
$=8$
\end{equation}\begin{equation}
$+$
\end{equation}Ss\begin{equation}
$+$
\end{equation}constanteseiguaisa\begin{equation}
$5”$
\end{equation}temos\begin{equation}
$var(r$
\end{equation}mensaléigualàraizda\begin{equation}
\left( variância,\right)
\end{equation}ou0421\begin{equation}
-2
\end{equation}\begin{equation}
$=$
\end{equation}\begin{equation}
$Se)$
\end{equation}miaa\begin{equation}
$+$
\end{equation}Haro\begin{equation}
$«-+$=21$
\end{equation}\begin{equation}
58.0
\end{equation}\begin{equation}
\left( Finalmente,\right)
\end{equation}odesviopadrãodoretornaei\begin{equation}
$Opções:$
\end{equation}OperandoaVolatilidade31Na\begin{equation}
$opções.$
\end{equation}Elaéumamedidadedispersãodepreçosfuturos\begin{equation}
\left( e,\right)
\end{equation}na\begin{equation}
\left( prática,\right)
\end{equation}medeoníveldeoscilaçãodeum\begin{equation}
$mercado:$
\end{equation}ummercadocalmopossuivolatilidade\begin{equation}
$baixa;$
\end{equation}ummercado\begin{equation}
\left( agitado,\right)
\end{equation}\begin{equation}
\left( nervoso,\right)
\end{equation}\begin{equation}
\left( incerto,\right)
\end{equation}possuivolatilidade\begin{equation}
$alta.$
\end{equation}\begin{equation}
$=$
\end{equation}Avolatilidadeésempretomadaemreferênciaaum\begin{equation}
\left( prazo,\right)
\end{equation}umintervalode\begin{equation}
$tempo.$
\end{equation}Podeserexpressaemqualquer\begin{equation}
\left( unidade,\right)
\end{equation}em\begin{equation}
\left( - dia + volatilidade,\right)
\end{equation}\begin{equation}
$volatilidade-$
\end{equation}\begin{equation}
\left( mês,\right)
\end{equation}\begin{equation}
\left( - ano + volatilidade,\right)
\end{equation}\begin{equation}
\left( - período + volatilidade,\right)
\end{equation}mastemdeentrarnasfórmulasnaunidadecompatívelcomaquesemedeo\begin{equation}
$tempo.$
\end{equation}\begin{equation}
Pode - se
\end{equation}chamarobinômioontdevolatilidadeefetivano\begin{equation}
$período.$
\end{equation}Não\begin{equation}
\left( há,\right)
\end{equation}nas\begin{equation}
$fórmu-$
\end{equation}lasde\begin{equation}
\left( precificação,\right)
\end{equation}nenhumlugaremqueconsteosemconstararaizquadradade\begin{equation}
$£.$
\end{equation}Issoquerdizerquenãohánenhumefeitopuro\begin{equation}
$e.simples$
\end{equation}davolatilidadesobreospreçosdas\begin{equation}
\left( opções,\right)
\end{equation}massimdavolatilidade\begin{equation}
$efetiva.$
\end{equation}Avolatilidadeécomoumataxade\begin{equation}
$juro:$
\end{equation}umamedidanominaldealgoqueacontececomopassardo\begin{equation}
$tempo.$
\end{equation}\begin{equation}
Passa - se
\end{equation}deumamedidaaoutradevolatilidade\begin{equation}
multiplicando - se
\end{equation}pelaraizquadradadarazãoentre\begin{equation}
$prazos.$
\end{equation}\begin{equation}
$De volatilidade-dia$
\end{equation}para\begin{equation}
$volatilidade-$
\end{equation}\begin{equation}
\left( ano,\right)
\end{equation}\begin{equation}
multiplica - se
\end{equation}por\begin{equation}
\frac{«}{255}
\end{equation}\begin{equation}
$=$
\end{equation}\begin{equation}
\left( 15, \  97\right)
\end{equation}\begin{equation}
$;$
\end{equation}de\begin{equation}
- mês + volatilidade
\end{equation}para\begin{equation}
$volatilidade-$
\end{equation}\begin{equation}
\left( dia,\right)
\end{equation}\begin{equation}
multiplica - se
\end{equation}por\begin{equation}
$;$
\end{equation}eassimpor\begin{equation}
$diante.$
\end{equation}Algumaconfusãoéfeitaquandoseutilizaacontagemde255ou250diasúteisoude360diascorridospor\begin{equation}
$ano.$
\end{equation}Quandosetrabalhacomaunidadedetempoemdias\begin{equation}
$(sempre$
\end{equation}\begin{equation}
$diasúteis)$
\end{equation}\begin{equation}
- e
\end{equation}conseguentementeseutilizaa\begin{equation}
$volatilidade-$
\end{equation}dianasfórmulas\begin{equation}
—
\end{equation}estaconfusãoé\begin{equation}
\left( inócua,\right)
\end{equation}poissejaacontagemqueforafetaráapenasaformaemqueagrandezaé\begin{equation}
$expressa.$
\end{equation}Bastaquesetenhao\begin{equation}
\left( cuidado,\right)
\end{equation}aoconversarcomoutra\begin{equation}
\left( pessoa,\right)
\end{equation}deuniformizaralinguagem\begin{equation}
$(atualmente$
\end{equation}\begin{equation}
—
\end{equation}finalde1995\begin{equation}
—
\end{equation}omercadobrasileiropareceestarconvergindoparaaunidadede255diaspor\begin{equation}
\left( ano,\right)
\end{equation}àsemelhançadomercado\begin{equation}
$internacional).$
\end{equation}Quandosetrabalhacomaunidadedetempoem\begin{equation}
\left( anos,\right)
\end{equation}odadodevolatilidadedeveserinformadocombasenonúmerodediasúteisdo\begin{equation}
$ano.$
\end{equation}Apesardequeonúmeroexatodediasúteisemumano\begin{equation}
\left( varie,\right)
\end{equation}Oerroemseconsiderar\begin{equation}
\left( 255,\right)
\end{equation}0queénaverdade252ou\begin{equation}
\left( 256,\right)
\end{equation}émuito\begin{equation}
$pequeno.$
\end{equation}Assim\begin{equation}
\left( sendo,\right)
\end{equation}neste\begin{equation}
\left( livro,\right)
\end{equation}expressamosasvolatilidadesanuaisnabasede255\begin{equation}
$dias.$
\end{equation}Édepossedadefiniçãodevolatilidadeedoconceitodaprecificaçãoprobabilísticaquesemodelamasforçasque dãopreçoàs\begin{equation}
$opções.$
\end{equation}\begin{equation}
\left( Assim,\right)
\end{equation}avolatilidade\begin{equation}
efetiva
\end{equation}\begin{equation}
\left( é,\right)
\end{equation}a\begin{equation}
\left( princípio,\right)
\end{equation}aúnicaresponsávelpelasdistânciasqueascurvas\begin{equation}
\left( 1, \  2000.0\right)
\end{equation}dográficoda\begin{equation}
$[fig$
\end{equation}\begin{equation}
$2.1]$
\end{equation}mantêmdalinhadevalor\begin{equation}
$intrínseco.$
\end{equation}Edeseesperar\begin{equation}
\left( que,\right)
\end{equation}quantomaioravolatilidade\begin{equation}
$(ou$
\end{equation}o\begin{equation}
$prazo),$
\end{equation}maisafastadaestejaalinhade\begin{equation}
$preço.$
\end{equation}Acurva1éumacurvadebaixavolatilidade\begin{equation}
$(e/ou$
\end{equation}prazo\begin{equation}
$curto),$
\end{equation}aopassoqueacurva3éumacurvadealtavolatilidade\begin{equation}
$(e/ou$
\end{equation}prazo\begin{equation}
$longo).$
\end{equation}SeummodelomatemáticoresultaemumaexpressãodeVemfunçãodeSeocompatívelcomesta\begin{equation}
\left( realidade,\right)
\end{equation}eleéummodelo\begin{equation}
$bem-sucedido.$
\end{equation}\includegraphics[width=0.8\textwidth]{output/image_264png}32\begin{equation}
$Ouções:$
\end{equation}OperandoaVolatilidadeÃRelaçõesnecessáriasPropriedadesdospreçosResumimosnosparágrafosanterioresoquesedevesaberparavisualizarospreçosdasopçõescomocompostosdeduas\begin{equation}
$parcelas:$
\end{equation}ovalorintrínseco\begin{equation}
$eo.$
\end{equation}prêmiode\begin{equation}
$risco.$
\end{equation}Todaa\begin{equation}
\left( discussão,\right)
\end{equation}quefoivoltadaparaocasodeumaopção\begin{equation}
“
\end{equation}\begin{equation}
\left( isoladamente,\right)
\end{equation}podesermelhorembasadaquandoserelacionamopçõescom\begin{equation}
$*$
\end{equation}mercados\begin{equation}
\left( futuros,\right)
\end{equation}e\begin{equation}
\left\lfloor{\frac{ca}{s}}\right\rfloor
\end{equation}computse\begin{equation}
$futuros.$
\end{equation}AsuposiçãodequeovaloresperadodeSemumadatafuturacoincidacomseuvalorfuturo\begin{equation}
\left( 7,\right)
\end{equation}por\begin{equation}
\left( exemplo,\right)
\end{equation}éumargumentodeequilíbrioparaaà\begin{equation}
$:$
\end{equation}precificaçãode\begin{equation}
$opções.$
\end{equation}Segundoessa\begin{equation}
\left( suposição,\right)
\end{equation}especuladoresearbitradores\begin{equation}
$;$
\end{equation}concordamcomovalormaisprováveldeSno\begin{equation}
$futuro.$
\end{equation}\begin{equation}
Pode - se
\end{equation}substituiraformulaçãodestahipótesepor\begin{equation}
\left( outra,\right)
\end{equation}quedizquemesmoqueosespeculadorestenhamexpectativasdiferentessobreovalordeSemdata\begin{equation}
\left( futura,\right)
\end{equation}ovalordefeovalorintrínsecodasopçõesserádadopelocarregamentodeSa\begin{equation}
$valor.$
\end{equation}\begin{equation}
$futuro.$
\end{equation}Essaformulaçãopartedopressupostoqueomercadoprefere\begin{equation}
$:$
\end{equation}semprequepossível\begin{equation}
$arbitrar.$
\end{equation}Isto\begin{equation}
\left( é,\right)
\end{equation}havendopossibilidadedeganhosem\begin{equation}
$|$
\end{equation}\begin{equation}
\left( risco,\right)
\end{equation}nadaimpediráquetodoovolumedomercadocorrapara\begin{equation}
\left( ela,\right)
\end{equation}anão\begin{equation}
$;$
\end{equation}serasuaprópria\begin{equation}
$extinção.$
\end{equation}Éumargumentodearbitragemparaaprecificaçãode\begin{equation}
$opções.$
\end{equation}ArbitragemUmaarbitrageméumaoperaçãodeganhosem\begin{equation}
$risco.$
\end{equation}Cabeperfeitamente\begin{equation}
$:$
\end{equation}nocontextodemercados\begin{equation}
$futuros.$
\end{equation}Seumativoénegociadoavistapor\begin{equation}
$$=$100$
\end{equation}\begin{equation}
$:$
\end{equation}eatermopor\begin{equation}
$?=$
\end{equation}\begin{equation}
$$110,$
\end{equation}eocustoefetivododinheiroentreadataatualea data\begin{equation}
$|$
\end{equation}devencimentodotermoéde\begin{equation}
$5%,$
\end{equation}umarbitradortoma\begin{equation}
$$100$
\end{equation}emprestadosa\begin{equation}
$5%$
\end{equation}ÀecomelescompraSa\begin{equation}
$vista;$
\end{equation}econtrataavendaatermopor\begin{equation}
$$110.$
\end{equation}Nadatadevencimentodo\begin{equation}
\left( termo,\right)
\end{equation}executaavendadoativopor\begin{equation}
$$110$
\end{equation}esaldaseu\begin{equation}
$emprés-$
\end{equation}timopagando\begin{equation}
$$105.$
\end{equation}Lucrou\begin{equation}
$$5$
\end{equation}sem\begin{equation}
$risco.$
\end{equation}Paraqueoarbitradornãopudesseganhardinheirosem\begin{equation}
\left( risco,\right)
\end{equation}acotaçãodotermodeveriaser\begin{equation}
$$105.$
\end{equation}Seomercadoconheceaoperaçãode\begin{equation}
\left( arbitragem,\right)
\end{equation}ese\begin{equation}
$'$
\end{equation}atodomomentoháarbitradorespotenciaisparacomprarevenderativose\begin{equation}
$'$
\end{equation}termosdemodoarealizarganhossem\begin{equation}
\left( risco,\right)
\end{equation}entãoéesperávelqueopreçodo\begin{equation}
$|$
\end{equation}termonuncadivirjamuitode\begin{equation}
$$105.$
\end{equation}\begin{equation}
\left( Aqui,\right)
\end{equation}nãoimportasenosreferimosanegóciosatermoouacontratos\begin{equation}
$futuros;$
\end{equation}ambostêmopreço\begin{equation}
$*$
\end{equation}Amodalidadedeliquidaçãonão\begin{equation}
\left( importa,\right)
\end{equation}podendoambasasespéciesdeoperaçãoseremgenericamentedenominadas\begin{equation}
$=;$
\end{equation}OperandoàVolatilidade33\begin{equation}
$“futuros.$
\end{equation}\begin{equation}
\left( Mas,\right)
\end{equation}paramaiorclarezadoqueserá\begin{equation}
\left( exposto,\right)
\end{equation}nosreferimos\begin{equation}
$inicial-$
\end{equation}menteanegóciosa\begin{equation}
\left( termo,\right)
\end{equation}comentregafísicano\begin{equation}
$vencimento.$
\end{equation}PricingporexpectativaXpricingporarbitragemAconsideraçãodeoperaçõesdearbitragemmelhorabastanteavalidade\begin{equation}
“de
\end{equation}modelosde\begin{equation}
$precificação.$
\end{equation}Na\begin{equation}
\left( verdade,\right)
\end{equation}nãohámuitadificuldadeemseprecificarumaopçãopor\begin{equation}
\left( expectativa,\right)
\end{equation}ou\begin{equation}
\left( seja,\right)
\end{equation}calculandoovalorpresentedaexpectativadevalorde\begin{equation}
$exercício:$
\end{equation}bastaprojetarváriosvaloresde\begin{equation}
\left( Sº,\right)
\end{equation}calcularpsvaloresde\begin{equation}
\left( V correspondentes,\right)
\end{equation}\begin{equation}
- los + ponderá
\end{equation}pelaprobabilidadede\begin{equation}
$ocorrên-$
\end{equation}iadecada\begin{equation}
$S*(que$
\end{equation}é\begin{equation}
\left( conhecida,\right)
\end{equation}umavezqueseconheceovalormaisprováveldeSteavolatilidadedo\begin{equation}
$ativo),$
\end{equation}etrazeramédiaponderadaavalor\begin{equation}
$presente.$
\end{equation}Oproblemacomestaabordageméprovarqueomercadoéneutro\begin{equation}
\left( aorisco,\right)
\end{equation}isto\begin{equation}
\left( é,\right)
\end{equation}queele\begin{equation}
$(incluindo$
\end{equation}os\begin{equation}
\left( especuladores,\right)
\end{equation}em\begin{equation}
$conjunto)$
\end{equation}realmenteprojetaSparaqfuturopelataxadejurolivrede\begin{equation}
$risco.$
\end{equation}OgrandepassodadoporBlackeScholesparaaprecificaçãodeopçõesfoiproporummodelonoqualasopçõessãoprecificadaspor\begin{equation}
$arbitragem:$
\end{equation}emtal\begin{equation}
\left( modelo,\right)
\end{equation}\begin{equation}
descobre - se
\end{equation}opreçoqueumaopçãodeveterparaquenãosejapossível\begin{equation}
“arbitrar
\end{equation}comela\begin{equation}
$(de$
\end{equation}formaparecidaàemquesedescobrequeopreçojustodofuturodoexemploanteriorseria\begin{equation}
$$105).$
\end{equation}Oargumentodearbitragemdispensaaprovadequeomercadorealéneutroao\begin{equation}
$risco.$
\end{equation}\begin{equation}
\left( e,\right)
\end{equation}\begin{equation}
\left( felizmente,\right)
\end{equation}osresultadosobtidosporambososmétodossão\begin{equation}
$idênticos.$
\end{equation}\begin{equation}
“
\end{equation}Damesmaformaqueanegociaçãoa\begin{equation}
\left( termo,\right)
\end{equation}oexercíciodeuma\begin{equation}
\left( opção,\right)
\end{equation}independentedaformacomovenhaaser\begin{equation}
\left( liquidado,\right)
\end{equation}equivaleauma\begin{equation}
$negoci-$
\end{equation}açãodeativoaqualépagaemduas\begin{equation}
$parcelas:$
\end{equation}opagamentodoprêmioda\begin{equation}
\left( opção,\right)
\end{equation}a\begin{equation}
\left( vista,\right)
\end{equation}\begin{equation}
\left( V,\right)
\end{equation}e o\begin{equation}
\left( pagamento,\right)
\end{equation}nadatade\begin{equation}
\left( exercício,\right)
\end{equation}dostrikeprice\begin{equation}
$K.$
\end{equation}Casoseja\begin{equation}
\left( exercida,\right)
\end{equation}uma\begin{equation}
\left( \frac{ca}{l},\right)
\end{equation}por\begin{equation}
\left( exemplo,\right)
\end{equation}representaacompradeativopagaemduasparcelasiguaisa\begin{equation}
\left( €,\right)
\end{equation}a\begin{equation}
\left( vista,\right)
\end{equation}e\begin{equation}
\left( X,\right)
\end{equation}nadatade\begin{equation}
$exercício.$
\end{equation}Ocandidatoaadquirirativoemumaoperaçãofinanciada\begin{equation}
$(com$
\end{equation}pagamentos\begin{equation}
$diferidos)$
\end{equation}podeescolherentre\begin{equation}
fazê - lo
\end{equation}pormeiodeumtermocompagamentodeZno\begin{equation}
\left( vencimento,\right)
\end{equation}ouatravésdeuma\begin{equation}
cal!
\end{equation}demesmo\begin{equation}
\left( prazo,\right)
\end{equation}compagamentodeCavistaeAno\begin{equation}
$vencimento.$
\end{equation}Acomprada\begin{equation}
\left\lfloor{\frac{cn}{possui}}\right\rfloor
\end{equation}umbenefícioa\begin{equation}
\left( mais,\right)
\end{equation}queéapossibilidadedeevitaronegóciocasoascondiçõessejam\begin{equation}
$desfavoráveis.$
\end{equation}\begin{equation}
\left( Porém,\right)
\end{equation}aescolha\begin{equation}
“pela
\end{equation}opçãojamaisteráumvalorinferiorqueaescolhapelo\begin{equation}
$termo.$
\end{equation}Equacionandoqualseriaopagamentomínimoaserefetuadoemuma\begin{equation}
“compra
\end{equation}financiadavia\begin{equation}
\left( opção,\right)
\end{equation}emfunçãodosvaloresdotermoedostrikeda\begin{equation}
\left( “opção,\right)
\end{equation}temos\begin{equation}
\left( que,\right)
\end{equation}paraaigualdadedeambososfluxosde\begin{equation}
\left( caixa,\right)
\end{equation}\begin{equation}
$C.$
\end{equation}\begin{equation}
$+$
\end{equation}\begin{equation}
\operatorname{VP}{\left(K \right)}
\end{equation}\begin{equation}
$=$
\end{equation}\begin{equation}
$VP),$
\end{equation}ouC\begin{equation}
$=$
\end{equation}\begin{equation}
$VP(F-$
\end{equation}\begin{equation}
$K).Isto$
\end{equation}\begin{equation}
\left( é,\right)
\end{equation}paraquenãohajaarbitragementrea\begin{equation}
\left\lfloor{\frac{cn}{e}}\right\rfloor
\end{equation}o\begin{equation}
\left( termo,\right)
\end{equation}Cterádesernomínimoigualaovalorpresentedadiferençaentre\begin{equation}
£
\end{equation}\includegraphics[width=0.8\textwidth]{output/image_266png}34\begin{equation}
$Opções:$
\end{equation}OperandoàVolntilidndeeXAssumindoqueopreçoPdofuturoé\begin{equation}
\left( justo,\right)
\end{equation}equeoativonãoproporciona\begin{equation}
\left( renda,\right)
\end{equation}isto\begin{equation}
\left( é,\right)
\end{equation}que7\begin{equation}
$=$
\end{equation}\begin{equation}
$VJ(S),$
\end{equation}afórmulaacima\begin{equation}
- se + torna
\end{equation}Cquim\begin{equation}
$=5—$
\end{equation}\begin{equation}
$VP(K).$
\end{equation}EstafórmulanãocorrespondeàrealidadeseKformaiordoque\begin{equation}
\left( Z, \  no\right)
\end{equation}casoda\begin{equation}
\left( call,\right)
\end{equation}ou\begin{equation}
\left( seja,\right)
\end{equation}quando\begin{equation}
$VP(P—$
\end{equation}\begin{equation}
$K)resultar$
\end{equation}umnúmerone\begin{equation}
$gativo.$
\end{equation}Dadoqueumaopçãonãopodevalermenosdoque\begin{equation}
\left( zero,\right)
\end{equation}poisonegóciodesfavorávelhãoé\begin{equation}
\left( obrigatório,\right)
\end{equation}ovalormínimoda\begin{equation}
$cal/$
\end{equation}deveserCn\begin{equation}
\left( maxiO,\right)
\end{equation}\begin{equation}
VAXF—
\end{equation}\begin{equation}
$K).$
\end{equation}Estafórmulajáé\begin{equation}
$conhecida:$
\end{equation}éaexpressãodovalorintrínsecodeuma\begin{equation}
$ca/$
\end{equation}deexercício\begin{equation}
$certo.$
\end{equation}Seoexercícioé\begin{equation}
\left( certo,\right)
\end{equation}nãoháriscoemrelaçãoa\begin{equation}
\left( ele,\right)
\end{equation}não\begin{equation}
\left( há,\right)
\end{equation}\begin{equation}
\left( pois,\right)
\end{equation}prêmiode\begin{equation}
\left( risco,\right)
\end{equation}etodoovalorda\begin{equation}
$ca//$
\end{equation}seresumeaseuprópriovalor\begin{equation}
$intrínseco.$
\end{equation}Um\begin{equation}
$ca//$
\end{equation}deexercíciocertoéumasituaçãopuramente\begin{equation}
\left( teórica,\right)
\end{equation}\begin{equation}
$didática.$
\end{equation}FuturosintéticoEvitandoagoraaabstraçãodeuma\begin{equation}
$ca//$
\end{equation}deexercício\begin{equation}
\left( certo,\right)
\end{equation}nãosepodegarantiroexercíciodeumaopção\begin{equation}
\left( individual,\right)
\end{equation}mas\begin{equation}
pode - se
\end{equation}garantiroexercício\begin{equation}
$;$
\end{equation}deumparde\begin{equation}
\left\lfloor{\frac{ca}{e}}\right\rfloor
\end{equation}\begin{equation}
$put.$
\end{equation}Entreumacalleumaputdemesmopreçode\begin{equation}
\left( exercício,\right)
\end{equation}umadelasseráforçosamente\begin{equation}
$exercida.$
\end{equation}Se\begin{equation}
$S*>$
\end{equation}Kacalldá\begin{equation}
$exercício:$
\end{equation}se\begin{equation}
$SE<K.$
\end{equation}\begin{equation}
$+$
\end{equation}Lfaputdáexercício\begin{equation}
$(se$
\end{equation}\begin{equation}
$S*=$
\end{equation}Xtantofazdizerquenenhumadasduasdeuexercício\begin{equation}
$|$
\end{equation}quantodizerquequalquerdasduasdeu\begin{equation}
$exercício).$
\end{equation}Seumoperadorcomprauma\begin{equation}
$ca//$
\end{equation}Cevendeumaput\begin{equation}
\left( P,\right)
\end{equation}ambasdestrike\begin{equation}
\left( K,\right)
\end{equation}eleoucompraráoativoporKexercendoa\begin{equation}
\left( call,\right)
\end{equation}outerádecompraroativoporXsendoexercidona\begin{equation}
$put.$
\end{equation}De\begin{equation}
$;$
\end{equation}qualquer\begin{equation}
\left( maneira,\right)
\end{equation}eletemumacompragarantidadeativopelopreçoAnadatade\begin{equation}
$vencimento.$
\end{equation}Estaposiçãodecalle\begin{equation}
\left( put,\right)
\end{equation}porgeraromesmoefeitono\begin{equation}
$.$
\end{equation}vencimentoqueumcontrato\begin{equation}
\left( futuro,\right)
\end{equation}échamadadefuturo\begin{equation}
$sintético.$
\end{equation}Seocompradordeumfuturosintético\begin{equation}
$(comprador$
\end{equation}decallevendedorde\begin{equation}
$;$
\end{equation}\begin{equation}
$pui)$
\end{equation}vendeumtermorealsobresua\begin{equation}
\left( posição,\right)
\end{equation}eleestarásimultaneamenteÉassegurandoavendadeativoporZeumfluxode\begin{equation}
$(P-$
\end{equation}KJno\begin{equation}
$vencimento.$
\end{equation}Osfluxos\begin{equation}
CP
\end{equation}nadataatuale\begin{equation}
$(PF$
\end{equation}\begin{equation}
—
\end{equation}\begin{equation}
$K)$
\end{equation}nadatadevencimento\begin{equation}
$são certos,$
\end{equation}estabelecidose\begin{equation}
$inalteráveis.$
\end{equation}Entre\begin{equation}
\left( eles,\right)
\end{equation}\begin{equation}
\left( então,\right)
\end{equation}deveexistirocarregamentodos\begin{equation}
$.$
\end{equation}jurosno\begin{equation}
$período.$
\end{equation}\begin{equation}
$Ou:$
\end{equation}\begin{equation}
$C-P=VP(E-K)$
\end{equation}\begin{equation}
$C-P=S-VP(K)$
\end{equation}Paridade\begin{equation}
- call + put
\end{equation}Apropriedade\begin{equation}
\left( acima,\right)
\end{equation}que uneospreçosdas\begin{equation}
\left\lfloor{\frac{ca}{s}}\right\rfloor
\end{equation}eputsdemesmostrikeÍaosvaloresdeSeKéconhecidacomoparidade\begin{equation}
\left( - call + pui,\right)
\end{equation}eéumimportante\begin{equation}
2.1
\end{equation}\begin{equation}
$|$
\end{equation}\begin{equation}
\left[ ão\right]
\end{equation}\begin{equation}
«Operando
\end{equation}nVolatifidade\begin{equation}
$,$
\end{equation}trumentodeanáliseede\begin{equation}
$operação.$
\end{equation}Aparidade\begin{equation}
- call + pyui
\end{equation}diz\begin{equation}
\left( que,\right)
\end{equation}\begin{equation}
$conhecen-$
\end{equation}\begin{equation}
do - se
\end{equation}opreçodeuma\begin{equation}
\left( call,\right)
\end{equation}\begin{equation}
\left( r,\right)
\end{equation}te\begin{equation}
\left( 5, \  0\right)
\end{equation}preçodaputseuparé\begin{equation}
\left( determinado,\right)
\end{equation}e\begin{equation}
$vice-$
\end{equation}HaYe\begin{equation}
$Isa.$
\end{equation}\begin{equation}
$.$
\end{equation}º\begin{equation}
$-$
\end{equation}Maisque\begin{equation}
\left( isso,\right)
\end{equation}aparidade\begin{equation}
- call + put
\end{equation}éaregradeformaçãodepreçodeum\begin{equation}
\left( turosintético,\right)
\end{equation}éaexpressãodaarbitragementrefuturossintéticos\begin{equation}
$ereais.$
\end{equation}ElaostraclaramentequeumfuturosintéticoéumfuturonoqualpartedopreçostáemdatafuturaeérepresentadaporXe aoutrapartedopreçoestáemnheiroa\begin{equation}
\left( vista,\right)
\end{equation}eérepresentadapor\begin{equation}
$C—-?.$
\end{equation}Seumfuturovale\begin{equation}
$F=$110,0$
\end{equation}futurotéticode\begin{equation}
$K=$100$
\end{equation}temdevalerovalorpresentede\begin{equation}
$$10$
\end{equation}\begin{equation}
$(em$
\end{equation}outras\begin{equation}
\left( palavras,\right)
\end{equation}calldeveser\begin{equation}
$VX$10)$
\end{equation}maiscaraquea\begin{equation}
$put).$
\end{equation}n\begin{equation}
$=$
\end{equation}Outraspropriedadesemergemdoconceitodefuturosintéticoedaparidade\begin{equation}
$put-call:$
\end{equation}pelasfórmulas\begin{equation}
\left( 2.1,\right)
\end{equation}adiferençadepreçoentreuma\begin{equation}
$cal/$
\end{equation}eimaputseupardependede\begin{equation}
\left( 5,\right)
\end{equation}Xre\begin{equation}
\left( £,\right)
\end{equation}masindependedavolatilidade\begin{equation}
\left( o,\right)
\end{equation}quecomojávimosé aresponsávelpelosprêmiosderiscodas\begin{equation}
$opções.$
\end{equation}Õfuturosintéticoéumaposiçãoemopçõesqueétotalmenteinvarianteà\begin{equation}
$volatilidade.$
\end{equation}\begin{equation}
$-$
\end{equation}Paraqueisso\begin{equation}
\left( aconteça,\right)
\end{equation}osprêmiosdeuma\begin{equation}
\left\lfloor{\frac{ca}{e}}\right\rfloor
\end{equation}umaputseupartêmdeexibir\begin{equation}
$*$
\end{equation}\begin{equation}
“mesmo
\end{equation}comportamentonuméricoemfaceaumavariaçãona\begin{equation}
$volatilidade.$
\end{equation}Se\begin{equation}
“
\end{equation}vimavariaçãodevolatilidadefazoprêmiodeuma\begin{equation}
$ca//$
\end{equation}subir\begin{equation}
$$2,$
\end{equation}temdefazer\begin{equation}
“opreço
\end{equation}daputdemesmostrikeeprazosubir\begin{equation}
$$2$
\end{equation}\begin{equation}
\left( também,\right)
\end{equation}paraqueadiferença\begin{equation}
$:C-P$
\end{equation}permaneça\begin{equation}
$constante.$
\end{equation}Delta\begin{equation}
eat - moneyness - the
\end{equation}Agorasuponhaqueopreço\begin{equation}
$$$
\end{equation}sobe\begin{equation}
$$1.$
\end{equation}Opreço7temdesubir\begin{equation}
$VASI).$
\end{equation}Comoofuturosintéticodeveacompanharo\begin{equation}
\left( futuro,\right)
\end{equation}eledevesubir\begin{equation}
$VARAS)$
\end{equation}também\begin{equation}
$—-$
\end{equation}emdata\begin{equation}
$futura.$
\end{equation}MasofuturosintéticoéformadoporumaquantiafuturafixaKeumaquantiavariávelavista\begin{equation}
$C-P.$
\end{equation}Emqualdasduasrecairáesteajustede\begin{equation}
$VH$1)?$
\end{equation}Naquantiaa\begin{equation}
\left( vista,\right)
\end{equation}\begin{equation}
$logicamente:$
\end{equation}asopçõesmudamdepreçoacompanhando\begin{equation}
$S.$
\end{equation}À\begin{equation}
cal!
\end{equation}deverá\begin{equation}
\left( subir,\right)
\end{equation}eaputdeverá\begin{equation}
\left( cair,\right)
\end{equation}\begin{equation}
\left( combinadas,\right)
\end{equation}VA\begin{equation}
$VHS)$
\end{equation}\begin{equation}
$=$
\end{equation}\begin{equation}
$$1.$
\end{equation}\begin{equation}
\left( Assim, \  pode - se\right)
\end{equation}enunciarqueadiferença\begin{equation}
$C-?$
\end{equation}devevariarexatamenteoquantovariar\begin{equation}
$S$.$
\end{equation}SeSsubir\begin{equation}
$$10,$
\end{equation}adiferença\begin{equation}
C - P
\end{equation}deverásubir910\begin{equation}
$também;$
\end{equation}seScair\begin{equation}
$$5,$
\end{equation}CPcairá\begin{equation}
$$5$
\end{equation}\begin{equation}
$também.$
\end{equation}Delta\begin{equation}
A
\end{equation}Paracada\begin{equation}
$$1$
\end{equation}devariaçãoparacimaem\begin{equation}
\left( 5, \  a\right)
\end{equation}\begin{equation}
$ca//$
\end{equation}devesubirumtanto\begin{equation}
\left( DC,\right)
\end{equation}eaputdevecairoutrotanto\begin{equation}
$2?$
\end{equation}demodoque\begin{equation}
$DC+$
\end{equation}\begin{equation}
$DP=$1.Sea$
\end{equation}\begin{equation}
\frac{cal}{subir}
\end{equation}\begin{equation}
\left( 80, \  70\right)
\end{equation}aputdevecair\begin{equation}
$$0,30;$
\end{equation}sea\begin{equation}
cal!
\end{equation}subir\begin{equation}
$$0,20,$
\end{equation}apirtdevecair\begin{equation}
$$0,80.$
\end{equation}Qualquerquesejao\begin{equation}
\left( caso,\right)
\end{equation}oimpactodaflutuaçãode\begin{equation}
$$1$
\end{equation}deSnopreçodecadaopçãosempre\includegraphics[width=0.8\textwidth]{output/image_268png}36\begin{equation}
$Ogvções:$
\end{equation}OperandoaVolatilidadeserámenordoque\begin{equation}
$$1.0$
\end{equation}prêmiodeumaopçãosemprevariaumafraçãodoquevariaOpreço\begin{equation}
5.0
\end{equation}Estafraçãoéchamada\begin{equation}
\left( \delta,\right)
\end{equation}erepresentadapor\begin{equation}
$A.$
\end{equation}Deltaéumnúmeroentrezeroeumpara\begin{equation}
\left( \left\lfloor{\frac{c_{4}}{s}}\right\rfloor,\right)
\end{equation}eentrezeroemenosumpara\begin{equation}
\left( puts,\right)
\end{equation}porqueasputsésãoinversamentecorrelacionadascomoa\begin{equation}
$vista.$
\end{equation}Adefiniçãodedelta\begin{equation}
$é:$
\end{equation}aquantidadedepreçoacrescidoaumaopçãoquandoseuobjetoSsofreumacréscimode\begin{equation}
$$1,$
\end{equation}\begin{equation}
$ou:$
\end{equation}\begin{equation}
—
\end{equation}dV\begin{equation}
$A=-—$
\end{equation}dsDapropriedadedofuturosintéticovistaemúltimo\begin{equation}
\left( lugar,\right)
\end{equation}podemosescreverqueodeltadeuma\begin{equation}
\left\lfloor{\frac{ca}{menos}}\right\rfloor
\end{equation}odeltada\begin{equation}
\frac{pw}{seu}
\end{equation}paréiguala\begin{equation}
\left( um,\right)
\end{equation}\begin{equation}
$ou:$
\end{equation}\begin{equation}
$Aç-$
\end{equation}Ap\begin{equation}
$=$
\end{equation}dda4ds\begin{equation}
$ds.$
\end{equation}\begin{equation}
\left( que,\right)
\end{equation}\begin{equation}
\left( aliás,\right)
\end{equation}é aderivaçãodafórmula\begin{equation}
2.1
\end{equation}emrelaçãoa\begin{equation}
5.0
\end{equation}Odeltadeumaopçãonãoéumnúmero\begin{equation}
$casual.$
\end{equation}Eletemavercomasperspectivasdeexercícioda\begin{equation}
\left( opção,\right)
\end{equation}epodesercalculadocomprecisãoporumafórmula\begin{equation}
$matemática.$
\end{equation}Emumacurva\begin{equation}
$V'x$
\end{equation}\begin{equation}
\left( 5,\right)
\end{equation}odeltapodeserinterpretadocomo\begin{equation}
$-$
\end{equation}amclinaçãotangencialdacurvanopontodadopeloS\begin{equation}
$atual.$
\end{equation}cSatual5Umaboainterpretaçãoparaodeltaéoníveldeagressividadedeuma\begin{equation}
\left( opção,\right)
\end{equation}ouoquantoelaécapazdeacompanharasvariaçõesdoa\begin{equation}
$vista.$
\end{equation}Deltas\begin{equation}
$*$
\end{equation}menoresindicamopçõesmenos\begin{equation}
$agressivas.$
\end{equation}Deltasmaioresindicamopções\begin{equation}
$-$
\end{equation}mais\begin{equation}
\left( agressivas,\right)
\end{equation}o\begin{equation}
$ações:$
\end{equation}OperndoaVolatilidade37AbaixoestãorepresentadosemumamesmacurvaVx5trêssituações\begin{equation}
$-$
\end{equation}\begin{equation}
$-$
\end{equation}\begin{equation}
$)$
\end{equation}\begin{equation}
$a.$
\end{equation}distintasdamesma\begin{equation}
$cz//,$
\end{equation}situaçõesemqueelaseencontrapara\begin{equation}
$S=51,82e$
\end{equation}\begin{equation}
$55:$
\end{equation}QS15253Cadaumadastrêsvariaçõesde\begin{equation}
$$$
\end{equation}sãode\begin{equation}
$$1.$
\end{equation}Quando\begin{equation}
$S =$
\end{equation}\begin{equation}
\left( 57,\right)
\end{equation}aopçãorepresentadasobevintecentavosparacada\begin{equation}
$$1$
\end{equation}queopreçoavista\begin{equation}
\left( suba,\right)
\end{equation}isto\begin{equation}
\left( é,\right)
\end{equation}seudeltaéiguala\begin{equation}
\left( 0, \  20.0\right)
\end{equation}\begin{equation}
Note - se
\end{equation}\begin{equation}
\left( que,\right)
\end{equation}quando\begin{equation}
$S'$
\end{equation}\begin{equation}
\left( sobe,\right)
\end{equation}aopçãosobede\begin{equation}
\left( preço,\right)
\end{equation}eoseudeltatambém\begin{equation}
\left( aumenta,\right)
\end{equation}isto\begin{equation}
\left( é,\right)
\end{equation}\begin{equation}
\left( graficamente,\right)
\end{equation}alinhadepreço\begin{equation}
- se + torna
\end{equation}cadavezmais\begin{equation}
$acentuada.$
\end{equation}Quandoopreço\begin{equation}
$$$
\end{equation}atinge\begin{equation}
\left( S_{2},\right)
\end{equation}aopçãojáéativaobastantepararespondercomumacréscimode\begin{equation}
$$0,50$
\end{equation}para cada\begin{equation}
$$1$
\end{equation}deacréscimoem\begin{equation}
$S(A=0,5).$
\end{equation}Quando\begin{equation}
$$=$
\end{equation}\begin{equation}
\left( S_{3}, \  0\right)
\end{equation}deltadaopção\begin{equation}
\left( é_{0}, \  90, \  e\right)
\end{equation}elavaria\begin{equation}
$$0,90$
\end{equation}para cada\begin{equation}
$$1$
\end{equation}devariaçãonoa\begin{equation}
$vista.$
\end{equation}Paravaloressuperioresde\begin{equation}
\left( 5,\right)
\end{equation}édeseesperarqueOdelta\begin{equation}
aproxime - se
\end{equation}cadavezmaisde\begin{equation}
\left( 1, \  0\right)
\end{equation}\begin{equation}
$(sem$
\end{equation}munca\begin{equation}
$alcançá-lo),$
\end{equation}equeacallrepresentada\begin{equation}
- se + torne
\end{equation}cadavezmaisagressivaatéestelimite\begin{equation}
$imaginário.$
\end{equation}\begin{equation}
\left( Analogamente,\right)
\end{equation}paravaloresdeSinferioresa\begin{equation}
\left( 57,\right)
\end{equation}\begin{equation}
espera - se
\end{equation}queOdeltada\begin{equation}
cal!
\end{equation}caiacadavezmaisatéolimite\begin{equation}
inalcançável
\end{equation}de\begin{equation}
\left( zero,\right)
\end{equation}equeaopção\begin{equation}
- se + torne
\end{equation}cadavezmenos\begin{equation}
$ativa.$
\end{equation}Seodeltadeumacal\begin{equation}
\left( sobe,\right)
\end{equation}odeltadaputseupartemde\begin{equation}
$cair.$
\end{equation}Paraa\begin{equation}
$ca//$
\end{equation}em\begin{equation}
$$3,$
\end{equation}umapuídemesmostriketeriadelta\begin{equation}
\left( 0, \  10\right)
\end{equation}eseriaumaopção\begin{equation}
$“morta”.$
\end{equation}Paraa\begin{equation}
\frac{col}{em}
\end{equation}\begin{equation}
\left( S_{1},\right)
\end{equation}aputteriadelta\begin{equation}
\left( 0, \  80\right)
\end{equation}eseriaumaopçãobem\begin{equation}
$ativa.$
\end{equation}Paraa\begin{equation}
\frac{ca}{í}
\end{equation}naposiçãointermediária\begin{equation}
$$2,$
\end{equation}aputteriadelta\begin{equation}
\left( 0, \  50\right)
\end{equation}eseriatãoelásticaaopreço\begin{equation}
$$$
\end{equation}quantoaprópria\begin{equation}
$cal/.$
\end{equation}o\begin{equation}
$;$
\end{equation}Astrêssituaçõespossíveisdeumaopçãoexemplificadasacimasãoidentificadascomoopção\begin{equation}
\left( - money - ofthe + out,\right)
\end{equation}opção\begin{equation}
\left( af - money - the,\right)
\end{equation}eopção\begin{equation}
$int-the-$
\end{equation}\begin{equation}
$Honey.$
\end{equation}Umaopçãoédita\begin{equation}
- money - ofthe + out
\end{equation}quandosuaprobabilidadedeexercícioemummundoneutroaoriscoé\begin{equation}
\left( baixa,\right)
\end{equation}eportantosuaelasticidadeàs\includegraphics[width=0.8\textwidth]{output/image_270png}38\begin{equation}
$Opções:$
\end{equation}OperandoaVolatiliciavariaçõesdeStambémébaixa\begin{equation}
$(como$
\end{equation}em\begin{equation}
$S7).$
\end{equation}Umaopçãoédita\begin{equation}
at - mone - the
\end{equation}quandoaschancesdeexercíciosãodeaproximadamenteumpara\begin{equation}
\left( um,\right)
\end{equation}\begin{equation}
«
\end{equation}portantosuaelasticidadeaSémédia\begin{equation}
$(como$
\end{equation}em\begin{equation}
$$2).$
\end{equation}Umaopçãoédita\begin{equation}
77 - the
\end{equation}moneyquandooseuexercícioémaisprováveldoqueoseu\begin{equation}
- exercício + não
\end{equation}\begin{equation}
$-$
\end{equation}quandoelapossuivalorintrínseco\begin{equation}
—
\end{equation}esuaelasticidadeaSépróximade\begin{equation}
1.0
\end{equation}Estclassificaçãopermiteidentificarrapidamentetodasasprincipaiscaracterísticasdeumaopçãoemdadasituação\begin{equation}
$(a$
\end{equation}mesmaopçãopode\begin{equation}
encontrar - se
\end{equation}emcadaumadastrês\begin{equation}
\left( fases,\right)
\end{equation}dependendodascircunstâncias\begin{equation}
—
\end{equation}dopreço\begin{equation}
$5).$
\end{equation}Emummercadodeopçõescomváriasséries\begin{equation}
\left( lançadas,\right)
\end{equation}algumasserãooutoutrasserão\begin{equation}
\left( at,\right)
\end{equation}outrasserão\begin{equation}
$171-Hhe-money.$
\end{equation}Oparâmetrodeltaajudaadiferenciarastrês\begin{equation}
$categorias:$
\end{equation}opções\begin{equation}
- ofthe + out
\end{equation}moneytêmdeltamenordoque\begin{equation}
$0,5;$
\end{equation}asopções\begin{equation}
at - money - the
\end{equation}têmdeltaemtorno\begin{equation}
\left( de_{0}, \  \frac{5}{e}\right)
\end{equation}asopções\begin{equation}
$ir-the-;money$
\end{equation}têmdeltamaiordoque\begin{equation}
\left( 0, \  5.0\right)
\end{equation}Oslimitesprecisodas\begin{equation}
\left( categorias,\right)
\end{equation}no\begin{equation}
\left( entanto,\right)
\end{equation}sópodemserestabelecidos\begin{equation}
$pessoalmente.$
\end{equation}Elesnãtêmnenhumautilidade\begin{equation}
\left( prática,\right)
\end{equation}anãoseruniformizaralinguagementreduapessoasqueestejamfalandosobre\begin{equation}
$opções.$
\end{equation}Umparâmetrocomumé\begin{equation}
$considerar:$
\end{equation}\begin{equation}
- money - of + out - thie
\end{equation}opçõesdedeltamenorque\begin{equation}
$0,35;$
\end{equation}\begin{equation}
at - money - the
\end{equation}opçõesdedeltaentre\begin{equation}
\left( 0, \  35\right)
\end{equation}e\begin{equation}
$0,75;$
\end{equation}e\begin{equation}
$in-the-money$
\end{equation}todoorestodaípara\begin{equation}
$cima.$
\end{equation}Algunsoperadoresutilizamasexpressões\begin{equation}
deep - of - out - rmoneye - the
\end{equation}\begin{equation}
$deep-in$
\end{equation}\begin{equation}
- money + the
\end{equation}paradesignaropçõesemsituações\begin{equation}
$extremas.$
\end{equation}Háquemconsidereumaopçãodedeita\begin{equation}
$(0,85$
\end{equation}\begin{equation}
$decp-in-the-smoney.$
\end{equation}Maisuma\begin{equation}
\left( vez,\right)
\end{equation}issosópode\begin{equation}
ser!
\end{equation}estabelecido\begin{equation}
$pessoalmente.$
\end{equation}Operadoresrigorososchamamdeopção\begin{equation}
$decp-in-:$
\end{equation}\begin{equation}
- money + the
\end{equation}aquelaquepossuideltamaiordoque\begin{equation}
\left( 0, \  99, \  e\right)
\end{equation}de\begin{equation}
$deep-out-ofthe-money:$
\end{equation}aquelaquepossuideltamenordoque\begin{equation}
\left( 0, \  5.0\right)
\end{equation}\begin{equation}
At - moneymess - the
\end{equation}Oparâmetrodeltaéapenasumdoscritériosde\begin{equation}
$af-he-moneyness.$
\end{equation}Adesignaçãoparafinsdejulgamentoelinguagemnãoémuito\begin{equation}
\left( crítica,\right)
\end{equation}mas\begin{equation}
$em.$
\end{equation}certas\begin{equation}
\left( ocasiões,\right)
\end{equation}\begin{equation}
pode - se
\end{equation}terdesaberexatamenteoquesequerdizerpor\begin{equation}
$uma.$
\end{equation}opção\begin{equation}
$«t-the-money.$
\end{equation}Umdoscasosmaissimpleséquandosecotamopçõe\begin{equation}
\left( custont - minde,\right)
\end{equation}\begin{equation}
\left( que,\right)
\end{equation}pornãoterempreçosdeexercício\begin{equation}
\left( especificados,\right)
\end{equation}sãodesignadascomo\begin{equation}
\left( - Hhe + at - mroney,\right)
\end{equation}\begin{equation}
at - money - the
\end{equation}\begin{equation}
$+$
\end{equation}\begin{equation}
$1%,$
\end{equation}\begin{equation}
$at-the-money-2%$
\end{equation}\begin{equation}
$etc.$
\end{equation}Hátrêscritériosdiferentesde\begin{equation}
- \frac{g}{the} - moneyness
\end{equation}\begin{equation}
$(at-the-moneyness$
\end{equation}\begin{equation}
\left( é,\right)
\end{equation}ao\begin{equation}
$pé:$
\end{equation}da\begin{equation}
\left( letra,\right)
\end{equation}aqualidadedeser\begin{equation}
$at-the-mmrorey).$
\end{equation}Os\begin{equation}
\left( americanos,\right)
\end{equation}habituados\begin{equation}
$com:$
\end{equation}opções\begin{equation}
\left( americanas,\right)
\end{equation}quepodemserexercidasaqualquer\begin{equation}
\left( momento,\right)
\end{equation}classificamí\begin{equation}
at - thesnoney
\end{equation}aopçãoparaaqual\begin{equation}
$K=$
\end{equation}\begin{equation}
$S.$
\end{equation}Aparcela\begin{equation}
$$—$
\end{equation}Kou\begin{equation}
\left( seja,\right)
\end{equation}ovalor\begin{equation}
$de.$
\end{equation}exercícioda\begin{equation}
\left( opção,\right)
\end{equation}éoquantoelaestádentrodo\begin{equation}
$dinheiro.$
\end{equation}Uma\begin{equation}
$ca//$
\end{equation}destrike\begin{equation}
$K=$100,$
\end{equation}quando\begin{equation}
$$=$110,$
\end{equation}está\begin{equation}
$$10$
\end{equation}dentrodo\begin{equation}
\left( dinheiro,\right)
\end{equation}independentede\begin{equation}
\left( prazo,\right)
\end{equation}sestaeOperandoaVolatilidade39adejuro\begin{equation}
$etc.$
\end{equation}Umaputde\begin{equation}
$K=$
\end{equation}\begin{equation}
$$100,$
\end{equation}quando\begin{equation}
$$'=$
\end{equation}\begin{equation}
$$90,$
\end{equation}tambémestá\begin{equation}
$$10$
\end{equation}dentrolo\begin{equation}
$dinheiro.$
\end{equation}Esteconceito\begin{equation}
$(X=$
\end{equation}\begin{equation}
$5)$
\end{equation}éoconceito\begin{equation}
aí - he - money
\end{equation}\begin{equation}
$spot.$
\end{equation}Para\begin{equation}
\left( nós,\right)
\end{equation}\begin{equation}
\left( brasileiros,\right)
\end{equation}habituadoscominflação\begin{equation}
\frac{e}{ou}
\end{equation}juros\begin{equation}
\left( altos,\right)
\end{equation}\begin{equation}
$dificil.$
\end{equation}nentehaveráformamaisútildedefinir\begin{equation}
- moneyness - \frac{a}{He}
\end{equation}doque\begin{equation}
$VP(K)=$
\end{equation}\begin{equation}
$S.$
\end{equation}Ou\begin{equation}
\left( a,\right)
\end{equation}umaopção\begin{equation}
at - money - tte
\end{equation}éaquelaparaaqualScarregadoajuroséigualpreçode\begin{equation}
$exercício.$
\end{equation}Nossasopçõessãovirtualouobrigatoriamente\begin{equation}
$européi-$
\end{equation}\begin{equation}
$;-seguindo-se$
\end{equation}daíqueoseuvalordeexercícioserásempredadopor5\begin{equation}
—
\end{equation}\begin{equation}
$ZP(K).$
\end{equation}Esteéoconceito\begin{equation}
$at-the-money-forward.$
\end{equation}Oterceirocritériode\begin{equation}
aí - moneyness - the
\end{equation}épelo\begin{equation}
$delta:$
\end{equation}umaopção\begin{equation}
al - money - the
\end{equation}éaquelaquetem4\begin{equation}
$=$
\end{equation}\begin{equation}
\left( 0, \  5.0\right)
\end{equation}Seriadeseesperarqueestecritériocoincidissecom\begin{equation}
“um
\end{equation}dosoutrosdois\begin{equation}
\left( anteriores,\right)
\end{equation}masnãoé\begin{equation}
$assim.$
\end{equation}Terdeltaiguala\begin{equation}
\left( 0, \  5\right)
\end{equation}nãoquerdizernemquesetemstrikeiguala\begin{equation}
\left( \mathtt{\text{S}},\right)
\end{equation}nemquesetemstrikeigualaScarregadoà\begin{equation}
$juros.$
\end{equation}\begin{equation}
\left( Contudo,\right)
\end{equation}todasascaracterísticasimportantesdocomportamentodeumaopçãotêmrelaçãocomofatodeseudeltaser\begin{equation}
\left( maior,\right)
\end{equation}menorouigual\begin{equation}
\left( 0, \  5.0\right)
\end{equation}\begin{equation}
\left( Portanto,\right)
\end{equation}neste\begin{equation}
\left( livro,\right)
\end{equation}quandonosreferirmosaumaopçãoser\begin{equation}
$aí, 7n$
\end{equation}ou\begin{equation}
$out-of-$
\end{equation}\begin{equation}
\left( - money + “sre,\right)
\end{equation}estaremosadotandoocritériodo\begin{equation}
$delta.$
\end{equation}\includegraphics[width=0.8\textwidth]{output/image_272png}\section{MODELO}\section{BLACK}\section{E}\section{SCHOLES}EquaçõesdiferenciaisAntesdeentrarnoqueéoBlackeScholespropriamente\begin{equation}
\left( dito,\right)
\end{equation}vamos\begin{equation}
“introduzir
\end{equation}oleitornosconceitosmatemáticosimportantesparaassimilar\begin{equation}
$-$
\end{equation}plenamenteomodeloesua\begin{equation}
$expressão.$
\end{equation}Ever\begin{equation}
$|$
\end{equation}Equaçãodiferencial\begin{equation}
“A
\end{equation}equaçãodeBlackeScholeséumaequação\begin{equation}
$diferencial.$
\end{equation}Umaequaçãoalgébricaéumarelaçãoentre\begin{equation}
$quantidades;$
\end{equation}umaequaçãodiferencialéumarelaçãoentrevariaçõesde\begin{equation}
$quantidades.$
\end{equation}Elanadadizsobreovalordeumacertagrandezaemfunçãodeoutras\begin{equation}
$grandezas;$
\end{equation}oúnicoesclarecimentoqueeladáéqualavariaçãopodemosesperarnagrandeza\begin{equation}
\left( x,\right)
\end{equation}emfacedasvariaçõesemoutras\begin{equation}
$grandezas.$
\end{equation}Umexemplomuitosimplesfaráaidéiabem\begin{equation}
$entendida.$
\end{equation}Denominemosxadistânciadeumobjetomóvelaumponto\begin{equation}
\left( fixo,\right)
\end{equation}e\begin{equation}
£o
\end{equation}tempocontadoapartirdeuminstante\begin{equation}
$qualquer.$
\end{equation}Notamosdrcomoavariação\begin{equation}
$(acréscimo$
\end{equation}ou\begin{equation}
$diminui-$
\end{equation}\begin{equation}
$ção)$
\end{equation}de\begin{equation}
\left( x,\right)
\end{equation}edicomoavariaçãode\begin{equation}
$£.$
\end{equation}\begin{equation}
\left( Então,\right)
\end{equation}senosdepararmoscomaequaçãodx\section{7L}\begin{equation}
3.1
\end{equation}A\begin{equation}
3.1
\end{equation}\begin{equation}
$(onde$
\end{equation}Zéumaconstantenumérica\begin{equation}
\left( qualquer,\right)
\end{equation}por\begin{equation}
\left( exemplo,\right)
\end{equation}\begin{equation}
$10),$
\end{equation}quesentidoestaequação\begin{equation}
$tem?$
\end{equation}Qualainformaçãoqueelanos\begin{equation}
$dá?$
\end{equation}Lendoaequaçãocom\begin{equation}
\left( naturalidade,\right)
\end{equation}podemos\begin{equation}
$pensar:$
\end{equation}avariaçãodexsobreavariaçãodeZéigual\begin{equation}
$àZ(no$
\end{equation}\begin{equation}
\left( exemplo,\right)
\end{equation}\begin{equation}
$10).Diante$
\end{equation}deumavariaçãoemtigualaumaunidade\begin{equation}
$(um$
\end{equation}\includegraphics[width=0.8\textwidth]{output/image_274png}42\begin{equation}
$Oyções:$
\end{equation}OperandoaVolatilidado\begin{equation}
\left( segundo,\right)
\end{equation}umahoraousejaláqualforamedidade\begin{equation}
$tempo),$
\end{equation}avariaçãodequesatisfazaequaçãoédeZunidades\begin{equation}
$(10$
\end{equation}\begin{equation}
\left( unidades,\right)
\end{equation}no\begin{equation}
$exemplo).$
\end{equation}Oqurepresenta\begin{equation}
$x?$
\end{equation}Àdistânciadesdeumcerto\begin{equation}
$ponto.$
\end{equation}Àequaçãodizqueestdistânciacrescede\begin{equation}
$Z(=$
\end{equation}\begin{equation}
$10)$
\end{equation}unidadesparacadaunidadedetempotranscor\begin{equation}
$rida.$
\end{equation}Emoutras\begin{equation}
\left( palavras,\right)
\end{equation}elanosinformaavelocidadedo\begin{equation}
$objeto.$
\end{equation}Nadaditosobreovalorde\begin{equation}
$x;$
\end{equation}\begin{equation}
sabe - se
\end{equation}apenasumacoisaaseu\begin{equation}
$respeito:$
\end{equation}eleaumentdedezunidadesacadaunidadede\begin{equation}
$tempo.$
\end{equation}CondiçõesiniciaisUmaequaçãodiferencialpodesermuitoútilporsi\begin{equation}
\left( só,\right)
\end{equation}paraselediretamenteasrelaçõesentrevariaçõeseenxergarosfenômenosassimrepr\begin{equation}
$sentados.$
\end{equation}\begin{equation}
\left( Mas,\right)
\end{equation}namaioriadasaplicações\begin{equation}
\left( práticas,\right)
\end{equation}asequaçõesdiferenciaiservemmesmoédeinstrumentoparasedeterminarovalordasgrandezas\begin{equation}
$(no$
\end{equation}nosso\begin{equation}
\left( exemplo,\right)
\end{equation}ovalorde\begin{equation}
$2).$
\end{equation}Comoaequaçãoemsinãomencionaeste\begin{equation}
$valor:$
\end{equation}anãoser\begin{equation}
\left( indiretamente,\right)
\end{equation}\begin{equation}
- se + tem
\end{equation}debuscarumcomplementoa\begin{equation}
\left( ela,\right)
\end{equation}quesão\begin{equation}
$as.$
\end{equation}condições\begin{equation}
$iniciais.$
\end{equation}Doexemplo\begin{equation}
\left( anterior,\right)
\end{equation}nadasepodeinferirdosvalores\begin{equation}
$que.$
\end{equation}xpossa\begin{equation}
\left( ter,\right)
\end{equation}amenosquese\begin{equation}
$saiba que,$
\end{equation}noinstantetidocomo\begin{equation}
\left( zero,\right)
\end{equation}ovalordexeradetanto\begin{equation}
—
\end{equation}\begin{equation}
\left( digamos,\right)
\end{equation}15\begin{equation}
$unidades.$
\end{equation}\begin{equation}
\left( Então,\right)
\end{equation}quando\begin{equation}
$£=$
\end{equation}\begin{equation}
\left( 1,\right)
\end{equation}sabemosquexvale15\begin{equation}
$+$
\end{equation}\begin{equation}
$10=25$
\end{equation}\begin{equation}
$unidades.$
\end{equation}\begin{equation}
Pode - se
\end{equation}saberovalorexatodexparaqualquervalorde\begin{equation}
£
\end{equation}Aequaçãox\begin{equation}
$=$
\end{equation}15notempozeroéacondiçãoinicialusadano\begin{equation}
$problema.$
\end{equation}Depossedestedadoque\begin{equation}
\left( faltava,\right)
\end{equation}oproblemaficacompletamente\begin{equation}
$coloca-$
\end{equation}\begin{equation}
$do:$
\end{equation}opardeequaçãodiferencialecondiçãoinicialpermitequesecalculeovalordexparaqualquervalorde\begin{equation}
$?$
\end{equation}\begin{equation}
$dado.$
\end{equation}Amaioriadasequaçõesdiferenciaisqueexpressamfenômenos\begin{equation}
$interes-:$
\end{equation}santes\begin{equation}
$(incluindo$
\end{equation}adeBlacke\begin{equation}
$Scholes)$
\end{equation}émuitomaiscomplexadoqueaque\begin{equation}
$se:$
\end{equation}influenciamasvariaçõesde\begin{equation}
$x;$
\end{equation}aequaçãoexprimecomoasvariaçõesdevidasacadafator\begin{equation}
$interagem;$
\end{equation}\begin{equation}
$b)$
\end{equation}existemvariaçõesdesegunda\begin{equation}
$ordem.$
\end{equation}Avariação\begin{equation}
$de;$
\end{equation}segundaordemdaposiçãoxéa\begin{equation}
\left( aceleração,\right)
\end{equation}isto\begin{equation}
\left( é,\right)
\end{equation}ataxaemquea\begin{equation}
$própria;$
\end{equation}velocidadedex\begin{equation}
$varia.$
\end{equation}SoluçãoanalíticaOquesefazdepossedeumaequaçãodiferencialeumacondição\begin{equation}
$inicial?$
\end{equation}Jáqueesteéopontodepartidaparasolucionaroproblemadeseencontrar0valordeumagrandezadadososvaloresdas\begin{equation}
\left( demais,\right)
\end{equation}\begin{equation}
procura - se
\end{equation}emprimeirolugardeterminarumaequaçãoalgébricaquerelacioneasgrandezasentresiEmnossoexemplocoma\begin{equation}
\left( velocidade,\right)
\end{equation}a\begin{equation}
$equação:$
\end{equation}\begin{equation}
$Exções:$
\end{equation}OperandoaVolatilidade43\begin{equation}
$x=x$
\end{equation}\begin{equation}
$+$
\end{equation}Zé\begin{equation}
$)$
\end{equation}\begin{equation}
\left( tou,\right)
\end{equation}comos\begin{equation}
\left( dados,\right)
\end{equation}x\begin{equation}
$=$
\end{equation}15\begin{equation}
$+$
\end{equation}\begin{equation}
$107)$
\end{equation}resolveriadeumavezportodaso\begin{equation}
$problema.$
\end{equation}Issosechamaresolveranaliticamenteumaequação\begin{equation}
$diferencial.$
\end{equation}Afórmulamaé asoluçãoanalíticadaequaçãodiferencial\begin{equation}
$(3.1).$
\end{equation}\begin{equation}
“Nem
\end{equation}sempreépossívelchegaraesta\begin{equation}
$fórmula.$
\end{equation}NocasodomodeloBlack\begin{equation}
\left( Scholes,\right)
\end{equation}ela\begin{equation}
\left( existe,\right)
\end{equation}e ébemmaisconhecidadoqueaequaçãodiferencialem\begin{equation}
“De
\end{equation}\begin{equation}
\left( antemão,\right)
\end{equation}deixamosclaroquemuitasopções\begin{equation}
- dependentnão + patt
\end{equation}admitemrecificaçãoviafórmulas\begin{equation}
$analíticas.$
\end{equation}MétodosnuméricosEoquesefazsenãoseconseguedeterminarumasolução\begin{equation}
$analítica?$
\end{equation}\begin{equation}
$Parte-$
\end{equation}\begin{equation}
»
\end{equation}paramétodos\begin{equation}
\left( numéricos,\right)
\end{equation}quesãopoucomaisdoquesimulaçõesfeitasem\begin{equation}
\left( computador,\right)
\end{equation}quepodemconsumirtempodeprocessamento\begin{equation}
\frac{e}{ou}
\end{equation}de\begin{equation}
$prepa-$
\end{equation}çãodo\begin{equation}
$modelo.$
\end{equation}Emúltimo\begin{equation}
\left( caso,\right)
\end{equation}\begin{equation}
pode - se
\end{equation}precificarqualqueropção\begin{equation}
$fazendo-$
\end{equation}2UMcomputadorsimular\begin{equation}
\left( centenas,\right)
\end{equation}milharesdecenáriospossíveisparaopreço4e\begin{equation}
\left( calcular,\right)
\end{equation}paracadaum\begin{equation}
\left( deles,\right)
\end{equation}quantovaleriaoexercícioda\begin{equation}
$opção;$
\end{equation}\begin{equation}
\left( depois,\right)
\end{equation}\begin{equation}
calcula - se
\end{equation}amédiadosvalorespresentesdosresultados\begin{equation}
$obtidos:$
\end{equation}esteseráopreçodaopção\begin{equation}
$(é$
\end{equation}asimulaçãotipoMonte\begin{equation}
$Carlo).$
\end{equation}\begin{equation}
$|$
\end{equation}Ométodo4Eclaroqueissopodesermuito\begin{equation}
\left( custoso,\right)
\end{equation}deondeseconcluiqueograndetrabalhodosespecialistasé\begin{equation}
$de:$
\end{equation}\begin{equation}
$a)$
\end{equation}descobrirumasolução\begin{equation}
$analítica;$
\end{equation}\begin{equation}
$b)$
\end{equation}senão\begin{equation}
$'*$
\end{equation}\begin{equation}
£onseguir
\end{equation}descobrirasolução\begin{equation}
\left( analítica,\right)
\end{equation}pelomenosdescobrirseelapodeacaboude\begin{equation}
$ver.$
\end{equation}Ascomplicaçõessão\begin{equation}
$duas:$
\end{equation}\begin{equation}
$a)$
\end{equation}existemoutrosfatoresqueíexistirou\begin{equation}
$não;$
\end{equation}\begin{equation}
$c)$
\end{equation}sesabeoudesconfiaqueelanão\begin{equation}
\left( existe,\right)
\end{equation}descobrirumaaproximaçãodoproblemaqueadmitasoluçãoanalítica\begin{equation}
$útil;ou$
\end{equation}\begin{equation}
$d)implementar$
\end{equation}ummétodoeconômicodesimulaçãonocomputador\begin{equation}
$(isso$
\end{equation}envolveestudarsimplificaçõesquepossamserfeitasparaagilizaroscálculossemperdade\begin{equation}
$precisao);$
\end{equation}\begin{equation}
$e)$
\end{equation}senãoháoutra\begin{equation}
\left( maneira,\right)
\end{equation}trabalharnabasenodesempenho\begin{equation}
$computacional:$
\end{equation}procurarcomputadoresmais\begin{equation}
\left( velozes,\right)
\end{equation}programasmais\begin{equation}
$sofis-$
\end{equation}\begin{equation}
\left( cados,\right)
\end{equation}linguagensmaisrápidas\begin{equation}
$etc.$
\end{equation}ConstruçãodemodelosTodosestespassossãoprecedidospelacriaçãodo\begin{equation}
$modelo.$
\end{equation}Nonosso\begin{equation}
\left( caso,\right)
\end{equation}cestá\begin{equation}
\left( estabelecido,\right)
\end{equation}éomodelo\begin{equation}
chamado
\end{equation}deBlacke\begin{equation}
$Scholes.$
\end{equation}Existemvários\begin{equation}
“Butros
\end{equation}modelosdeprecificaçãodeopções\begin{equation}
$v7r:lla$
\end{equation}\begin{equation}
\left( européias,\right)
\end{equation}emuitosoutros\includegraphics[width=0.8\textwidth]{output/image_276png}dd\begin{equation}
$Opções:$
\end{equation}OperandoaVolatilidadeaOperandoaVolatilidade4oqueservemàprecificaçãodeoutrostiposde\begin{equation}
$opção.$
\end{equation}Cadaumdelespartedepremissasqueficamàesperade\begin{equation}
\left( confirmação,\right)
\end{equation}isto\begin{equation}
\left( é,\right)
\end{equation}dequetestesrevelemqueomodeloprevêqualovalordemercadodeuma\begin{equation}
\left( opção,\right)
\end{equation}dadascondições\begin{equation}
$objetivas.$
\end{equation}Quandoseestabeleceum\begin{equation}
\left( modelo,\right)
\end{equation}\begin{equation}
fazem - se
\end{equation}suposiçõessobrea\begin{equation}
$realidade;$
\end{equation}Seomodeloassimcriadofor\begin{equation}
\left( adequado,\right)
\end{equation}produziráresultados\begin{equation}
\left( utilizáveis,\right)
\end{equation}Resultadosutilizáveissãomaisabrangentesdoqueresultadosexatosousejalácomquenomesequeirafocaraprecisãodosnúmerosaquesechegausandoseo\begin{equation}
$modelo:$
\end{equation}oBlackeScholeséparticularmenteimprecisoquandosetratadprecificaropçõesdeexercíciomuitopouco\begin{equation}
$provável;$
\end{equation}algumasdesuas\begin{equation}
$supo-$
\end{equation}siçõessãosimplificações\begin{equation}
$grosseiras;$
\end{equation}\begin{equation}
\left( rigorosamente,\right)
\end{equation}elesópodeserutilizadoparaopçõesvanrlla\begin{equation}
\left( européias,\right)
\end{equation}eoriginariamentesobre\begin{equation}
$ações.$
\end{equation}\begin{equation}
\left( Contudo,\right)
\end{equation}eleésimplesepermitequeseganhedinheiroquandosetratadeoperarasopçõesdemaiorliquidezemmercadoscomoo\begin{equation}
\left( brasileiro,\right)
\end{equation}Issoéumfatorde\begin{equation}
\left( sucesso,\right)
\end{equation}alinharetacominclinaçãoiguala\begin{equation}
$Z.$
\end{equation}Acondiçãoinicial\begin{equation}
$2(0)$
\end{equation}\begin{equation}
$=$
\end{equation}15completata\begin{equation}
\left( informação,\right)
\end{equation}especificandoondeestalinhaseria\begin{equation}
$assentada.$
\end{equation}Osgráficosdasequaçõesaquitratadassãográficosdefunçõesdeduasariáveis\begin{equation}
$(sendo$
\end{equation}\begin{equation}
\left( juros,\right)
\end{equation}volatilidadeestrike\begin{equation}
\left( constantes,\right)
\end{equation}opreçodeumaopçãounçãodeSede\begin{equation}
$2),$
\end{equation}queformamnãouma\begin{equation}
\left( linha,\right)
\end{equation}masumasuperfícienoespaço\begin{equation}
$imensional.$
\end{equation}Àequaçãodiferencialdefinequepropriedadesgeraispodemesta\begin{equation}
\left( superfície,\right)
\end{equation}eacondiçãoinicialforneceoperfilapartirdoqualaperfícieé\begin{equation}
$gerada.$
\end{equation}Oresultadopodeserumgráficocomo\begin{equation}
$este:$
\end{equation}PittaGráficosdesoluçãoÀrepresentaçãográficadasoluçãodeumaequaçãodiferencialdáumaboaidéiadofenômeno\begin{equation}
\left( representado,\right)
\end{equation}eémuitoútilquandosetratade\begin{equation}
\left( opções,\right)
\end{equation}paraasquaisascurvas\begin{equation}
$/x$
\end{equation}SeRxSsãode\begin{equation}
$interesse.$
\end{equation}Àrepresentaçãodasolução\begin{equation}
$x=$
\end{equation}15\begin{equation}
$+$
\end{equation}\begin{equation}
\frac{10}{da}
\end{equation}nossa\begin{equation}
equação - exemplo
\end{equation}éalgo\begin{equation}
$como:$
\end{equation}xEstegráficorepresentaumafunçãodeduasvariáveis\begin{equation}
$a(z,$
\end{equation}\begin{equation}
9.0
\end{equation}\begin{equation}
\left( Nele,\right)
\end{equation}andição\begin{equation}
\left( inicial,\right)
\end{equation}umafunção\begin{equation}
$x(z,$
\end{equation}\begin{equation}
$0)$
\end{equation}estádestacadaemlinhamais\begin{equation}
$forte.$
\end{equation}Estaeráarepresentaçãoutilizadanosexemplosdeste\begin{equation}
$capítulo.$
\end{equation}\begin{equation}
Note - se
\end{equation}queascondiçõesiniciaisdoproblemadeprecificaçãodeopçõesodadaspelacurva\begin{equation}
V x
\end{equation}\begin{equation}
$S*-$
\end{equation}\begin{equation}
\left( ou,\right)
\end{equation}utilizandoanotação\begin{equation}
\left( funcional,\right)
\end{equation}pelafunção\begin{equation}
$$%,$
\end{equation}\begin{equation}
$0)$
\end{equation}\begin{equation}
—
\end{equation}isto\begin{equation}
\left( é,\right)
\end{equation}pelaúnicarelaçãoentreprêmioeSdefinidacom\begin{equation}
$certeza.$
\end{equation}\begin{equation}
embremo - nos
\end{equation}dequetempo\begin{equation}
$/significa$
\end{equation}prazoenãotempo\begin{equation}
\left( decorrido,\right)
\end{equation}equeortantoécontadoapartirdovencimentopara\begin{equation}
$trás.$
\end{equation}\begin{equation}
$x(0)=15$
\end{equation}ModeloBlackeScholesModeloBlackeScholesBlackeScholesabordaramoproblemadopreçodasopçõesapartirdaóticade\begin{equation}
\left( que,\right)
\end{equation}sendoumaopçãoumderivativode\begin{equation}
\left( \mathtt{\text{S}},\right)
\end{equation}eladeveservirparaOropósitode\begin{equation}
\left( hedge,\right)
\end{equation}e\begin{equation}
\left( que,\right)
\end{equation}seháumaformadeimplementarsistematicamentetedgecom\begin{equation}
\left( opções,\right)
\end{equation}opreçoatribuídoaelasparaestafinalidadeéoseupreçoOnde\begin{equation}
x{\left(0 \right)}
\end{equation}\begin{equation}
$=$
\end{equation}15éacondição\begin{equation}
$inicial.$
\end{equation}Desprovidadacondição\begin{equation}
\left( inicial,\right)
\end{equation}umaequaçãodiferencialnosinformaapenasquetipodecurvaspodemosesperardo\begin{equation}
$gráfico.$
\end{equation}Emnosso\begin{equation}
\left( caso,\right)
\end{equation}nosinformariaquepodemosesperarcomográfico\includegraphics[width=0.8\textwidth]{output/image_278png}46\begin{equation}
$Opções:$
\end{equation}Operandoa\begin{equation}
\left( Volatilidad,\right)
\end{equation}\begin{equation}
$justo.$
\end{equation}Portrásdestacolocaçãoestáumargumentode\begin{equation}
$arbitragem:$
\end{equation}opreçojustdaopçãoéaquelequepermiteaentradaemumaposição\begin{equation}
$Aed'geada,$
\end{equation}isto\begin{equation}
\left( é,\right)
\end{equation}umposiçãosem\begin{equation}
\left( riscos,\right)
\end{equation}deformaqueoresultadoaolongodotemposejasempr\begin{equation}
$zero.$
\end{equation}Seopreçodemercadodeumaopçãodifere\begin{equation}
\left( deste,\right)
\end{equation}épossívelaumoperadortomarumaposiçãoJedgeada\begin{equation}
$-$
\end{equation}semriscos\begin{equation}
—e
\end{equation}comresultadodiferentdezero\begin{equation}
$(o$
\end{equation}operadortomaráaposiçãodeformaaproduziroresultadopositivpara\begin{equation}
$ele).$
\end{equation}Emummundoemqueosagentespreferemosganhossemriscoquaisquertiposde\begin{equation}
\left( aposta,\right)
\end{equation}umenormevolumedeoperaçõesdestetipresponderiaimediatamenteaqualquerdistorçãoentreopreçodemercadodaopçõeseseupreço\begin{equation}
\left( justo,\right)
\end{equation}eaconsequênciadissoéquetaldistorçãosequechegariaase\begin{equation}
$verificar.$
\end{equation}OinteressantedaabordagemdeBlackeScholesé\begin{equation}
\left( que,\right)
\end{equation}senãoexplictotalmenteovalordemercadodas\begin{equation}
\left( opções,\right)
\end{equation}pelomenosindicaumamaneiradganhardinheirocom\begin{equation}
\left( ele,\right)
\end{equation}definindoumaoperaçãodearbitragemquepoderserutilizadacasohajaumadiscrepânciaentreospreçosdemercadoeaquele\begin{equation}
$preconizados.$
\end{equation}PressupostosdomodeloO\begin{equation}
\left( modelo,\right)
\end{equation}criadonoiníciodosanos\begin{equation}
\left( 70,\right)
\end{equation}pressupõealgumascoisassobrarealidadedos\begin{equation}
$mercados:$
\end{equation}\begin{equation}
$1)$
\end{equation}osretornosdopreçoSapresentamdistribuiçãnormaldemédiatedesviopadrão\begin{equation}
volatilidade
\end{equation}igualao\begin{equation}
—
\end{equation}estadistribuiçãoresultadeumpasseio\begin{equation}
$aleatório;$
\end{equation}\begin{equation}
$2)$
\end{equation}avolatilidadede5é\begin{equation}
\left( constante,\right)
\end{equation}invariávno\begin{equation}
$tempo;$
\end{equation}\begin{equation}
$3)$
\end{equation}ataxadejurortambémé\begin{equation}
\left( constante,\right)
\end{equation}enãoháspreadentreataxdecaptaçãoedeaplicaçãododinheiro\begin{equation}
—
\end{equation}ambassãoiguaisa\begin{equation}
$7;$
\end{equation}\begin{equation}
$4)$
\end{equation}nãohácustode\begin{equation}
$corretagem.$
\end{equation}Umpressupostocomo\begin{equation}
1
\end{equation}énecessárioaqualquermodelodeprecificação\begin{equation}
$probabilística:$
\end{equation}definiradistribuiçãode\begin{equation}
5.0
\end{equation}OfatoruéacorreçãodeSparaseuvalormaisprovávelno\begin{equation}
\left( futuro,\right)
\end{equation}ouseu\begin{equation}
$carregamento.$
\end{equation}Emummundo\begin{equation}
$zisá-$
\end{equation}\begin{equation}
\left( neutral,\right)
\end{equation}comojá\begin{equation}
\left( vimos,\right)
\end{equation}oqueseesperadeumativoéqueeleacompanheocarregamentode\begin{equation}
$juros.$
\end{equation}Issonosfaráidentificarcom\begin{equation}
7.0
\end{equation}Opressuposto\begin{equation}
2
\end{equation}éútiparasimplificaroencaminhamentodaidéia deedgedinâmicoesimplificarotratamentomatemático\begin{equation}
$posterior.$
\end{equation}\begin{equation}
\left( Atualmente,\right)
\end{equation}hávariantesdoBlackeScholesquepermitemrelaxarestacondiçãosem\begin{equation}
$problemas.$
\end{equation}Ospressupostos\begin{equation}
3
\end{equation}e\begin{equation}
$(4$
\end{equation}sãonecessáriosparaaidéiadeedge\begin{equation}
\left( dinâmico,\right)
\end{equation}noqualháumaoperaçãoconstantedosmercadosdeSedetaxade\begin{equation}
\left( juro,\right)
\end{equation}\begin{equation}
comprando - se
\end{equation}e\begin{equation}
- se + vendendo
\end{equation}\begin{equation}
$$e$
\end{equation}\begin{equation}
dando - se
\end{equation}e\begin{equation}
- se + tomando
\end{equation}dinheiroinfinitasvezespor\begin{equation}
$dia.$
\end{equation}Ospressupostosdequeavolatilidadeeosjurossãoconstantes\begin{equation}
—
\end{equation}\begin{equation}
2
\end{equation}e\begin{equation}
3
\end{equation}\begin{equation}
—
\end{equation}têmum\begin{equation}
$papelimportante:$
\end{equation}elesreduzemonúmerodevariáveisdoproblemammOperandoaVolatilidade47\begin{equation}
$ois:$
\end{equation}Se\begin{equation}
$/$
\end{equation}Paraquaisquersituações\begin{equation}
\left( práticas,\right)
\end{equation}oprêmiodeumaopçãoéfunçãocinco\begin{equation}
$variáveis:$
\end{equation}\begin{equation}
\left( 5,\right)
\end{equation}K\begin{equation}
\left( 7,\right)
\end{equation}\begin{equation}
£e
\end{equation}\begin{equation}
\left( o,\right)
\end{equation}\begin{equation}
\left( mas,\right)
\end{equation}paraaprecificaçãodeumaopção\begin{equation}
\left( pecífica,\right)
\end{equation}\begin{equation}
- se + tendo
\end{equation}\begin{equation}
\left( K,\right)
\end{equation}7 e6constantessobtodasas\begin{equation}
\left( condições,\right)
\end{equation}oproblema\begin{equation}
duz - se
\end{equation}adeterminaracurva\begin{equation}
\left( VS,\right)
\end{equation}\begin{equation}
$/$
\end{equation}queatendeàequaçãodiferencialdeBlackholeseàscondições\begin{equation}
$iniciais.$
\end{equation}eltahedgeAessênciadomodeloéométodode4edge\begin{equation}
$desenvolvido.$
\end{equation}Aocontráriodenhedgeem\begin{equation}
\left( futuros,\right)
\end{equation}paraoqualbastatomar\begin{equation}
$uma"posição$
\end{equation}contráriaemntratosfuturosparaneutralizarumaposiçãoema\begin{equation}
\left( vista,\right)
\end{equation}o4edgeemopçõesopoderáserimplementadocomumaúnica\begin{equation}
\left( operação,\right)
\end{equation}porquearelaçãotreavariaçãodepreçodeSeavariaçãoconsequentedepreçodaopçãonão\begin{equation}
$onstante.$
\end{equation}Estarelaçãoédadapelodeltadeuma\begin{equation}
$opção.$
\end{equation}Éodeltaqueinformaaloacréscimo\begin{equation}
$(ou$
\end{equation}\begin{equation}
$decréscimo)$
\end{equation}nopreçodeumaopçãocausadopeloréscimode\begin{equation}
$$1,00$
\end{equation}nopreçodoa\begin{equation}
$vista.$
\end{equation}\begin{equation}
\left( Se,\right)
\end{equation}para4edgenrnosumaposiçãoema\begin{equation}
\left( vista,\right)
\end{equation}vendermos\begin{equation}
\frac{1}{A}
\end{equation}\begin{equation}
$cal!s$
\end{equation}sobredaunidadedea\begin{equation}
\left( vista,\right)
\end{equation}garantiremosumedge\begin{equation}
$instantâneo.$
\end{equation}Seuma\begin{equation}
\left\lfloor{\frac{ca}{tem}}\right\rfloor
\end{equation}\begin{equation}
\left( 0, \  40\right)
\end{equation}equeremos\begin{equation}
- la + usá
\end{equation}paraprotegerumacarteirade\begin{equation}
1.0
\end{equation}unidadesdevoa\begin{equation}
\left( vista,\right)
\end{equation}vendemos\begin{equation}
1.0
\end{equation}\begin{equation}
$+$
\end{equation}\begin{equation}
$0,40.=$
\end{equation}\begin{equation}
2.5
\end{equation}unidadesdesta\begin{equation}
$ca//.$
\end{equation}Caso0avistaba\begin{equation}
$$1,00,$
\end{equation}impactandoaposiçãoem\begin{equation}
\left( 81.0,\right)
\end{equation}asopçõessubirãocadauma\begin{equation}
$$0,40,$
\end{equation}pactandoaposiçãodemodocontrárionosmesmos\begin{equation}
\left( 0, \  40\right)
\end{equation}x\begin{equation}
2.5
\end{equation}\begin{equation}
$=$
\end{equation}\begin{equation}
$$1.000.$
\end{equation}\begin{equation}
\left( tão,\right)
\end{equation}dadoqueasopçõesadmitemum\begin{equation}
\left( \delta,\right)
\end{equation}ohedçeinstantâneoé\begin{equation}
$possível.$
\end{equation}\begin{equation}
\left( Contudo,\right)
\end{equation}ovalordeAnãoéconstanteaolongodavidada\begin{equation}
$opção.$
\end{equation}\begin{equation}
\left( rticularmente,\right)
\end{equation}elesofreinfluênciadoprópriovalorde\begin{equation}
$S:$
\end{equation}quantomaissobe\begin{equation}
«mais
\end{equation}sobeo\begin{equation}
$delta.$
\end{equation}Aconsequênciadissoé\begin{equation}
\left( que,\right)
\end{equation}tãologoo\begin{equation}
“edge
\end{equation}instantâneoscritoacimatenhasidoestabelecidoefuncionadoparaavariaçãode\begin{equation}
$$1,00,$
\end{equation}leterádeser\begin{equation}
\left( recalculado,\right)
\end{equation}poisodeltada\begin{equation}
$ca//,$
\end{equation}duranteoacréscimonopreçoterá\begin{equation}
$variado.$
\end{equation}Suponhamos\begin{equation}
\left( que,\right)
\end{equation}paraonovo\begin{equation}
$5$1,00$
\end{equation}maiscaroqueo\begin{equation}
\left( original,\right)
\end{equation}\begin{equation}
“delta
\end{equation}tenhapassadoa\begin{equation}
\left( 0, \  50.0\right)
\end{equation}Aquantidadede\begin{equation}
$ca//$
\end{equation}asermantidanaposiçãoleveráseragora1\begin{equation}
$=$
\end{equation}\begin{equation}
\left( 0, \  50\right)
\end{equation}x\begin{equation}
1.0
\end{equation}\begin{equation}
$=$
\end{equation}\begin{equation}
2.0
\end{equation}\begin{equation}
$unidades.$
\end{equation}\begin{equation}
\left( Menor,\right)
\end{equation}\begin{equation}
\left( portanto,\right)
\end{equation}quearantidadeoriginalde\begin{equation}
$2.500.$
\end{equation}Parareequilibraro\begin{equation}
\left( hedge,\right)
\end{equation}énecessáriorecomprarOunidadesde\begin{equation}
\left\lfloor{\frac{ca}{s}}\right\rfloor
\end{equation}\begin{equation}
$vendidas.$
\end{equation}CasoStivessecaídoemvezde\begin{equation}
\left( subir,\right)
\end{equation}ovalorAteria\begin{equation}
\left( diminuído,\right)
\end{equation}eoequilíbriodohedgeseriarestabelecidoporumavenda\begin{equation}
$icional.$
\end{equation}ParacadanovovalordeSnovascomprasouvendassãonecessáriasaposiçãode\begin{equation}
$opções.$
\end{equation}Estemétodode\begin{equation}
\left( hedge,\right)
\end{equation}\begin{equation}
\left( então,\right)
\end{equation}necessitaconstanteoperaçãodas\begin{equation}
$quantida-$
\end{equation}desde\begin{equation}
$opções.$
\end{equation}Em\begin{equation}
\left( teoria,\right)
\end{equation}otermoconstantesigníficacontínuono\begin{equation}
\left( tempo,\right)
\end{equation}ealizadoemintervalosinfinitesimaisde\begin{equation}
$tempo.$
\end{equation}Na\begin{equation}
\left( prática,\right)
\end{equation}constante\begin{equation}
$signi-$
\end{equation}\includegraphics[width=0.8\textwidth]{output/image_280png}48\begin{equation}
$Opções:$
\end{equation}OperandoaVolatilidad\begin{equation}
$ções:$
\end{equation}OuerandoaVolatilidadeficaserexecutadoalgumasvezespor\begin{equation}
$dia.$
\end{equation}Essetipodehedgeédesignadocomheslge\begin{equation}
$dinâmico.$
\end{equation}Asistematizaçãodohedgedinâmicodaposiçãodeltachamasedelta\begin{equation}
$hedging.$
\end{equation}Todooesquemaanteriorsófazsentidoapartirdequeopçõesadmitaum\begin{equation}
$delta.$
\end{equation}Supondoque\begin{equation}
\left( sim,\right)
\end{equation}oprópriovalordodeltaéumouipuídomodeldeprecificação\begin{equation}
$(caso$
\end{equation}\begin{equation}
\left( contrário,\right)
\end{equation}omodeloseria\begin{equation}
\left( desnecessário,\right)
\end{equation}poisapartirdovaloresdedelta\begin{equation}
pode - se
\end{equation}reconstituiropreçodeuma\begin{equation}
$opção).$
\end{equation}ariáveisdo\begin{equation}
$modelo:$
\end{equation}preço\begin{equation}
\left( 5,\right)
\end{equation}strike\begin{equation}
\left( K,\right)
\end{equation}prazo\begin{equation}
£
\end{equation}juros7evolatilidade\begin{equation}
$o.$
\end{equation}ostraremosaquiospassos\begin{equation}
\left( detalhados,\right)
\end{equation}facilitandoomaispossívelsua\begin{equation}
$mpreensão.$
\end{equation}\begin{equation}
\left( Apartirdeagora,\right)
\end{equation}substituiremosadefiniçãodeposição\begin{equation}
\left( hedgendn,\right)
\end{equation}orpura\begin{equation}
\left( comodidade,\right)
\end{equation}deumaposiçãocontendoNunidadesdescontra\begin{equation}
$N/$
\end{equation}unidadesde\begin{equation}
\left( opção,\right)
\end{equation}peladeumaposiçãocontendoAunidadesdeScontra1nidadede\begin{equation}
$opção.$
\end{equation}\begin{equation}
\left( Assim,\right)
\end{equation}\begin{equation}
$(AS$
\end{equation}\begin{equation}
—
\end{equation}\begin{equation}
$O)$
\end{equation}\begin{equation}
- se + torna
\end{equation}ovalorlíquidodestaposiçãoemdado\begin{equation}
$instante.$
\end{equation}l\begin{equation}
Note - se
\end{equation}tambémquenadaimpedequesemonteumaposiçãohedgeada\begin{equation}
lizando - se
\end{equation}\begin{equation}
$puts.$
\end{equation}Aexpressão\begin{equation}
$(AS$
\end{equation}\begin{equation}
—
\end{equation}\begin{equation}
$?)$
\end{equation}servetambémparaexprimirovalordetal\begin{equation}
$posição.$
\end{equation}Comoodeltadeuma\begin{equation}
\frac{p}{t}
\end{equation}éumnúmeró\begin{equation}
\left( negativo,\right)
\end{equation}teríamosdeumaposiçãohedgendacomputsconsistiriadeum\begin{equation}
\frac{s}{ortem}
\end{equation}ativosScumavendade\begin{equation}
$puts.$
\end{equation}Paratodosos\begin{equation}
\left( efeitos,\right)
\end{equation}ovalordeumaposiçãohedgeadacomopçõespode\begin{equation}
»r
\end{equation}tanto\begin{equation}
$(AS$
\end{equation}\begin{equation}
—
\end{equation}\begin{equation}
$€)$
\end{equation}quanto\begin{equation}
$(AS$
\end{equation}\begin{equation}
—
\end{equation}\begin{equation}
$7),$
\end{equation}\begin{equation}
\left( ou,\right)
\end{equation}\begin{equation}
\left( genericamente,\right)
\end{equation}\begin{equation}
$(AS$
\end{equation}\begin{equation}
—
\end{equation}\begin{equation}
$(7.$
\end{equation}\begin{equation}
$|$
\end{equation}Oesquemadedeltahedgingcondizcomconceitosquepodemser\begin{equation}
$expres-$
\end{equation}spelasseguintesequações\begin{equation}
$diferenciais:$
\end{equation}Àprimeiradelas\begin{equation}
$é:$
\end{equation}adDetalhesdodeltahedgeAntesdepartirparaa\begin{equation}
\left( matemática,\right)
\end{equation}cabeintroduzirdoisconceitos\begin{equation}
$adicio-$
\end{equation}\begin{equation}
\left( nais,\right)
\end{equation}aosquaissópoderemosretornarquandotratardeajustedohedgeemoperaçõesdevolatilidade\begin{equation}
$(Capítulo$
\end{equation}\begin{equation}
$6)$
\end{equation}edeopçõessintéticas\begin{equation}
$(Capítulo$
\end{equation}\begin{equation}
8.0
\end{equation}\begin{equation}
$primeiro:$
\end{equation}odeltahedgedeopçõesvendidasinduzaum\begin{equation}
\left( custo,\right)
\end{equation}advindodascomprasapreçosmaioresedasvendasapreçosmenores\begin{equation}
$(note-se$
\end{equation}quemanteracarteirahedgendaenvolveexatamenteessaperda\begin{equation}
$necessária).$
\end{equation}Umavez queumacarteira\begin{equation}
\left( hedgenda,\right)
\end{equation}por\begin{equation}
\left( definição,\right)
\end{equation}nãodálucronem\begin{equation}
\left( prejuízo,\right)
\end{equation}ocustododeltahedgedeveserabsorvidoporalgumelementoda\begin{equation}
$carteira.$
\end{equation}Esteelemento\begin{equation}
\left( éaopção,\right)
\end{equation}queemagrececomopassardo\begin{equation}
$tempo.$
\end{equation}Conformeseconduzacarteiradelta\begin{equation}
\left( hedgeada,\right)
\end{equation}ocustodo hedgevaisendocompensadopelareduçãodoprêmio\begin{equation}
$daopção.$
\end{equation}Éexatamenteporissoqueomodeloconsegueprecificarumaopçãoopreçodeumaopçãoéaquelequepagaodeltahedgecom\begin{equation}
$ela.$
\end{equation}Maistardevoltaremosa\begin{equation}
$isso.$
\end{equation}Osegundo\begin{equation}
$conceito:$
\end{equation}oriscode1quantidadedeopçãoétalque\begin{equation}
$coincide:$
\end{equation}comoriscodeÀquantidadesde\begin{equation}
\left( 5,\right)
\end{equation}\begin{equation}
podendo - se
\end{equation}assimconsideraracarteirdeltahedgendacomoumacarteiracompostadeduaspartes\begin{equation}
\left( simétricas,\right)
\end{equation}que5\begin{equation}
$anulam.$
\end{equation}\begin{equation}
\left( Porém,\right)
\end{equation}seseanulamquantoao\begin{equation}
\left( risco,\right)
\end{equation}nãoseanulamquantoaoefeito\begin{equation}
\left( caixa,\right)
\end{equation}pois1quantidadedeopçãocustarábemmenosdoqueÀquantidadesd\begin{equation}
$ativo.$
\end{equation}Parasolucionaresta\begin{equation}
\left( assimetria,\right)
\end{equation}omodeloBlackeScholesimplicitamenteconsideraqueodinheiroobtidoparacomprar5venhadeumafonted\begin{equation}
\left( financiamento,\right)
\end{equation}\begin{equation}
\left( ou,\right)
\end{equation}no\begin{equation}
\left( mínimo,\right)
\end{equation}tenhaumcustodeoportunidadeiguala\begin{equation}
7.0
\end{equation}Estéomotivopelo\begin{equation}
\left( qual,\right)
\end{equation}\begin{equation}
\left( enfim,\right)
\end{equation}opçõessãoinstrumentossensíveisa\begin{equation}
$juros.$
\end{equation}BlackeScholesestabeleceramqueopreçodeumaopçãodeveriaatendàscondiçõesgeradasporumJedge\begin{equation}
\left( dinâmico,\right)
\end{equation}ecombinaramestarealidadcomumapropriedadegeraldepreçosderivativosdeumSqueseenquadreeumpasseio\begin{equation}
$aleatório;$
\end{equation}daíresultouumaequaçãodiferencialqueserveatodosostiposdeopção\begin{equation}
$européia.$
\end{equation}\begin{equation}
\left( Então,\right)
\end{equation}resolveramestaequaçãoparaumacondiçãinicialespecífica\begin{equation}
—
\end{equation}dadapelafunção\begin{equation}
V x
\end{equation}Sínovencimento\begin{equation}
—
\end{equation}e\begin{equation}
$obtiveram:$
\end{equation}expressõesanalíticasparaopreçodeuma\begin{equation}
\left\lfloor{\frac{ca}{e}}\right\rfloor
\end{equation}deumapuíemfunçãodas\begin{equation}
$cinco:$
\end{equation}\begin{equation}
$AMAS-V)$
\end{equation}\section{Ta}\begin{equation}
$MAS=V)$
\end{equation}\begin{equation}
3.2
\end{equation}uetemaseguinte\begin{equation}
$interpretação:$
\end{equation}oúnicoefeitodopassardotemposobreumaosiçãodeltafedgeadaéacorreçãodeseuvalora\begin{equation}
$juros.$
\end{equation}\begin{equation}
\left( Formalmente,\right)
\end{equation}\begin{equation}
\frac{dx}{dt}
\end{equation}éacréscimooudecréscimodagrandezaxemumaunidadede\begin{equation}
$tempo;$
\end{equation}zxéaantidadededinheiroobtidacomaaplicaçãodastaxasdejuro7sobreoontantexemumaunidadedetempo\begin{equation}
$(não$
\end{equation}esquecerque7eestãoexpressosunidades\begin{equation}
$coerentes).$
\end{equation}Aequação\begin{equation}
3.2
\end{equation}nãoénecessariamenteverdadeiraatodoinstantedeumatuação\begin{equation}
\left( real,\right)
\end{equation}quandovariaçõesemSenodeltapodemocorrersemseremiptadasecorrigidaspelo\begin{equation}
\left( operador,\right)
\end{equation}maséválidaparaumperíodoinfinitesimalde\begin{equation}
$tempo.$
\end{equation}Emoutras\begin{equation}
\left( palavras,\right)
\end{equation}seexecutássemosumdeltaAcdgeetirássemosumafotografiainstantâneadarealidade\begin{equation}
\left( gerada,\right)
\end{equation}aspropriedadesnelaobservadasebedeceriamàequação\begin{equation}
$(3.2).$
\end{equation}Tratarcomintervalosinfinitesimaisdetemponplica\begin{equation}
pró - ratear
\end{equation}astaxasdejuroevolatilidadesparaestesperíodos\begin{equation}
\left( ínfimos,\right)
\end{equation}ouenãoépráticonem\begin{equation}
\left( necessário,\right)
\end{equation}umavezqueestamostrabalhandoapenasliteralmentecomasgrandezas7eoquepodemassimestaremquaisquerescalasdeunida\begin{equation}
$de.$
\end{equation}\begin{equation}
\left( Assim,\right)
\end{equation}supomosqueodinheironãoéremuneradoapenasno\begin{equation}
\left( Overnight,\right)
\end{equation}mascontinuamenteatodo\begin{equation}
\left( instante,\right)
\end{equation}oqueemnadanos\begin{equation}
$atrapalha.$
\end{equation}\begin{equation}
$-$
\end{equation}ÀsegundaequaçãoimportanteparasechegaràequaçãodiferencialdeBlackeScholeséaexpressãodo\begin{equation}
$óbvio:$
\end{equation}adequeAnãoéumnúmerocasualmas\includegraphics[width=0.8\textwidth]{output/image_282png}50\begin{equation}
$Opções:$
\end{equation}OperandoaVolatitidCuerandoaVolatilidade\begin{equation}
$s)$
\end{equation}idaparatodososcasos\begin{equation}
\left( 8,\right)
\end{equation}OeOvistosnoCapítulo1\begin{equation}
$(isso$
\end{equation}incluiasmbinaçõesdecallseputs\begin{equation}
\left( européias,\right)
\end{equation}conhecidascomumentecomo\begin{equation}
$estratégi-$
\end{equation}contidasnocaso\begin{equation}
$8).$
\end{equation}Oquediferiráumasoluçãodaoutraéexatamenteadiçãoinicialquealimentaráa\begin{equation}
\left( equação,\right)
\end{equation}\begin{equation}
\left( ou,\right)
\end{equation}em\begin{equation}
\left( suma,\right)
\end{equation}acurva\begin{equation}
V x
\end{equation}\begin{equation}
$S*decada$
\end{equation}so\begin{equation}
$particular.$
\end{equation}Asderivadasparciaisquefiguramnafórmula\begin{equation}
$(022$
\end{equation}\begin{equation}
$V/282,9V/9Se9$
\end{equation}\begin{equation}
$V/92)$
\end{equation}emsermelhoridentificadaspelasletras\begin{equation}
$gregas:$
\end{equation}iguala\begin{equation}
$4V/dS.$
\end{equation}Aequação\begin{equation}
$é:$
\end{equation}\begin{equation}
$HAs-V)$
\end{equation}ds\begin{equation}
$=0$
\end{equation}\begin{equation}
$(33$
\end{equation}que\begin{equation}
$significa:$
\end{equation}avariaçãodeumaposiçãodeltahedgendaemfacedevariaçõese\begin{equation}
$Sénula.$
\end{equation}\begin{equation}
\left( Ora,\right)
\end{equation}éporissoqueelaédelta\begin{equation}
$Aedgenda.$
\end{equation}Estamesmaequaçãopodeserescrita\begin{equation}
$como:$
\end{equation}42Anisdvo\begin{equation}
A
\end{equation}deltacomojáé\begin{equation}
\left( sabido,\right)
\end{equation}\begin{equation}
- derivada + éa
\end{equation}\begin{equation}
$(taxa$
\end{equation}de\begin{equation}
$variação)$
\end{equation}deVemrelaçãoASasO\begin{equation}
\left( \frac{9}{95},\right)
\end{equation}ouqualfraçãoderealvariaopreçodeumaopçãoparacada\begin{equation}
$R$1,00$
\end{equation}avacréscimonopreçoa\begin{equation}
$vista.$
\end{equation}Comoéoresultadodadivisãodeuma\begin{equation}
$A=$
\end{equation}asntidadededinheiropor\begin{equation}
\left( outra,\right)
\end{equation}deltaé\begin{equation}
$adimensional;$
\end{equation}éexpressocomoummerosemunidadede\begin{equation}
$medida.$
\end{equation}AúltimaequaçãoprovémdoLemade\begin{equation}
\left( Iô,\right)
\end{equation}queéumdadogeralparaamma\begin{equation}
1
\end{equation}qualquerderivativodeumpreçoSqueobedeçaàscondiçõesdeumpasse\begin{equation}
$:$
\end{equation}\begin{equation}
$aleatório.$
\end{equation}AexpressãomatemáticadoLemadeIô\begin{equation}
$é:$
\end{equation}\begin{equation}
7
\end{equation}gammaéadesignaçãode92\begin{equation}
\left( \frac{V}{982},\right)
\end{equation}ouaderivadasegundadeVemaçãoa\begin{equation}
\left( 5,\right)
\end{equation}ouaderivadadodeltaemrelaçãoa\begin{equation}
$S(V,$
\end{equation}AeTestãoparaSassimmoa\begin{equation}
\left( posição,\right)
\end{equation}avelocidadeea\begin{equation}
\left( aceleração,\right)
\end{equation}\begin{equation}
\left( respectivamente,\right)
\end{equation}deummóvelãoparao\begin{equation}
$tempo).$
\end{equation}Gammaéumamedidadequantovariaodeltadeumaçãoparacada\begin{equation}
$R$$
\end{equation}\begin{equation}
\left( 1, \  0\right)
\end{equation}queopreçoS\begin{equation}
$subir.$
\end{equation}Seumaopçãode\begin{equation}
$A=$
\end{equation}\begin{equation}
\left( 0, \  50\right)
\end{equation}tiverummmade\begin{equation}
$0,02,$
\end{equation}entãoseopreçodoavistasobe\begin{equation}
$$1,a$
\end{equation}opçãopassaater\begin{equation}
$A=$
\end{equation}\begin{equation}
\left( 0, \  52.0\right)
\end{equation}umamedidadeoquãorápidoaopçãopodemodificaroseu\begin{equation}
\left( estado,\right)
\end{equation}nsitandoentre\begin{equation}
- money - ofthe + ouf
\end{equation}e\begin{equation}
$in-the-money.$
\end{equation}\begin{equation}
\left( Graficamente,\right)
\end{equation}éoquantonacurvaturanográficoVxSé\begin{equation}
$acentuada:$
\end{equation}avoDV\section{DV}aveVas\section{Lad}\section{sa}2as\begin{equation}
$o'Sdt$
\end{equation}\begin{equation}
34
\end{equation}ondeVéqualquerfunçãode\begin{equation}
\left( \mathtt{\text{S}}, \  e\right)
\end{equation}Séumavariável\begin{equation}
\left( aleatória,\right)
\end{equation}dedesviopadrãoonaunidadede\begin{equation}
$tempo.$
\end{equation}OLemadeIôéencontradoemquestõesdefísicaestatísticaeprocessos\begin{equation}
$aleatórios.$
\end{equation}Seu\begin{equation}
- chave + papel
\end{equation}ésubstituirotermoestocástico\begin{equation}
aleatório
\end{equation}associadoaSpelotermo\begin{equation}
\left( - aleatório + não,\right)
\end{equation}\begin{equation}
\left( determinístico,\right)
\end{equation}1\begin{equation}
$/$
\end{equation}\begin{equation}
$2I(92$
\end{equation}\begin{equation}
$V/$
\end{equation}\begin{equation}
$9S92)x$
\end{equation}0252x\begin{equation}
$df.$
\end{equation}ÉdevidoaoLemadetôquesepodeencontrarumaexpressãodeterminísticaquedescrevaumadependênciadeumfenômeno\begin{equation}
$aleatório.$
\end{equation}EguaçãodeBlackeScholesUsandoastrêsequações\begin{equation}
\left( 3.2,\right)
\end{equation}\begin{equation}
3.3
\end{equation}e\begin{equation}
\left( 3.4,\right)
\end{equation}chegamosàequaçãodeBlacke\begin{equation}
$Scholes:$
\end{equation}LanDVoDVdv\begin{equation}
$+$
\end{equation}ll\section{gel}LL\begin{equation}
$y=0$
\end{equation}\begin{equation}
$2º“$
\end{equation}asPas\section{a}\begin{equation}
$(83$
\end{equation}queexprimeocomportamentodegualgueropçãovanília\begin{equation}
\left( européia,\right)
\end{equation}sendCammaaltoCammabaixopa\includegraphics[width=0.8\textwidth]{output/image_284png}52\begin{equation}
$Opções:$
\end{equation}Operandoa\begin{equation}
\left( Volatilidad,\right)
\end{equation}\begin{equation}
$es:$
\end{equation}OperandoàVolatilidade532ss\begin{equation}
$:$
\end{equation}Ocálculodethetanãolevaemcontaqueaopçãoestá\begin{equation}
“sentindo”
\end{equation}umvalorSmaiornodia\begin{equation}
\left( seguinte,\right)
\end{equation}e\begin{equation}
\left( que,\right)
\end{equation}paraa\begin{equation}
\left( opção,\right)
\end{equation}umSigualemtermosnominaisa\begin{equation}
\left( verdade,\right)
\end{equation}um\begin{equation}
$$mais$
\end{equation}baixoemtermos\begin{equation}
$reais.$
\end{equation}AparceladepreçoqueaopçãosuiporvisualizarumSmaisaltoequeéperdidacasoSpermaneçaomoéchamadade\begin{equation}
- juro + \theta
\end{equation}edada\begin{equation}
$por:$
\end{equation}Nosdoisgráficos\begin{equation}
\left( acima,\right)
\end{equation}temosgammas\begin{equation}
\left( positivos,\right)
\end{equation}oprimeiromaiorquo\begin{equation}
$segundo.$
\end{equation}Casoasconcavidadesfossemvoltadaspara\begin{equation}
\left( baixo,\right)
\end{equation}teríamogamumas\begin{equation}
$negativos.$
\end{equation}Gammaéoresultadodeumavariaçãodeumagrandezadimensional\begin{equation}
\delta
\end{equation}sobreumavariaçãoemreaisdopreço\begin{equation}
\left( 5,\right)
\end{equation}sendo\begin{equation}
$portantá;$
\end{equation}expressaem\begin{equation}
\left( \frac{1}{real},\right)
\end{equation}ou\begin{equation}
$real”.$
\end{equation}O\begin{equation}
$=7(SA-D$
\end{equation}juroTheta\begin{equation}
8
\end{equation}parceladepreçoqueaopção\begin{equation}
\left( perde,\right)
\end{equation}deumdiaparao\begin{equation}
\left( outro,\right)
\end{equation}devidoapenasduçãodeseuprêmioderiscoéchamadatheta\begin{equation}
$líquido.$
\end{equation}Thetalíquidoémpreumnúmeronegativo\begin{equation}
\left( —pois,\right)
\end{equation}mantidas\begin{equation}
\left( - condições + demais,\right)
\end{equation}oprêmiodecodeumaopçãoapenassereduzcomopassardotempo\begin{equation}
$—-$
\end{equation}emedeoquantoopção\begin{equation}
$“emagrece”.$
\end{equation}\begin{equation}
$-$
\end{equation}Ovalortotaldethetaéasomadosdois\begin{equation}
$efeitos.$
\end{equation}\begin{equation}
\left( Portanto,\right)
\end{equation}aexpressãoidadethetalíquido\begin{equation}
$é:$
\end{equation}thetamenos\begin{equation}
\left( - juro + \theta,\right)
\end{equation}\begin{equation}
$ou:$
\end{equation}0\begin{equation}
$;=0-7"(SA-—$
\end{equation}\begin{equation}
$1)$
\end{equation}digO\begin{equation}
\theta
\end{equation}éadesignaçãode\begin{equation}
$-dV/d+,$
\end{equation}ouaderivadadeVemrelaçãoa\begin{equation}
$tempo.$
\end{equation}Thetarepresentaaquantidadededinheiroqueoprêmiodeumaopçãadquire\begin{equation}
$(ou$
\end{equation}\begin{equation}
$perde)$
\end{equation}aoexpirarumaunidadedetempo\begin{equation}
$(um$
\end{equation}\begin{equation}
$dia)$
\end{equation}deseupraz\begin{equation}
$(a$
\end{equation}quantidadedeprêmioqueseráperdidaouadquiridadehojeparaamanhãSempre\begin{equation}
convenciona - se
\end{equation}9comomenos\begin{equation}
\left( \frac{dV}{dt},\right)
\end{equation}porque\begin{equation}
£
\end{equation}éoriginariamentcontadodetráspara\begin{equation}
$frente:$
\end{equation}\begin{equation}
\frac{dV}{dt}
\end{equation}éavariaçãodeprêmiodaopçãoaosacrescentarumdiaemseu\begin{equation}
\left( prazo,\right)
\end{equation}aopassoquemenosd\begin{equation}
\frac{V}{dté}
\end{equation}avariaçãodprêmiodaopçãoaosediminuirumdiadeseu\begin{equation}
$prazo.$
\end{equation}\begin{equation}
Note - se
\end{equation}\begin{equation}
\left( que,\right)
\end{equation}porserumaderivada\begin{equation}
\left( parcial,\right)
\end{equation}Oé aexpressãodoefeitodtemposobreopreçodasopções\begin{equation}
mantendo - se
\end{equation}osdemaisfatores\begin{equation}
$inalterados.$
\end{equation}Odemais\begin{equation}
\left( fatores,\right)
\end{equation}inclusive\begin{equation}
—
\end{equation}eprincipalmente\begin{equation}
—
\end{equation}\begin{equation}
5.0
\end{equation}Quer\begin{equation}
\left( dizer,\right)
\end{equation}\begin{equation}
- se + supondo
\end{equation}quSpermaneçanominalmente\begin{equation}
\left( igual,\right)
\end{equation}qualseráoacréscimooudecréscimodpreçorefletidona\begin{equation}
$opção.$
\end{equation}SubstituindoestanovanotaçãonafórmuladeBlacke\begin{equation}
\left( Scholes,\right)
\end{equation}ficamo\begin{equation}
$com:$
\end{equation}SubstituindoafórmulaacimanaequaçãodeBlacke\begin{equation}
\left( Scholes,\right)
\end{equation}chegamosàmamaissimpleseesclarecedoradesta\begin{equation}
\left( equação,\right)
\end{equation}que\begin{equation}
$é:$
\end{equation}1oqa2ºS\begin{equation}
$V=0$
\end{equation}AequaçãodeBlacke\begin{equation}
\left( Scholes,\right)
\end{equation}nessa\begin{equation}
\left( forma,\right)
\end{equation}podeserlidacomo\begin{equation}
“quanto
\end{equation}taiorO\begin{equation}
\left( \Gamma,\right)
\end{equation}maiorotheta\begin{equation}
$líquido",$
\end{equation}ou\begin{equation}
$"quanto$
\end{equation}maioro\begin{equation}
\left( \Gamma,\right)
\end{equation}maiorO\begin{equation}
$emagrecimento”.$
\end{equation}\begin{equation}
$0"$
\end{equation}\begin{equation}
$+1SA+0-1V$
\end{equation}\begin{equation}
$=$
\end{equation}0queémaisconhecidanaliteraturasobreomodelocomoarelaçãoentredeltgammae\begin{equation}
$theta.$
\end{equation}ropriedadesdaequaçãodeBlackeSholesAequaçãodeBlackeScholestemsemelhançascomduasequaçõesássicasda\begin{equation}
$física:$
\end{equation}aequaçãodocalorThetaliquido\begin{equation}
$(O$
\end{equation}\begin{equation}
$líquido)$
\end{equation}Nopróximo\begin{equation}
\left( capítulo,\right)
\end{equation}serátratadomaisdetalhadamenteumconceitoquéoportunointroduzir\begin{equation}
$aqui:$
\end{equation}odethetalíquidoou\begin{equation}
$emagrecimento.$
\end{equation}Aderivad9VEdV\begin{equation}
$=0$
\end{equation}parcial\begin{equation}
\left( \theta,\right)
\end{equation}conformevista\begin{equation}
\left( anteriormente,\right)
\end{equation}medeoquantooprêmiodeumdt\begin{equation}
$dx?$
\end{equation}opçãovariacomoreduzirdeumdiaemseu\begin{equation}
\left( prazo,\right)
\end{equation}permanecendoosdemaºfatores\begin{equation}
$inalterados.$
\end{equation}Porvários\begin{equation}
\left( motivos,\right)
\end{equation}émaisadequadodizer\begin{equation}
\left( que,\right)
\end{equation}seestamoaequaçãodotransporteprecificandoaopçãosegundoasregrasdeummundo\begin{equation}
\left( - neutral + zisk,\right)
\end{equation}entãodevavovserincluídonocálculodessavariaçãoofatode5terumvaloresperad\section{E}r\begin{equation}
$3”$
\end{equation}0xsuperiornodia\begin{equation}
$seguinte.$
\end{equation}\includegraphics[width=0.8\textwidth]{output/image_286png}54\begin{equation}
$Ovções:$
\end{equation}OverandoaVolatilicnoj\begin{equation}
$-$
\end{equation}OperandoaVolutitidadeasAequaçãodo\begin{equation}
\left( calor,\right)
\end{equation}comorepresentada\begin{equation}
\left( aqui,\right)
\end{equation}modelaadifusãodecalosobreumcorpounidimensional\begin{equation}
$(uma$
\end{equation}barrade\begin{equation}
\left( ferro,\right)
\end{equation}por\begin{equation}
$exemplo).$
\end{equation}Elrelacionaatemperaturaemqualquerpontoda\begin{equation}
\left( barra,\right)
\end{equation}aqualquertempcom\begin{equation}
- a + x
\end{equation}posiçãodopontona\begin{equation}
\left( barra,\right)
\end{equation}e\begin{equation}
$!-$
\end{equation}otempotranscorridodesdeurinstantezeronoqualdefinimosacondiçãoinicialdoproblema\begin{equation}
$(qual$
\end{equation}éconfiguraçãoinicialdetemperaturaaolongoda\begin{equation}
$barra).$
\end{equation}Aparcela\begin{equation}
$9V/d/($
\end{equation}nosso\begin{equation}
$theta)$
\end{equation}éoquantoatemperaturaemdadopontoaumentaoudiminuquandootempo\begin{equation}
$avança;$
\end{equation}aparcela92\begin{equation}
$V/9x2$
\end{equation}\begin{equation}
$(o$
\end{equation}nosso\begin{equation}
$gamma)$
\end{equation}éoquantodistribuiçãodetemperaturaéacentuadaemumsegmentoda\begin{equation}
$barra.$
\end{equation}Sacendemosummaçaricosobreumúnico\begin{equation}
\left( ponto,\right)
\end{equation}criamosumasituaçãodalto\begin{equation}
$gamma;$
\end{equation}\begin{equation}
\left( se,\right)
\end{equation}ao\begin{equation}
\left( contrário,\right)
\end{equation}aquecemosiguaimentetodaa\begin{equation}
\left( barra,\right)
\end{equation}criamoumasituaçãodebaixo\begin{equation}
$gamma.$
\end{equation}Aequaçãodocalorpodeserresumid\begin{equation}
$como:$
\end{equation}quantomaioro\begin{equation}
\left( \Gamma,\right)
\end{equation}maioro\begin{equation}
\left( \theta,\right)
\end{equation}isto\begin{equation}
\left( é,\right)
\end{equation}quantomaisacentuadasasdiferençasdetemperaturaentredoispontosda\begin{equation}
\left( barra,\right)
\end{equation}maisrápidelasse\begin{equation}
\left( modificarão,\right)
\end{equation}procurandorecairemumestado\begin{equation}
$homogêneo.$
\end{equation}Quantmaisotempo\begin{equation}
\left( transcorre,\right)
\end{equation}maishomogêneasetornaatemperaturaaolongda\begin{equation}
\left( barra,\right)
\end{equation}sejamquaistenhamsidoascondições\begin{equation}
$iniciais.$
\end{equation}AfiguradapáginseguinteéumgráficodeVcomofunçãodexe\begin{equation}
£
\end{equation}soluçãodaequaçãodcalorparaacondiçãoinicial\begin{equation}
$(desenhada$
\end{equation}emtraço\begin{equation}
$forte)$
\end{equation}deumpulsodtemperatura concentradonomeioda\begin{equation}
$barra.$
\end{equation}\begin{equation}
$,$
\end{equation}quepossuasaltoscomoasfunções\begin{equation}
V x
\end{equation}\begin{equation}
S“do
\end{equation}tipodeopção8mostradoocapítulo\begin{equation}
$1),$
\end{equation}éadoçadapela\begin{equation}
$difusão:$
\end{equation}emqualquerinstantediferentede\begin{equation}
$/=0,$
\end{equation}inçãoVxré\begin{equation}
$contínua.$
\end{equation}\begin{equation}
“
\end{equation}Aequaçãodotransportecaracterizaodeslocamentodeuma\begin{equation}
$característi-$
\end{equation}no\begin{equation}
$tempo.$
\end{equation}É\begin{equation}
\left( adequada,\right)
\end{equation}por\begin{equation}
\left( exemplo,\right)
\end{equation}paradescreveromovimentodeumsoemummeio\begin{equation}
$qualquer.$
\end{equation}Abaixoestáográficodasoluçãodaequaçãodonsportequandoacondiçãoinicialéamesmaqueadadanoproblemado\begin{equation}
$or:$
\end{equation}espaçoxtemperaturaVAequaçãodotransportedeslocaqualquerforma dadacomocondiçãoialno\begin{equation}
\left( tempo,\right)
\end{equation}\begin{equation}
\left( continuamente,\right)
\end{equation}sem\begin{equation}
$alteração.$
\end{equation}NaequaçãodeBlacke\begin{equation}
\left( oles,\right)
\end{equation}elaé aresponsávelpelocarregamentode\begin{equation}
$juros.$
\end{equation}Emtermos\begin{equation}
\left( simples,\right)
\end{equation}dacaracterísticagráfica\begin{equation}
\left( que,\right)
\end{equation}novencimentodeuma\begin{equation}
\left( opção,\right)
\end{equation}acontecerpara\begin{equation}
\left( S“acontecerá,\right)
\end{equation}emqualquerdata\begin{equation}
\left( anterior,\right)
\end{equation}para\begin{equation}
$$=$
\end{equation}\begin{equation}
$VP(S9.$
\end{equation}\begin{equation}
$Porexemplo: um$
\end{equation}stradlecompostodecalleputdestrike\begin{equation}
$K=$
\end{equation}\begin{equation}
$$100$
\end{equation}possuiseuvalormínimonoencimentoquando\begin{equation}
$$=$100,hoje,$
\end{equation}seuvalormínimonãosedaráparaS\begin{equation}
$=$100,$
\end{equation}impara\begin{equation}
$S=$
\end{equation}\begin{equation}
\operatorname{VA}{\left(100 \right)}
\end{equation}\begin{equation}
$(se$
\end{equation}\begin{equation}
$7=$
\end{equation}\begin{equation}
$5%$
\end{equation}deefetivano\begin{equation}
\left( período,\right)
\end{equation}sedará para\begin{equation}
$S=$
\end{equation}\begin{equation}
$100/$
\end{equation}2\begin{equation}
$=$
\end{equation}\begin{equation}
$695,28).$
\end{equation}\begin{equation}
$-$
\end{equation}Tendoaspropriedadesdeambasas\begin{equation}
\left( equações,\right)
\end{equation}aequaçãodeBlackecholesexprimeocomportamentodeumfenômenoquesedeslocae\begin{equation}
\mathtt{\text{derrete}}
\end{equation}moaumentodo\begin{equation}
$tempo.$
\end{equation}Diferentementedosdoisfenômenosfísicos\begin{equation}
\left( correlatos,\right)
\end{equation}umentodotempodeumaopçãoéumaumentoemseu\begin{equation}
\left( prazo,\right)
\end{equation}ouumagressãonosentidocomumdo\begin{equation}
$tempo.$
\end{equation}Adatadeexercícioéotempozeroda\begin{equation}
\left( ção,\right)
\end{equation}eéasituaçãonaqual\begin{equation}
deve - se
\end{equation}especificaracondiçãoinicialdoproblemaser\begin{equation}
$resolvido.$
\end{equation}ParaseconsideraraformaçãodepreçodeumaopçãocomoumNômeno\begin{equation}
\left( físico,\right)
\end{equation}\begin{equation}
deve - se
\end{equation}pensardetráspara\begin{equation}
\left( frente,\right)
\end{equation}partindodacurva\begin{equation}
$*x$
\end{equation}\begin{equation}
$9*$
\end{equation}ENNotequeocomportamentoprincipaldofenômenodedifusãoé\begin{equation}
$-$
\end{equation}com\begin{equation}
- o + onomeindica
\end{equation}\begin{equation}
$“derretimento”:$
\end{equation}osperfismaisacentuadossãoadoçadoscoopassardo\begin{equation}
\left( tempo,\right)
\end{equation}tantomaisrapidamentequantomaissãoacentuadc\begin{equation}
$(quanto$
\end{equation}maior\begin{equation}
\left( \Gamma,\right)
\end{equation}maior\begin{equation}
$theta).$
\end{equation}Aequaçãodo calorpossuiumapropridadechamada\begin{equation}
$normalizadora:$
\end{equation}mesmoumacondiçãoinicialdescontínua\begin{equation}
$(o$
\end{equation}\includegraphics[width=0.8\textwidth]{output/image_288png}\begin{equation}
$ções:$
\end{equation}Operando\begin{equation}
«
\end{equation}Volatilidade5756\begin{equation}
$Cvções:$
\end{equation}OperandoaVolatilidndsEqL\begin{equation}
$-$
\end{equation}Estarepresentaçãoébemmelhorquea\begin{equation}
\left( tridimensional,\right)
\end{equation}masaindaéconvenientepelapoucaclarezaqueas\begin{equation}
\left( curvas,\right)
\end{equation}deslocadasumasdas\begin{equation}
\left( outras,\right)
\end{equation}primemao\begin{equation}
$traçado.$
\end{equation}Namaioriadas\begin{equation}
\left( vezes,\right)
\end{equation}nãotrabalhamoscomvaloresominaisde\begin{equation}
\left( \mathtt{\text{S}},\right)
\end{equation}esimcomvalores\begin{equation}
\left( reais,\right)
\end{equation}razãopelaqualseriapreferíveliformizarasmedidasdoeixo5paraummesmo\begin{equation}
$padrão.$
\end{equation}PodemosabstrairefeitodetransportedaequaçãodeBlacke\begin{equation}
\left( Scholes,\right)
\end{equation}imaginandoquecadarvaVxSédesenhadaemumaescaladiferentede\begin{equation}
\left( \mathtt{\text{S}},\right)
\end{equation}detalmodoqueosloresreais\begin{equation}
$coincidam.$
\end{equation}\begin{equation}
\left( Assim,\right)
\end{equation}umaretaverticalnográficodeixade\begin{equation}
$represen-$
\end{equation}arumvalornominalde\begin{equation}
\left( 5,\right)
\end{equation}pararepresentarumvalor\begin{equation}
\left( real,\right)
\end{equation}\begin{equation}
\left( que,\right)
\end{equation}senovencimentoé\begin{equation}
$5”,$
\end{equation}entãoemcadadiaanterioraovencimentoé\begin{equation}
$VP(S).$
\end{equation}Oefeitoo\begin{equation}
$seguinte:$
\end{equation}nadatadeexercício\begin{equation}
$(a$
\end{equation}únicacoisarealmenteconhecidasobrea\begin{equation}
$opção),$
\end{equation}eimaginandoaformasedeslocarederreterconfomeprogrideparatrásnótempoatéadata\begin{equation}
$atual.$
\end{equation}Abaixoestáumgráficodaevoluçãodopreçodeuma\begin{equation}
\left( opção,\right)
\end{equation}sendodadacomocondiçãoinicialomesmopulsoexemplificadonasequaçõesanteriores\begin{equation}
—
\end{equation}pulsoestequepodeserinterpretadocomoumaposição\begin{equation}
$butterfly.$
\end{equation}vfzLAST\section{A}77AbstraçãodocarregamentoGráficostridimensionaiscomoosanterioressão\begin{equation}
\left( ilustrativos,\right)
\end{equation}masnãoseprestamaanálisesmaisprofundas\begin{equation}
$(é$
\end{equation}quaseimpossívellocalizarumpontoespecíficodentro\begin{equation}
$deles),$
\end{equation}razãopelaqualépreferívelusarcurvassuperpostasnomesmográfico\begin{equation}
$bidimensional.$
\end{equation}Nonosso\begin{equation}
\left( caso,\right)
\end{equation}representarváriascurvasVxSparadiferentesprazosou\begin{equation}
$datas:$
\end{equation}umaprojeçãodográficotridimensionalacimasobreoplanoVx\begin{equation}
\left( 5,\right)
\end{equation}comomostraa\begin{equation}
$figura:$
\end{equation}Emtodososgráficosrepresentadosdaquipor\begin{equation}
\left( diante,\right)
\end{equation}usaremosestaonvençãodeabstrairocarregamentode\begin{equation}
$juros.$
\end{equation}Tambémreduziremosomerodecurvasaomínimonecessário\begin{equation}
$(3$
\end{equation}ou\begin{equation}
$4)$
\end{equation}paraaclarezadoqueseretende\begin{equation}
$expor.$
\end{equation}SoluçãoeparâmetrosSoluçãodaequaçãodeBlackeScholes\begin{equation}
Tendo - se
\end{equation}aequaçãodiferencialdeBlackeScholeseumafunção\begin{equation}
$V*=$
\end{equation}\begin{equation}
$AS”)$
\end{equation}mocondição\begin{equation}
\left( inicial,\right)
\end{equation}\begin{equation}
pode - se
\end{equation}determinarafunção\begin{equation}
\left( VS,\right)
\end{equation}\begin{equation}
$4)$
\end{equation}valordaopçãocujoyojfédadocomocondição\begin{equation}
\left( inicial,\right)
\end{equation}emqualquer\begin{equation}
\left( data,\right)
\end{equation}eparaqualquervalore\begin{equation}
5.0
\end{equation}BlackeScholesutilizaramsuaequaçãodiferencialsobreacondição\begin{equation}
$inicial:$
\end{equation}\begin{equation}
$C(S%$
\end{equation}\begin{equation}
$0)$
\end{equation}\begin{equation}
$=$
\end{equation}\begin{equation}
$max(0,$
\end{equation}\begin{equation}
$S*—$
\end{equation}\begin{equation}
$X),$
\end{equation}queéopayoffsdeuma\begin{equation}
\left( cal,\right)
\end{equation}eobtiveramumaSolução\begin{equation}
$analítica:$
\end{equation}\includegraphics[width=0.8\textwidth]{output/image_290png}58\begin{equation}
$Opções:$
\end{equation}OperandoaVolatilidade\begin{equation}
$.$
\end{equation}\begin{equation}
$ões:$
\end{equation}OperandoaVolatilidade59\begin{equation}
$-$
\end{equation}\begin{equation}
$.$
\end{equation}\begin{equation}
3.6
\end{equation}\begin{equation}
$(=$
\end{equation}6x\begin{equation}
$Ma)$
\end{equation}VEKOx\section{O}\begin{equation}
3.6
\end{equation}Gregasderivadasde\begin{equation}
$V-$
\end{equation}Vega\begin{equation}
7
\end{equation}erhô\begin{equation}
$tp)$
\end{equation}DepossedasoluçãoanalíticadaequaçãodeBlacke\begin{equation}
\left( Scholes,\right)
\end{equation}podemosbterfórmulasparavaloresde\begin{equation}
\left( \delta,\right)
\end{equation}gammae\begin{equation}
\left( \theta,\right)
\end{equation}porsimples\begin{equation}
$derivação.$
\end{equation}Essesvalorespodemserchamadosdederivadasdafórmulade\begin{equation}
\left( precificação,\right)
\end{equation}asébemmaiscomum\begin{equation}
chamá - los
\end{equation}de\begin{equation}
\left( gregas,\right)
\end{equation}pelofatodequeelessãoeralmentenotadosporletras\begin{equation}
$gregas.$
\end{equation}Acrescentaremosduasnovas\begin{equation}
$gregas.$
\end{equation}\begin{equation}
$Primeira:$
\end{equation}\begin{equation}
\left( rhô,\right)
\end{equation}representado\begin{equation}
$porp:a$
\end{equation}erivadadoprêmiodeumaopçãoemrelaçãoàtaxadejuro\begin{equation}
7.0
\end{equation}Rhôéumaedidadequantosreais\begin{equation}
$(ou$
\end{equation}\begin{equation}
$centavos)$
\end{equation}opreçodeuniaopçãosobe\begin{equation}
$(ou$
\end{equation}\begin{equation}
$cai)$
\end{equation}setaxasdejurosubirem\begin{equation}
$1%.$
\end{equation}E\begin{equation}
\left( vega,\right)
\end{equation}representadopor\begin{equation}
\mathtt{\text{V}}
\end{equation}manuscrito\begin{equation}
$(vega$
\end{equation}ãoéumaletra\begin{equation}
$grega),$
\end{equation}\begin{equation}
\left( 7,\right)
\end{equation}aderivadadoprêmiodeumaopçãoemrelaçãoà\begin{equation}
$olatilidade.$
\end{equation}Vegamedequantosreais\begin{equation}
$(ou$
\end{equation}\begin{equation}
$centavos)$
\end{equation}opreçodaopçãosobesevolatilidadesobe\begin{equation}
$1%$
\end{equation}\begin{equation}
$(alguns autores$
\end{equation}designamestaderivadapor\begin{equation}
\left( \kappa,\right)
\end{equation}eapresentampelaletrakappa\begin{equation}
—
\end{equation}\begin{equation}
$x).$
\end{equation}Asgregasdeuma\begin{equation}
cal!
\end{equation}\begin{equation}
—
\end{equation}asderivadasdoprêmiodeuma\begin{equation}
\left\lfloor{\frac{ca}{—}}\right\rfloor
\end{equation}\begin{equation}
$são:$
\end{equation}\section{E}oNt\begin{equation}
$CIn(S/VP(K))$
\end{equation}\section{A}oNt2onde\begin{equation}
$Mw)$
\end{equation}éafunçãodeprobabilidadeacumuladaabaixodacurva\begin{equation}
\left( normal,\right)
\end{equation}\begin{equation}
$MO)$
\end{equation}éiguala\begin{equation}
$0,50;$
\end{equation}\begin{equation}
$M(1-+0o)$
\end{equation}éiguala\begin{equation}
$1;e$
\end{equation}\begin{equation}
Ma»
\end{equation}\begin{equation}
—
\end{equation}\begin{equation}
$oo)$
\end{equation}éiguala\begin{equation}
$O.$
\end{equation}Nãoexisteumaexpressãoalgébricaparaafunção\begin{equation}
$Ma),$
\end{equation}queéiguala\begin{equation}
$N()$
\end{equation}\begin{equation}
$=>$
\end{equation}\begin{equation}
$[e$
\end{equation}\begin{equation}
$"ax$
\end{equation}1V27\begin{equation}
$+.$
\end{equation}\begin{equation}
$=N(a)$
\end{equation}Aau\begin{equation}
$=$
\end{equation}masseuvalorpodeserdeterminadoporintegraçãonumérica\begin{equation}
$(o$
\end{equation}método1menos\begin{equation}
$recomendado),$
\end{equation}porbusca emtabelas\begin{equation}
$(M(x)$
\end{equation}estátabuladaemváriosLat\begin{equation}
$=$
\end{equation}\begin{equation}
$Na)$
\end{equation}ioam\begin{equation}
$:$
\end{equation}\begin{equation}
«
\end{equation}\begin{equation}
$.$
\end{equation}\begin{equation}
$:$
\end{equation}Sostlivrosdeprobabilidadee\begin{equation}
$estatística),$
\end{equation}ouporumaaproximaçãopolinomialqueseráexpostanoúltimo\begin{equation}
$capítulo.$
\end{equation}Mais\begin{equation}
\left( tarde,\right)
\end{equation}otermo\begin{equation}
$Ma)$
\end{equation}seráidentificadoopor nóscomo\begin{equation}
$delta.$
\end{equation}Bean\begin{equation}
$=$
\end{equation}SE\begin{equation}
\operatorname{Nº}{\left(a \right)}
\end{equation}rx\begin{equation}
$VPÇONÇ)$
\end{equation}Asoluçãoacimaéparaaprecificaçãode\begin{equation}
$cal!s.$
\end{equation}\begin{equation}
Alimentando - se
\end{equation}aequaçãodiferencialdeBlackeScholescomacondiçãoinicialdeuma\begin{equation}
\left( put,\right)
\end{equation}aqualéASpEx\begin{equation}
\operatorname{VP}{\left(K \right)}
\end{equation}\begin{equation}
b
\end{equation}\begin{equation}
$0)$
\end{equation}\begin{equation}
$=$
\end{equation}\begin{equation}
$max(0,X—$
\end{equation}\begin{equation}
\left( 5,\right)
\end{equation}chegamosàfórmulaanalíticadopreçodeuma\begin{equation}
$pi/f.$
\end{equation}Umaato\begin{equation}
8
\end{equation}maneiraalternativaemaissimplesde\begin{equation}
- lo + obtê
\end{equation}ésubstituirafórmuladopreçodacalinadaparidade\begin{equation}
$put-cal!.$
\end{equation}Dequalquer\begin{equation}
\left( forma,\right)
\end{equation}\begin{equation}
$obtém-se:$
\end{equation}\begin{equation}
$P=VP(K)$
\end{equation}\begin{equation}
$xN(ca)-S$
\end{equation}x\begin{equation}
- b
\end{equation}\begin{equation}
$(3.$
\end{equation}Vea\begin{equation}
$=$
\end{equation}\begin{equation}
$SVtNa)$
\end{equation}nde\begin{equation}
x
\end{equation}éafunçãodedensidadedaprobabilidadesobacurva\begin{equation}
$normal.$
\end{equation}Elaáovalordaordenadadafunçãodedistribuição\begin{equation}
\left( normal,\right)
\end{equation}aopassoque\begin{equation}
$Ma)$
\end{equation}âovalordaáreasobacurva\begin{equation}
$(a$
\end{equation}\begin{equation}
$integral).$
\end{equation}Aocontráriode\begin{equation}
$Ma),$
\end{equation}quetemdeseralculadapormétodos\begin{equation}
\left( numéricos,\right)
\end{equation}háumafórmulapara\begin{equation}
$N(%),$
\end{equation}que\begin{equation}
$é:$
\end{equation}\begin{equation}
“q
\end{equation}\begin{equation}
$onde:$
\end{equation}\begin{equation}
$—n[S/VP(S)]$
\end{equation}\section{SE}\begin{equation}
$*$
\end{equation}\begin{equation}
$+$
\end{equation}nmGQ\begin{equation}
$N()=—p=e""$
\end{equation}2x\begin{equation}
$In[S/VPUO]$
\end{equation}1oNE\begin{equation}
$b=""DE$
\end{equation}\section{5}Paradeterminarasgregasdasputs\begin{equation}
$-$
\end{equation}asderivadasdopreço\begin{equation}
$?$
\end{equation}deuma\begin{equation}
$p71//$
\end{equation}étambémmaissimpleslançarmãodaparidade\begin{equation}
$put-call.$
\end{equation}\begin{equation}
Derivando - se
\end{equation}a\includegraphics[width=0.8\textwidth]{output/image_292png}60\begin{equation}
$Opções:$
\end{equation}OperandoaVolatilida\begin{equation}
$as:$
\end{equation}OperandoaVolatilidade\begin{equation}
$6]$
\end{equation}paridade\begin{equation}
- \frac{cal}{em} + put
\end{equation}relaçãoa\begin{equation}
\left( 5, \  a\right)
\end{equation}rea\begin{equation}
£
\end{equation}\begin{equation}
obtêm - se
\end{equation}relaçõesentre\begin{equation}
\left\lfloor{\frac{454}{€}}\right\rfloor
\end{equation}\begin{equation}
\left( Ay,\right)
\end{equation}entre\begin{equation}
\left\lfloor{\frac{Pca}{€}}\right\rfloor
\end{equation}PputeentreOcaire6putquefornecemasgregasdas\begin{equation}
$puís.$
\end{equation}\begin{equation}
\left( Note - se,\right)
\end{equation}queveganãoestápresentenaparidade\begin{equation}
\left( - call + puí,\right)
\end{equation}aqualexprimeumarelaçãdepreço\begin{equation}
$necessária.$
\end{equation}Elícitoderivarafórmuladaparidadeemrelaçãoavegobtendoassimque\begin{equation}
\left( V, \  \left\lfloor{\frac{4}{—}}\right\rfloor\right)
\end{equation}Vout\begin{equation}
$=$
\end{equation}\begin{equation}
\left( 0, \  0\right)
\end{equation}queequivaleadizerque\begin{equation}
\frac{cal}{e}
\end{equation}putÀmesmovencimentoestrikedevemtervegas\begin{equation}
$iguais.$
\end{equation}Omesmoraciocínio5aplicaaderivarmosafórmuladaparidadeduasvezesemrelaçãoa\begin{equation}
\left( 5,\right)
\end{equation}parobtermosumarelaçãoentreosgammasde\begin{equation}
\left\lfloor{\frac{ca}{e}}\right\rfloor
\end{equation}de\begin{equation}
$put.$
\end{equation}Osresultados\begin{equation}
$são:$
\end{equation}ópriadistribuiçãode\begin{equation}
\left( 5,\right)
\end{equation}sejamdependentesdaescalaemqueseobservaovimentodo\begin{equation}
$mercado);$
\end{equation}e\begin{equation}
$3)$
\end{equation}adequeocaminhamentode\begin{equation}
$$S$
\end{equation}incluaômenosdeautocorrelação\begin{equation}
tendências
\end{equation}edeheterocedasticidade\begin{equation}
$(ten-$
\end{equation}ciasde\begin{equation}
$volatilidade).$
\end{equation}Nesseterceiro\begin{equation}
\left( caso,\right)
\end{equation}omodeloquesesupõesentarocaminhamentodopreçodoobjetonãopodemaisser\begin{equation}
$consi-$
\end{equation}adoumpasseio\begin{equation}
\left( aleatório,\right)
\end{equation}sequerpor\begin{equation}
\left( aproximação,\right)
\end{equation}e\begin{equation}
- se + tem
\end{equation}debuscarsmodelosde\begin{equation}
$comportamento.$
\end{equation}buiçõesalternativasÁ\begin{equation}
$=$
\end{equation}Aca\begin{equation}
-1
\end{equation}puéPout\begin{equation}
$=$
\end{equation}\begin{equation}
$=0$
\end{equation}Aprimeiralinhadecontestaçãoé arespostaimediataao\begin{equation}
- ajustamento + não
\end{equation}modeloBlackeScholesaqualquercondiçãodemercado\begin{equation}
$não-prevista.$
\end{equation}\begin{equation}
\left( Se,\right)
\end{equation}\begin{equation}
“dado
\end{equation}\begin{equation}
\left( momento,\right)
\end{equation}asopçõesmais\begin{equation}
$7x-the-money$
\end{equation}estãomaiscarasdoquecall9\begin{equation}
— + \operatorname{VP}{\left(K \right)}
\end{equation}callput\begin{equation}
\operatorname{txVP}{\left(K \right)}
\end{equation}conizao\begin{equation}
\left( modelo,\right)
\end{equation}ouasopçõesmais\begin{equation}
- money - \frac{\left\lfloor{\tilde{\infty} o_{1}}\right\rfloor}{the}
\end{equation}estãomais\begin{equation}
\left( baratas,\right)
\end{equation}ouPput\begin{equation}
$=$
\end{equation}Poa\begin{equation}
“ao
\end{equation}Iquerquesejao\begin{equation}
\left( desvio,\right)
\end{equation}eleéconsideradoprimariamenteumerroemumiradistribuiçãodeprobabilidadescomo\begin{equation}
$lognormai.$
\end{equation}VousVetUmerrosistemáticodomodeloBlackeScholesquepareceteressalicaçãoéadepreciaçãodospreçosprevistosparaasopçõesnão\begin{equation}
$af-the-$
\end{equation}\begin{equation}
\left( rey,\right)
\end{equation}observadanosmercados\begin{equation}
\left( internacionais,\right)
\end{equation}chamadasite\begin{equation}
\left( effect,\right)
\end{equation}efeito\begin{equation}
\left( riso,\right)
\end{equation}ouvolatility\begin{equation}
\left( smile,\right)
\end{equation}etratadamelhornoCapítulo\begin{equation}
\left( 6,\right)
\end{equation}naseçãosobretimativasde\begin{equation}
$volatilidade”.$
\end{equation}Paracontornarassupostasdeficiênciasdasuposiçãode\begin{equation}
\left( lognormalidade,\right)
\end{equation}\begin{equation}
- se
\end{equation}tentado introduziroutras\begin{equation}
\left( distribuições,\right)
\end{equation}especialmentedistribuiçõespíricasobservadasna\begin{equation}
$prática.$
\end{equation}Oproblemacomestaabordageméaextremaculdadeemobterumadistribuiçãorazoávelapartirdedados\begin{equation}
$históricos.$
\end{equation}\begin{equation}
\left( averdade,\right)
\end{equation}omercadoestásempre\begin{equation}
$mudando:$
\end{equation}seemumperíododetempoeleexibeocomportamento\begin{equation}
\left( de,\right)
\end{equation}por\begin{equation}
\left( exemplo,\right)
\end{equation}acentuarosgrandesgaysdepreçoa\begin{equation}
\left( cima,\right)
\end{equation}nooutroelepodeexibirocomportamento\begin{equation}
$inverso.$
\end{equation}Àspoucasservaçõesdemovimentosextremos\begin{equation}
$(grandes$
\end{equation}altasebaixas\begin{equation}
$rápidas),$
\end{equation}quetendemexplicara\begin{equation}
- aderência + não
\end{equation}doBlackeScholesemcertos\begin{equation}
\left( casos,\right)
\end{equation}stituemum\begin{equation}
“quarto
\end{equation}\begin{equation}
escuro”
\end{equation}estatísticoamenosquesejamtomadasemnero\begin{equation}
\left( suficiente, \  ou\right)
\end{equation}\begin{equation}
\left( seja,\right)
\end{equation}emumhistórico\begin{equation}
$longo.$
\end{equation}\begin{equation}
\left( Mas,\right)
\end{equation}aosetomarumótico\begin{equation}
\left( longo,\right)
\end{equation}\begin{equation}
abre - se
\end{equation}mãodecaptarascaracterísticasmaisrecentesdo\begin{equation}
$ercado.$
\end{equation}Esteéoparadoxoinerenteaqualquerajustamento\begin{equation}
$temporal.$
\end{equation}Oproblemaparece\begin{equation}
agravar - se
\end{equation}no\begin{equation}
\left( Brasil,\right)
\end{equation}emquenenhumhistóricoparececontadamiríadedepropriedadesdiferentesdos\begin{equation}
$mercados.$
\end{equation}ÀHiplicidadedepolíticascambiaisatrapalhaaanálisehistóricadospreçostermnacionaiscotadosemmoedalocal\begin{equation}
$(como$
\end{equation}osdo\begin{equation}
$ouro).$
\end{equation}Ospreçosqueelhorseprestariamatalanáliseseriamosde\begin{equation}
\left( ações,\right)
\end{equation}porseromercadodeCríticasaomodeloCríticasaoBlackeScholes\begin{equation}
—
\end{equation}HipótesedelognormalidadeJásevãomaisde20anosdesdequeomodeloBlackeScholesfoi\begin{equation}
$lançado.$
\end{equation}Énatural\begin{equation}
\left( que,\right)
\end{equation}tantoemseusdiasiniciaisquanto\begin{equation}
\left( agora,\right)
\end{equation}hajacertascríticasaseremfeitasa\begin{equation}
$ele.$
\end{equation}TalvezapremissamaiscriticadadomodeloBlackeScholessejaadalognormalidadede\begin{equation}
$S.$
\end{equation}Issonaseconomias\begin{equation}
\left( estabilizadas,\right)
\end{equation}porquenoBrasildaaltainflação\begin{equation}
$+$
\end{equation}jurosedosprazos\begin{equation}
\left( curtos,\right)
\end{equation}aestabilidadedastaxasdejuroéapremissamais\begin{equation}
$fraca.$
\end{equation}Sejacomo\begin{equation}
$for,$
\end{equation}váriosestudostentarammensurarainexatidãodahipótesedelognormalidadede\begin{equation}
\left( 5,\right)
\end{equation}ou\begin{equation}
- la + substitui
\end{equation}poroutra\begin{equation}
$distribuição.$
\end{equation}Quasesempreosresultadossãomodeloscomplicadoseporissopouco\begin{equation}
$usados.$
\end{equation}MasrelativizarosresultadosdoBlacke\begin{equation}
\left( Scholes,\right)
\end{equation}isto\begin{equation}
\left( é,\right)
\end{equation}definircondiçõeselimitesàsua\begin{equation}
\left( aplicação,\right)
\end{equation}podeseruma\begin{equation}
$necessidade.$
\end{equation}Apremissadelognormalidadepodesercontestadaemtrês\begin{equation}
$frentes:$
\end{equation}\begin{equation}
$1)$
\end{equation}ade\begin{equation}
\left( que,\right)
\end{equation}dadoum\begin{equation}
\left( prazo,\right)
\end{equation}odesenhodadistribuiçãodeprobabilidades\begin{equation}
afaste - se
\end{equation}dopreconizadopelacurva\begin{equation}
$lognormal;2)a$
\end{equation}dequeocaminhamentode\begin{equation}
$$$
\end{equation}nãosejaperfeitamente\begin{equation}
\left( aleatório,\right)
\end{equation}equeavolatilidadeparaperíodosmaislongosoumovimentosmaiores\begin{equation}
difira - se
\end{equation}davolatilidadeobservadanosperíodoscurtosemovimentoscontidos\begin{equation}
$(ou$
\end{equation}\begin{equation}
\left( seja,\right)
\end{equation}quea\begin{equation}
\left( volatilidade,\right)
\end{equation}e\includegraphics[width=0.8\textwidth]{output/image_294png}62\begin{equation}
$Opções:$
\end{equation}OperandoaVolatilida\begin{equation}
$=:$
\end{equation}OperandoàVolatilidade63Algumasinformaçõespodemsertiradasdesse\begin{equation}
$gráfico:$
\end{equation}\begin{equation}
$1)$
\end{equation}mesmo\begin{equation}
$conside-$
\end{equation}\begin{equation}
ndo - se
\end{equation}umperíododetempolargo\begin{equation}
$(nove$
\end{equation}\begin{equation}
$anos),$
\end{equation}eumpapelque sofreupoucarreçãoneste\begin{equation}
\left( tempo,\right)
\end{equation}asprobabilidadesdealtaaparecemmaioresqueasdeixa\begin{equation}
$(a$
\end{equation}áreadohistogramaàdireitadozeroémaiordoqueaáreaà\begin{equation}
$esquerda);$
\end{equation}distribuiçãoé\begin{equation}
\left( leptocúrtica,\right)
\end{equation}ou\begin{equation}
\left( seja,\right)
\end{equation}apesardeteramesmamédiaedesviorãorepresentadosnalinha\begin{equation}
\left( sobreposta,\right)
\end{equation}elaémaisconcentradanopicoe\begin{equation}
$extremos.$
\end{equation}Akurtosiséumparâmetroextraqueindicaoquantoumastribuiçãoseconcentranosextremosenopico\begin{equation}
$(kurtosis$
\end{equation}\begin{equation}
$positiva)$
\end{equation}ouontrário\begin{equation}
$(kurtosis$
\end{equation}\begin{equation}
$negativa);$
\end{equation}\begin{equation}
$3)$
\end{equation}existemalgumasobservaçõesalémdo\begin{equation}
$inter-$
\end{equation}o\begin{equation}
$(-0,2a$
\end{equation}\begin{equation}
$+0,2).$
\end{equation}Paraumadistribuição\begin{equation}
\left( normal, \  sem\right)
\end{equation}\begin{equation}
\left( kurtosis,\right)
\end{equation}essas\begin{equation}
$observa-$
\end{equation}esseriamextremamenteimprováveis\begin{equation}
$(na$
\end{equation}amostrade\begin{equation}
2.2
\end{equation}\begin{equation}
\left( observações,\right)
\end{equation}enosde\begin{equation}
$0,01%$
\end{equation}\begin{equation}
—
\end{equation}isto\begin{equation}
\left( é,\right)
\end{equation}menosdezrrzaobservação\begin{equation}
—
\end{equation}poderiacalharparaIquerdos\begin{equation}
$lados;$
\end{equation}na\begin{equation}
\left( realidade,\right)
\end{equation}asobservaçõesalémdoslimitessomam\begin{equation}
\left( 8,\right)
\end{equation}cluindoumaquedade\begin{equation}
$40%$
\end{equation}emumúnicodiadoCaso\begin{equation}
\left( Nahas,\right)
\end{equation}\begin{equation}
\left( que,\right)
\end{equation}nocasoderdeserprognosticadaporumadistribuição\begin{equation}
\left( normal,\right)
\end{equation}poderiaocorrerumaacada4milhõesde\begin{equation}
$anos).$
\end{equation}AexistênciadessasobservaçõesextremasajudaexplicarporqueoBlackeScholesdepreciaopções\begin{equation}
$out-of-the-money:$
\end{equation}na\begin{equation}
\left( »rdade,\right)
\end{equation}movimentosextremossãomaisprováveisdoquepodemospredizerbaseemumadistribuição\begin{equation}
\left( normal,\right)
\end{equation}sem\begin{equation}
\left( kurtosis,\right)
\end{equation}eportantoopções\begin{equation}
$out-of-$
\end{equation}moneytêmumapeloextra\begin{equation}
$não-previsto.$
\end{equation}Ascaracterísticasdessegráficosãobemparecidascomasencontradasnasõesdeempresasamericanas\begin{equation}
—
\end{equation}inclusiveakurtosis\begin{equation}
$positiva.$
\end{equation}\begin{equation}
\left( Recentemente,\right)
\end{equation}\begin{equation}
m - se
\end{equation}dadoatençãoespecialadistribuiçõescomo\begin{equation}
$esta.$
\end{equation}\begin{equation}
Note - se
\end{equation}\begin{equation}
\left( que,\right)
\end{equation}paraingirummínimodeconforto\begin{equation}
\left( estatístico,\right)
\end{equation}buscamosnoveanosdedados\begin{equation}
$stóricos.$
\end{equation}Qualaformaqueseoriginariadeumasériemaiscurtade\begin{equation}
\left( dados,\right)
\end{equation}émapergunta\begin{equation}
$interessante.$
\end{equation}\begin{equation}
$"$
\end{equation}Apesardetodasas\begin{equation}
\left( dificuldades,\right)
\end{equation}apropostadeajustamentoda\begin{equation}
$distribui-$
\end{equation}odeprobabilidadesestálongedeser\begin{equation}
$inútil.$
\end{equation}Muitosoperadores\begin{equation}
\left( conseguem,\right)
\end{equation}elhorarsensivelmenteaqualidadedesuasprevisõessobreopreçodeercadodasopçõesutilizandodistribuições\begin{equation}
$alternativas.$
\end{equation}Ométodousualérimeirodescobriradistribuiçãoquemelhorse\begin{equation}
\left( encaixa,\right)
\end{equation}paradepois\begin{equation}
$utilizá-$
\end{equation}\begin{equation}
$sistematicamente.$
\end{equation}Essemétodoécorrelatoàanálise\begin{equation}
- orientada + preço
\end{equation}\begin{equation}
$(Gá$
\end{equation}rdadanocapítulo\begin{equation}
$1),$
\end{equation}emqueoobjetivoédeterminaropreçojustodas\begin{equation}
$ões.$
\end{equation}bolsaummercado\begin{equation}
$livre.$
\end{equation}\begin{equation}
\left( Contudo,\right)
\end{equation}omercadoacionáriobrasileirovive\begin{equation}
$u;$
\end{equation}momentoexcepcionaldesde\begin{equation}
\left( 1990,\right)
\end{equation}quando\begin{equation}
iniciou - se
\end{equation}oprocessodeajustametoda\begin{equation}
\left( economia,\right)
\end{equation}etodaanálisefeitaapartirdeentãoincorporaráestepanod\begin{equation}
$fundo.$
\end{equation}Antesde\begin{equation}
\left( 1990,\right)
\end{equation}muitos\begin{equation}
\left( papéis,\right)
\end{equation}comoTelebrás\begin{equation}
$(que$
\end{equation}nessesquatúltimosanosrespondeuporemmédia\begin{equation}
$40%$
\end{equation}dovolumeda\begin{equation}
$Bovespa)$
\end{equation}nãoeraativamentenegociadosounão\begin{equation}
$existiam.$
\end{equation}AprópriaTelebrássofreuumajusdesetevezesemseu\begin{equation}
$preço.$
\end{equation}Aosetomardadosdasdécadas\begin{equation}
\left( anteriores,\right)
\end{equation}vamverificarquemuitasempresasbrasileirasantespreponderantesdeixaramdserativamentenegociadasembolsa\begin{equation}
$(viraram$
\end{equation}\begin{equation}
$“micos"),$
\end{equation}equeoutrassurgira\begin{equation}
$recentemente.$
\end{equation}Emmeioàinstabilidadegeralqueimpedeumaabordageestatísticamais\begin{equation}
\left( acurada,\right)
\end{equation}parecem\begin{equation}
salvar - se
\end{equation}alguns\begin{equation}
\left( papéis,\right)
\end{equation}comoValedoRi\begin{equation}
\left( Doce,\right)
\end{equation}BancodoBrasile\begin{equation}
$Brahma.$
\end{equation}AbaixoestáumhistogramadasdiferençlogarítmicasdiáriasdeValedoRioDoce\begin{equation}
$PN.$
\end{equation}Sobrepostoa\begin{equation}
\left( ele,\right)
\end{equation}estáalinhadistribuiçãonormaldeprobabilidadesparaavolatilidademedidanesteperíod\begin{equation}
$(de$
\end{equation}janeirode1986adezembrode\begin{equation}
$1994),$
\end{equation}aqualfoide\begin{equation}
$5,4%$
\end{equation}\begin{equation}
$(base$
\end{equation}\begin{equation}
$diária).$
\end{equation}\begin{equation}
120.0
\end{equation}\begin{equation}
$+$
\end{equation}\begin{equation}
100.0
\end{equation}\begin{equation}
$+$
\end{equation}\begin{equation}
80.0
\end{equation}7\begin{equation}
60.0
\end{equation}\begin{equation}
$+$
\end{equation}\begin{equation}
40.0
\end{equation}7\begin{equation}
20.0
\end{equation}\begin{equation}
$+$
\end{equation}\begin{equation}
0.0
\end{equation}erreiAescalaverticaldográficoestágraduadaemnúmerodeobservaçõesnessesnoveanos\begin{equation}
\left( analisados,\right)
\end{equation}ecadabarradohistogramaabrangeumintervalode\begin{equation}
$0,005,$
\end{equation}ouaproximadamente\begin{equation}
$0,5%$
\end{equation}demovimentono\begin{equation}
$papel.$
\end{equation}Avolatilidadestáassinaladapelasduaslinhaspontilhadasquedelimitamointervalodeumdesviopadrãoparacimaeparabaixoda\begin{equation}
$média.$
\end{equation}Aescalahorizontalesgraduadaempontosdediferençaentreologaritmodopreçoemumdia\begin{equation}
e!
\end{equation}logaritmodopreçonodia\begin{equation}
\left( anterior,\right)
\end{equation}ouaproximadamenteavariaçãopercentudeumdiaparao\begin{equation}
$outro.$
\end{equation}\begin{equation}
\left( Assim,\right)
\end{equation}ospontosextremos\begin{equation}
\left( 0, \  2\right)
\end{equation}e\begin{equation}
\left( 0, \  2\right)
\end{equation}indicanrespectivamentemovimentosdiáriosdeaproximadamente\begin{equation}
$20%$
\end{equation}parabaixoparacima\begin{equation}
$(rigorosamente,$
\end{equation}\begin{equation}
$18%$
\end{equation}parabaixoe\begin{equation}
$22%$
\end{equation}para\begin{equation}
$cima).$
\end{equation}DistribuiçãodependentedaescalaAsegundalinhadecontestaçãoatentaparaqueadistribuiçãode\begin{equation}
$proba-$
\end{equation}idadesdeSpodenãopossuirasmesmascaracterísticasparaqualquerprazo\includegraphics[width=0.8\textwidth]{output/image_296png}64\begin{equation}
$Opções:$
\end{equation}Operandoa\begin{equation}
\left( Volatilidad,\right)
\end{equation}\begin{equation}
$ses;$
\end{equation}OperandoaVolatilidade65e\begin{equation}
\left( amplitude,\right)
\end{equation}ou\begin{equation}
\left( seja,\right)
\end{equation}podeserdependentedaescalaemqueseobservacomportamentode\begin{equation}
$S.$
\end{equation}HoroomedidaAshipótesesde\begin{equation}
\left( lognormalidade,\right)
\end{equation}ausênciadeautocorrelação\begin{equation}
$(retorno$
\end{equation}7TO2705diáriosindependentesentre\begin{equation}
$si)$
\end{equation}ehomocedasticidade\begin{equation}
$(volatilidade$
\end{equation}constante51538371implicamqueopreçoSpossa\begin{equation}
distanciar - se
\end{equation}infinitamentedopontodepartid109408526atual\begin{equation}
$-$
\end{equation}seuvalor\begin{equation}
$hoje.$
\end{equation}Nãoérazoável\begin{equation}
esperar - se
\end{equation}issodopreçodeumaação18822531\begin{equation}
$(nem$
\end{equation}denenhumpreço\begin{equation}
$real).$
\end{equation}Todososativosreaisexibemalgumtipode7ncay40\begin{equation}
32.67
\end{equation}7047reversion\begin{equation}
$(reversão$
\end{equation}paraa\begin{equation}
$média),$
\end{equation}ou\begin{equation}
\left( seja,\right)
\end{equation}tendênciaembuscaralgumvalo80so58\begin{equation}
25.89
\end{equation}histórico\begin{equation}
$fundamental.$
\end{equation}Nocasodastaxasde\begin{equation}
\left( juro,\right)
\end{equation}estatendênciaémuit70097139\begin{equation}
\left( pronunciada,\right)
\end{equation}apontodeinviabilizaralargautilizaçãoqueoBlackeSchole120áxúencontranasopçõesdebolsaede\begin{equation}
$commodities.$
\end{equation}Nãoestamosfalandodeopçõedejuro\begin{equation}
\left( agora,\right)
\end{equation}masoefeitoda7xeamreversionexisteemquaisquer\begin{equation}
$mercados.$
\end{equation}deseesperarquequantomaiorfora\begin{equation}
\left( volatilidade,\right)
\end{equation}medidanocurto\begin{equation}
\left( prazo,\right)
\end{equation}dum\begin{equation}
\left( mercado,\right)
\end{equation}maisesteefeitoatrapalharáasuposiçãodelognormalidaddeste\begin{equation}
$mercado.$
\end{equation}UmmercadoqueexcursionafrequentementeentreseusvaloresmáximoemínimofundamentaishistóricostemtudoparanãopassarnParavaloresde77menoresque40dias\begin{equation}
\left( úteis,\right)
\end{equation}podemosconsiderarquealeseassentarazoavelmentenoperfil\begin{equation}
\left( lognormal,\right)
\end{equation}masparavalores\begin{equation}
\left( periores,\right)
\end{equation}oresultadoé\begin{equation}
$sofrível.$
\end{equation}Comopropósitodetraçaruma\begin{equation}
$compa-$
\end{equation}\begin{equation}
\left( ão,\right)
\end{equation}incluímosumatabelasemelhanteparaoÍndiceDowJones\begin{equation}
$(Industri-$
\end{equation}\begin{equation}
$Average),$
\end{equation}calculadasobreumasériededezanos\begin{equation}
$(de$
\end{equation}1984a\begin{equation}
$1994)$
\end{equation}deeçosde\begin{equation}
$fechamento:$
\end{equation}testedasnossashipóteses\begin{equation}
$simplificadoras.$
\end{equation}PodemostestarfacilmenteseummercadoéindependentedaescalaouB\begin{equation}
$crio?$
\end{equation}omedida\begin{equation}
$não:$
\end{equation}enquantoasériederetornosdeumpreçonãoexibirautocorrelaçãonemheterocedasticidade\begin{equation}
$(isto$
\end{equation}\begin{equation}
\left( é,\right)
\end{equation}enquantoosretornosforem\begin{equation}
\left( independentes,\right)
\end{equation}e1\begin{equation}
\left( 1, \  0\right)
\end{equation}2140volatilidadefor\begin{equation}
$constante),$
\end{equation}odesviopadrãodosretornosde\begin{equation}
»zdias
\end{equation}deveserJn5\begin{equation}
\left( 4, \  96\right)
\end{equation}\begin{equation}
\left( 21, \  31\right)
\end{equation}vezesodesviopadrãodosretornosdiários\begin{equation}
$(esta$
\end{equation}éanossaconhecidafórmul10\begin{equation}
\left( 9, \  81\right)
\end{equation}\begin{equation}
\left( 21, \  19\right)
\end{equation}deconversãodeunidadesde\begin{equation}
$volatilidade).$
\end{equation}\begin{equation}
\left( Ou,\right)
\end{equation}deoutro\begin{equation}
\left( modo,\right)
\end{equation}segéodesvio20\begin{equation}
\left( 19, \  24\right)
\end{equation}\begin{equation}
\left( 20, \  99\right)
\end{equation}padrãodosretornosdiáriose67odesviodosretornosem77\begin{equation}
\left( dias,\right)
\end{equation}\begin{equation}
$então:$
\end{equation}40\begin{equation}
$37,02$
\end{equation}\begin{equation}
\left( 20, \  59\right)
\end{equation}80\begin{equation}
\left( 64, \  82.0\right)
\end{equation}\begin{equation}
\left( 19, \  26\right)
\end{equation}120\begin{equation}
\left( 83, \  69\right)
\end{equation}\begin{equation}
\left( 17, \  88\right)
\end{equation}Comosepode\begin{equation}
\left( ver,\right)
\end{equation}aquitambémofenômenode7ncanreversionfazcairOalordavolatilidadedelongo\begin{equation}
$prazo.$
\end{equation}Neste\begin{equation}
\left( mercado,\right)
\end{equation}aquedaéde\begin{equation}
$praticamen-$
\end{equation}\begin{equation}
$3,5%.$
\end{equation}Outrofatorquefazcairasvolatilidadesparaprazosmaislongoséateficiênciadomercadoemserindependenteda\begin{equation}
$escala.$
\end{equation}Emummercadoquexibacomportamentodepreçosindependenteda\begin{equation}
\left( escala,\right)
\end{equation}gráficosde\begin{equation}
\left( 5,\right)
\end{equation}\begin{equation}
\left( 30,\right)
\end{equation}120\begin{equation}
\left( inutos,\right)
\end{equation}\begin{equation}
\left( diários,\right)
\end{equation}semanaisemensais\begin{equation}
parecem - se
\end{equation}\begin{equation}
$substancialmente:$
\end{equation}sevocêospresentaaum\begin{equation}
\left( operador,\right)
\end{equation}semmostraroperíododetempoque\begin{equation}
\left( abrangem,\right)
\end{equation}eleãoconseguirá\begin{equation}
identificá - los
\end{equation}àprimeira\begin{equation}
$vista.$
\end{equation}Osmercadosreais\begin{equation}
—
\end{equation}ealgunsaisdoqueoutros\begin{equation}
\left( —,\right)
\end{equation}é\begin{equation}
\left( claro,\right)
\end{equation}nãosão\begin{equation}
$assim.$
\end{equation}Desdemuito\begin{equation}
\left( tempo,\right)
\end{equation}grafistaseperadorestécnicos\begin{equation}
$(technical$
\end{equation}\begin{equation}
$traders)$
\end{equation}sabemqueépreferívelacompanharosovimentosmostradosnosgráficosdeprazosmaiores\begin{equation}
$(diário$
\end{equation}e\begin{equation}
$semanal)$
\end{equation}aOprocedimentoé\begin{equation}
$simples:$
\end{equation}\begin{equation}
calculam - se
\end{equation}volatilidadessobreretornosdvárias\begin{equation}
$periodicidades:$
\end{equation}\begin{equation}
\left( diária,\right)
\end{equation}pelodesviopadrãodosretornosdeSentredia\begin{equation}
$tet+$
\end{equation}\begin{equation}
$1;$
\end{equation}de5em5\begin{equation}
\left( dias,\right)
\end{equation}tomandoosretornosentrediaste\begin{equation}
$+45;$
\end{equation}de10em10diasde20em20\begin{equation}
\left( dias,\right)
\end{equation}eassimpor\begin{equation}
$diante.$
\end{equation}Paracadavalordevolatilidadeobtido\begin{equation}
calcula - se
\end{equation}\begin{equation}
$g72/62?,$
\end{equation}e\begin{equation}
compara - se
\end{equation}comaperiodicidade\begin{equation}
$(5,$
\end{equation}\begin{equation}
\left( 10,\right)
\end{equation}20\begin{equation}
$etc.).$
\end{equation}Quantomaisosdoisvalores\begin{equation}
\left( diferirem,\right)
\end{equation}menosvalerãoashipótese\begin{equation}
$simplificadoras.$
\end{equation}EiumaexperiênciacomvaloresdeValePNentre1990e1994\begin{equation}
—
\end{equation}acoluna\begin{equation}
$"o$
\end{equation}\begin{equation}
medida”
\end{equation}mostraavolatilidadecalculadaemcada\begin{equation}
$periodicidade:$
\end{equation}\includegraphics[width=0.8\textwidth]{output/image_298png}\begin{equation}
«Ouerando
\end{equation}nVolatilidade6766\begin{equation}
$Opções:$
\end{equation}OperandoaVolatilidadOperartânciadosjurosacompanharosmovimentosno\begin{equation}
$7xtraday.$
\end{equation}Atribuemissoàpresençamaiordespeculaçãonocurto\begin{equation}
\left( prazo,\right)
\end{equation}especulaçãoestaqueéfiltradanosprazos\begin{equation}
$maio.$
\end{equation}\begin{equation}
$res.$
\end{equation}Nãosesabeatéquepontoestainterpretaçãoéválidaou\begin{equation}
\left( útil,\right)
\end{equation}masofatoqueosmercadosrealmentetendemasermenoserráticosnosprazosmailongosdoquenosmais\begin{equation}
$curtos.$
\end{equation}OutracríticaquesepodefazeraomodeloBlackeScholesé adasuposiçãoaxasdejuro\begin{equation}
\left( constantes,\right)
\end{equation}principalmenteno\begin{equation}
$Brasil.$
\end{equation}AcontecequetodaãosobreS\begin{equation}
\left( é,\right)
\end{equation}\begin{equation}
\left( indiretamente,\right)
\end{equation}umaopçãosobre\begin{equation}
$juros:$
\end{equation}seosjuros\begin{equation}
\left( caírem,\right)
\end{equation}oyrfuturodeStambém\begin{equation}
\left( cai,\right)
\end{equation}podendoatétransformaruma\begin{equation}
\left\lfloor{\frac{ca}{em}}\right\rfloor
\end{equation}\begin{equation}
$out-of-the-$
\end{equation}1ouumaputem\begin{equation}
$in-the-money.$
\end{equation}Se osjuros\begin{equation}
\left( sobem,\right)
\end{equation}ovalorfuturode5ém\begin{equation}
\left( sobe,\right)
\end{equation}podendotransformaruma\begin{equation}
\left\lfloor{\frac{ca}{em}}\right\rfloor
\end{equation}\begin{equation}
$in-the-money,$
\end{equation}ouumapuíem\begin{equation}
$ftihe-money.$
\end{equation}Emambosos\begin{equation}
\left( casos,\right)
\end{equation}odeslocamentodasopções\begin{equation}
deveu - se
\end{equation}asàvariaçãodos\begin{equation}
$juros.$
\end{equation}Issoquerdizerqueumaopçãopodeentraresairdinheiromovidaa\begin{equation}
$juros.$
\end{equation}SuponhamosooperadordodeltahedgequeproporcionouomodeloBlack\begin{equation}
$choles:$
\end{equation}eleestácompradoemAunidadesdoativo5evendidoemumadadede\begin{equation}
cal!
\end{equation}Seosjurossobemeesta\begin{equation}
\frac{cal}{entra}
\end{equation}no\begin{equation}
\left( dinheiro,\right)
\end{equation}porconseguinte\begin{equation}
\left( umenta,\right)
\end{equation}eooperadorprecisaadquirirmais\begin{equation}
5.0
\end{equation}Parapagarestacompra\begin{equation}
\left( onal,\right)
\end{equation}eletomadinheiroa\begin{equation}
$juros.$
\end{equation}\begin{equation}
\left( Se,\right)
\end{equation}apósrestabelecidooequilíbriodo\begin{equation}
\left( lee,\right)
\end{equation}osjurosvoltama\begin{equation}
\left( cair,\right)
\end{equation}oAdaopçãodamesmaforma\begin{equation}
\left( cai,\right)
\end{equation}ooperadordeve\begin{equation}
fazer - se
\end{equation}de\begin{equation}
\left( 5,\right)
\end{equation}epassarajurosodinheiroobtidoda\begin{equation}
$venda.$
\end{equation}Nessaaltaebaixa\begin{equation}
$;$
\end{equation}\begin{equation}
\left( juros,\right)
\end{equation}ooperadoracaboutendodetomardinheirocaroedardinheiro\begin{equation}
\left( to,\right)
\end{equation}exatamentedamesmaformaqueeletemquecomprar\begin{equation}
$$caro$
\end{equation}e\begin{equation}
$vendê-$
\end{equation}baratoparasatisfazerodeltahedgeemfacedavariaçõesde\begin{equation}
5.0
\end{equation}\begin{equation}
\left( Então,\right)
\end{equation}ailaçãodejurosprovocaumcustosemelhanteaocustododelta\begin{equation}
$hedge.$
\end{equation}\begin{equation}
\left( Se,\right)
\end{equation}nojá\begin{equation}
\left( vimos,\right)
\end{equation}éocustododeltaAedgeoresponsávelpelasopçõesteremoçoque\begin{equation}
\left( têm,\right)
\end{equation}ocustoadicionaldorhô\begin{equation}
\left( hedge,\right)
\end{equation}ouhedgecontra\begin{equation}
\left( juros,\right)
\end{equation}deveplicarumprêmioextraaseracrescidonopreçodas\begin{equation}
$opções.$
\end{equation}\begin{equation}
Segue - se
\end{equation}daíquetodaopçãosituadaemumambientedejurosvoláteisvecarregarumprêmioderiscodevidoàvolatilidadedos\begin{equation}
$juros.$
\end{equation}Ummodeloeleveemcontaestarealidadedeveincluirumsegundotermode\begin{equation}
$volatili-$
\end{equation}\begin{equation}
\left( de,\right)
\end{equation}Oyqueéavolatilidadedos\begin{equation}
$juros.$
\end{equation}ComojáfoivistonoiníciodaplanaçãosobremodeloBlacke\begin{equation}
\left( Scholes,\right)
\end{equation}istotransformariaVemumafunçãotrêsvariáveis\begin{equation}
$(5,$
\end{equation}\begin{equation}
£
\end{equation}e\begin{equation}
$7)$
\end{equation}emvezdeapenasduas\begin{equation}
$(S$
\end{equation}e\begin{equation}
$7),$
\end{equation}ecomplicariansivelmenteos\begin{equation}
$cálculos.$
\end{equation}Autocorrelaçãoeheterocedasticidade\begin{equation}
\left( Ultimamente,\right)
\end{equation}nenhumdosenfoquesbasicamentequalitativosdasanálisesanteriorestemmerecidomaisatençãodoqueatentativadereformularprocessodocaminhamentode\begin{equation}
\left( 5,\right)
\end{equation}derandômicoparaalgumtipodeprocessqueincluaapossibilidadedetendências\begin{equation}
$(/rends)$
\end{equation}nopreço\begin{equation}
\frac{e}{ou}
\end{equation}nasu\begin{equation}
$variância.$
\end{equation}Autocorrelaçãoéonomequesedáaofenômenoem\begin{equation}
\left( que,\right)
\end{equation}emumaséritemporal\begin{equation}
$(como$
\end{equation}umasériediáriaderetornosde\begin{equation}
$ativos),$
\end{equation}odado7mantémcorrelaçãocomosdados\begin{equation}
$(7—$
\end{equation}\begin{equation}
$1),$
\end{equation}\begin{equation}
5
\end{equation}\begin{equation}
$etc.$
\end{equation}Na\begin{equation}
\left( prática,\right)
\end{equation}correspondeadizerquoretornodeumpreçonumadatadependeemalgumgraudeseuretornondata\begin{equation}
\left( anterior,\right)
\end{equation}ou\begin{equation}
\left( que,\right)
\end{equation}aocontráriodoquepostulaa\begin{equation}
\left( aleatoriedade,\right)
\end{equation}existalgumamemórianocomportamentodos\begin{equation}
$preços.$
\end{equation}Em umcenáriod\begin{equation}
\left( autocorrelação,\right)
\end{equation}partedosdesviosdeixadeserexplicadapela\begin{equation}
\left( volatilidade,\right)
\end{equation}passaaserexplicadapelaautocorrelaçãoem\begin{equation}
$si.$
\end{equation}\begin{equation}
\left( Então,\right)
\end{equation}modelosdautocorrelaçãopodemfornecermedidasdevolatilidademenoresdoquemodelo\begin{equation}
\left( lognormal,\right)
\end{equation}sefordetectadatendênciade\begin{equation}
$preços.$
\end{equation}Heterocedasticidadeé acaracterísticadeumasérietemporalemquevariância\begin{equation}
$(em$
\end{equation}nosso\begin{equation}
\left( caso,\right)
\end{equation}\begin{equation}
\left( identicamente,\right)
\end{equation}a\begin{equation}
$volatilidade)$
\end{equation}dodado7mantenhcorrelaçãocomavariânciadosdados\begin{equation}
\left( 6,\right)
\end{equation}\begin{equation}
-1
\end{equation}\begin{equation}
$etc.$
\end{equation}Na\begin{equation}
\left( prática,\right)
\end{equation}correspondadizerqueavolatilidadepossa\begin{equation}
\left( tender,\right)
\end{equation}mudardinamicamentecomotempotalvez\begin{equation}
\left( continuamente,\right)
\end{equation}aocontráriodapremissatradicionaldequeavolatildadedevepermanecerconstanteduranteoprazoquese\begin{equation}
$analisa.$
\end{equation}ModelosARCHAspremissasdeautocorrelaçãoeheterocedasticidadesãoutilizadasenalgunsmodelosrelativamente\begin{equation}
\left( recentes,\right)
\end{equation}masdepoucaaplicaçãoaindan\begin{equation}
$Brasil.$
\end{equation}Sãomodelosde\begin{equation}
Auto - Regression
\end{equation}ConditionalHeteroscedasticityARCHeGeneral\begin{equation}
\left( ARCH - GARCH,\right)
\end{equation}dosquaisomaiorméritotemsidoestabelecernovasestimativasde\begin{equation}
\left( volatilidade,\right)
\end{equation}considerandoatendênciadavolat\begin{equation}
$dade.$
\end{equation}Paraaslongasopçõesdejuros\begin{equation}
\left( externos,\right)
\end{equation}incluindo\begin{equation}
\left( caps,\right)
\end{equation}foorsecollarestapodeseruma\begin{equation}
$necessidade.$
\end{equation}\includegraphics[width=0.8\textwidth]{output/image_300png}
\end{document}
